\documentclass{book}
\usepackage{amsmath}
\usepackage{amsthm}
\usepackage{amsfonts}
\usepackage{chngcntr}
\usepackage{bm}
\usepackage[usenames,dvipsnames]{xcolor}
\usepackage{tikz}
\usepackage{hyperref}

\hypersetup{
    colorlinks,
    linkcolor={red!30!black},
    citecolor={blue!50!black},
    urlcolor={blue!80!black}
}

%\renewcommand{\thechapter}{\Roman{chapter}}
%\makeatletter
%  \renewcommand\l@chapter{\@dottedtocline{2}{1.5em}{2.5em}}
%  \renewcommand\l@section{\@dottedtocline{2}{2.5em}{3.5em}}
%\makeatother

\counterwithout{section}{chapter}
\numberwithin{equation}{section}
\renewcommand{\theequation}{\textbf{\arabic{section}}.\arabic{equation}}

\theoremstyle{plain}
\newtheorem{thm}{Theorem}[section]
\newtheorem{lem}[thm]{Lemma}
\newtheorem{prop}{Proposition}
\newtheorem{cor}{Corollary}

\renewcommand{\thethm}{\arabic{thm}}

\theoremstyle{definition}
\newtheorem{defn}{Definition}[section]

\renewcommand{\thedefn}{\arabic{defn}}

\theoremstyle{remark}
\newtheorem{rem}{Remark}

\renewcommand{\therem}{}

\theoremstyle{remark}
\newtheorem{ex}{Example}

\renewcommand{\theex}{}

\begin{document}

% annotation macros
\newcommand{\repl}[2]{{\color{gray} [#1] }{\color{blue} #2}}
\newcommand{\add}[1]{{\color{blue} #1}}
\newcommand{\del}[1]{{\color{gray} [#1]}}
\newcommand{\note}[1]{{\color{OliveGreen}\small [\textbf{Comment.} #1]}}

\newcommand{\hl}[1]{{\color{red} #1}}

\tableofcontents


\part{NEWTONIAN MECHANICS}
\chapter{Experimental facts}

\section{The principles of relativity and determinancy}

\section{The galilean group and Newton's equations}

\section{Examples of mechanical systems}



\chapter{Investigation of the equations of motion}

\section{Systems with one degree of freedom}

\section{Systems with two degrees of freedom}

\section{Conservative force fields}

\section{Angular momentum}

\section{Investigation of motion in a central field}

\section{The motion of a point in three-space}

\section{Motions of a system of $n$ points}

\section{The method of similarity}

\part{LAGRANGIAN MECHANICS}

\chapter{Variational principles}

\section{Calculus of variations}

\section{Lagrange's equations}

\section{Legendre transformations}

\section{Hamilton's equations}

\section{Liouville's theorem}

\chapter{Lagrangian mechanics on manifolds}

\section{Holonomic constraints}

\section{Differentiable manifolds}

\section{Lagrangian dynamical systems}

\section{E. Noether's theorem}

\section{D'Alembert's principle}

\chapter{Oscillations}

\section{Linearization}

\section{Small oscillations}

\section{Behavior of characteristic frequencies}

\section{Parametric resonance}

\chapter{Rigid bodies}

\section{Motion in a moving coordinate system}

\section{Inertial forces and the Coriolis force}

\section{Rigid bodies}

\section{Euler's equations. Poinsot's description of the motion}

\section{Lagrange's top}

\section{Sleeping tops and fast tops}

\part{HAMILTONIAN MECHANICS}

\chapter{Differential forms}

\section{Exterior forms}

\section{Exterior multiplication}

\section{Differential forms}

\section{Integration of differential forms}

\section{Exterior differentiation}


\chapter{Symplectic manifolds}

\section{Symplectic structures on manifolds}

\subsection{A. Definition}



\begin{defn}
Let $M^{2n}$ be an even-dimensional differentiable manifold.
A \emph{symplectic structure} on $M^{2n}$ is a closed nondegenerate differential 2-form
$\omega^2$ on $M^{2n}$:
$$
d\omega^2 = 0
$$
and
$$
\forall \pmb \xi \ne 0
\quad
\exists \pmb \eta:
\;
\omega^2(\pmb \xi, \pmb \eta) \ne 0
\quad
(\pmb \xi, \pmb \eta \in TM_{\pmb x}).
$$
The pair $(M^{2n}, \omega^2)$ is called
a \emph{symplectic manifold}.
\end{defn}

\begin{ex}
$\omega^2 = \sum_{i = 1}^n dp_i \wedge dq_i$.
\end{ex}


\subsection{B. The cotangent bundle and its symplectic structure}



Let $V$ be an $n$-dimensional differentiable manifold.
%
A 1-form on the tangent space to $V$ at a point $\mathbf x$ is called
a \emph{cotangent vector} to $V$ at $\mathbf x$.

This vector space of cotangent vectors is denoted by $T^*V_{\mathbf x}$,
called the \emph{cotangent space} to $V$ at $\mathbf x$.

The union of the cotangent spaces to the manifold at all of its points
is called the \emph{cotangent bundle}, and denoted by $T^*V$.

The set $T^*V$ has a natural structure of a differentiable manifold of
dimension $2 n$.
(Note. Each point $\mathbf x \in V$ has a $n$-dimensional vectors space;
the dimensionality of $V$ is $n$).
%
A point of $T^*V$ is a 1-form on the tangent space to $V$ at some point of $V$.
%
If $\mathbf q$ is a choice of a local coordinates for points in $V$,
then such a form is given by its $n$ components $\mathbf p$.
%
Together, the $2n$ numbers $\mathbf p, \mathbf q$ form a collection of local coordinates
for points in $T^*V$.


There is a natural projection
$f: T^*V \rightarrow V$
(note: sending every 1-form on $TV_{\mathbf x}$
to the point $\mathbf x$).
%
The projection $f$ is differentiable and surjective
(note: each $\mathbf x \in V$ has at least one preimage $T^*V$).
%
The preimage of a point $\mathbf x \in V$ under $f$
is the cotangent space $T^*V_{\mathbf x}$.


\begin{thm}
The cotangent bundle $T^*V$ has a natural symplectic structure.
%
In the local coordinates described above,
the symplectic structure is given by the formula
$\omega^2 = d\mathbf p \wedge d\mathbf q = dp_1 \wedge dq_1 + \cdots + dp_n \wedge dq_n$.
\end{thm}

\begin{proof}
Define a distinguished 1-form on $T^*V$.
%
Let $\pmb \xi \in T(T^*V)p$ be a vector tangent to the cotangent bundle at
the point $p \in T^*V_{\mathbf x}$.
[Note: this means $p$ is a 1-form with coordinates $\mathbf p$,
and we can express $\pmb \xi$ in local coordinates as $(d\mathbf p, d\mathbf q)$.]
%
The derivative $f_{\mathbf x}: T(T^*V) \rightarrow TV$
of the natural projection $f: T^*V \rightarrow V$,
which takes $\pmb \xi$ to a vector $f_{\mathbf x} \pmb \xi$ ($d\mathbf q$)
tangent to $V$ at $\mathbf x$.
%
We define a 1-form $\omega^1$ on $T^*V$ by the relation
$\omega^1(\pmb \xi) \equiv p( f_{\mathbf x}\pmb \xi )$
[Note: remember that $p$ is a 1-form so it accepts
  a tangent vector $d\mathbf q = f_{\mathbf x} \pmb \xi$
as input and returns a number.]
%
In the local coordinates described above,
this form is
$\omega^1 = \mathbf p \cdot d\mathbf q$,
and
$\omega^2 = d\omega^1$ is nondegenerate.
\end{proof}

\begin{rem}
Let $L = L(\mathbf q, \dot{\mathbf q})$  be a Lagrangian.
%
Then, $\dot{\mathbf q}$ is a tangent vector;
%
$\mathbf p = \partial L/\partial \dot{\mathbf q}$
is a cotangent vector.
%
The $\mathbf p, \mathbf q$ phase space of the Lagrangian system
is the cotangent of the configuration manifold.
%
The above theorem shows that the phase space of
a mechanical problem has a natural symplectic
manifold structure.
\end{rem}


\subsection{C. Hamiltonian vector fields}



A Riemann structure on a manifold establishes an isomorphism between
the spaces of tangent vectors and 1-forms.
%
A symplectic structure establishes a similar isomorphism.


\begin{defn}
%
To each vector $\pmb \xi$
tangent to a symplectic manifold $(M^{2n}, \omega^2)$
at point $\mathbf x$, we associate a 1-form $\omega^1_{\pmb \xi}$
on $TM_{\mathbf x}$ by the formula
$$
\omega^1_{\pmb \xi}(\pmb \eta) = \omega^2(\pmb \eta, \pmb \xi),
\quad \forall \pmb \eta \in TM_{\mathbf x}.
$$
%
The correspondence $\pmb \xi \rightarrow \omega^1_{\mathbf \xi}$
is an isomorphism between $2n$-dimensional vector spaces of vectors
and those of 1-forms.
\end{defn}


\begin{ex}
%
For $n = 1$,
$\omega^2 = \pmb \eta \wedge \pmb \xi = \eta_1 \xi_2 - \eta_2 \xi_1$.
Then
$\omega^1_{\pmb \xi}(\pmb \eta) = (\pmb \eta, \mathbf J^{-1} \pmb \xi)$,
where $\mathbf J^{-1}$ means a rotation of $-\pi/2$.
\end{ex}

Thus, $\omega^1_{\pmb \xi}$ is represented by the vector $\mathbf J^{-1} \pmb \xi$.
%
We will denote by $I$ the isomorphism
$I: T*M_{\mathbf x} \rightarrow TM_{\mathbf x}$
(from 1-forms to tangent vectors,
$\omega^1_{\pmb \xi} \rightarrow \pmb \xi$).
%
Now let $H$ be a function on a symplectic manifold $M^{2n}$.
%
Then $dH$ is a differential 1-form on $M$ and
for every point there is a tangent vector to $M$ associated to it.
%
In this way, we obtain a vector field $IdH$ on $M$.


\begin{defn}
%
The vector field $IdH$ is called \emph{Hamiltonian vector field}.
%
$H$ is called the \emph{Hamiltonian function}.
%
In the above example,
the 1-form is represented by the vector
$\nabla H = (\partial H/\partial \mathbf p, \partial H/\partial q)^T$
to invert this vector to corresponding tangent vector we multiple it by $\mathbf J$
%
\[
\mathbf J \cdot \nabla H
=
\left(
  \begin{array}{ccc}
    0 & -1 \\
    1 & 0
  \end{array}
\right)
\left(
  \begin{array}{ccc}
    \frac{ \partial H }{ \partial \mathbf p } \\
    \frac{ \partial H }{ \partial \mathbf q }
  \end{array}
\right)
=
\left(
  \begin{array}{ccc}
    -\frac{ \partial H }{ \partial \mathbf q } \\
    \frac{ \partial H }{ \partial \mathbf p }
  \end{array}
\right)
=
\left(
  \begin{array}{ccc}
    \dot{\mathbf p} \\
    \dot{\mathbf q}
  \end{array}
\right).
\]
\end{defn}


\section{Hamiltonian phase flows and their integral invariants}

Liouville's theorem asserts that the phase flow
preserve the symplectic structure.
%
Poincar\'e found a whole series of differential forms
which are preserved by the hamiltonian flow.

\subsection{Hamiltonian phase flows preserve the symplectic structure}

Let $(M^{2n}, \omega^2)$ be a symplectic manifold and
$H: M^{2n} \rightarrow \Re$ a function.
Assume that the vector field $IdH$
corresponds to $H$ gives a 1-parameter group
of diffeomorphism: $g^t: M^{2n} \rightarrow M^{2n}$:
$$
\frac{d}{dt}\Big|_{t = 0}
= I dH(\mathbf x).
$$
This group $g^t$ is the hamiltonian phase flow with
hamiltonian function $H$.


\begin{thm}
A hamiltonian phase flow preserves the symplectic structure:
$$
(g^t)^* \omega^2 = \omega^2.
$$
\end{thm}

Notes on the notation.
$g^t$ applies to $\mathbf x$,
$(g^t)^*$ applies to $\omega$.

In the case $n = 1$, $M^{2n} = \Re^2$, this theorem says
that the phase flow $g^t$ preserves area (Liouville's theorem).
%
For the proof of this theorem, it is useful to introduce
the following notation.

Let $M$ be an arbitrary manifold,
$c$ a $k$-chain on $M$ and $g^t: M \rightarrow M$
a one-parameter family of differentiable mappings.
We will construct a $(k+1)$-chain $Jc$ on $M$,
which we will call the
\emph{track of the chain $c$ under the homotopy
$g^t$, $0 \le t \le \tau$.}

[Note. The ``homotopy'' means ``time evolution''.]

Let $(D, f, \mathrm{Or})$ be one of the cells in the chain $c$\footnote{
  $D$ is the domain, convex polygon in $\Re^k$,
  $f$ is a mapping from the euclidean space to the manifold $\Re^k \rightarrow M$,
  $\mathrm{Or}$ is the orientation.
  see page 184, \S 35.D.
},


\section{The Lie algebra of vector fields}

\section{The Lie algebra of hamiltonian functions}

\section{Symplectic geometry}

\section{Parametric resonance in systems with many degrees of freedom}

\section{A symplectic atlas}

\chapter{Canonical formalism}

\section{The integral invariant of Poincar\'e-Cartan}

\section{Applications of the integral invariant of Poincar\'e-Cartan}

\section{Huygens' principle}

\section{The Hamilton-Jacobi method for integrating Hamilton's canonical equations}

\section{Generating functions}

\chapter{Introduction to perturbation theory}

\section{Integrable systems}

\section{Action-angle variables}

\section{Averaging}

\section{Averaging of perturbations}

\appendix
\renewcommand{\thechapter}{\arabic{chapter}}

\chapter{Riemannian curvature}

\chapter{Geodesics of left-invariant metrics on Lie groups and
the hydrodynamics of ideal fluids}

\chapter{Symplectic structures on algebraic manifolds}

\chapter{Contact structures}

\chapter{Dynamical systems with symmetries}

\chapter{Normal forms of quadratic hamiltonians}

\chapter{Normal forms of hamiltonian systems near stationary points
and closed trajectories}

\chapter{Theory of perturbations of conditionally periodic motion,
and Kolmogorov's theorem}

\chapter{Poincar\'e's geometric theorem, its generalizations and
applications}

\chapter{Multiplicities of characteristic frequencies, and ellipoids
depending on parameters}

\chapter{Short wave asymptotics}

\chapter{Lagrangian singularities}

\chapter{The Korteweg-de Vries equation}

\chapter{Poisson structures}

\chapter{On elliptic coordinates}

\chapter{Singularities of ray systems}

\end{document}

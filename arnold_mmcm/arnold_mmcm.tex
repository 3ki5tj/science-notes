\documentclass{book}
\usepackage{amsmath}
\usepackage{bm}
\usepackage[usenames,dvipsnames]{xcolor}
\usepackage{tikz}
\usepackage{hyperref}

\hypersetup{
    colorlinks,
    linkcolor={red!30!black},
    citecolor={blue!50!black},
    urlcolor={blue!80!black}
}


\begin{document}

% annotation macros
\newcommand{\repl}[2]{{\color{gray} [#1] }{\color{blue} #2}}
\newcommand{\add}[1]{{\color{blue} #1}}
\newcommand{\del}[1]{{\color{gray} [#1]}}
\newcommand{\note}[1]{{\color{OliveGreen}\small [\textbf{Comment.} #1]}}

\newcommand{\hl}[1]{{\color{red} #1}}

\tableofcontents




\chapter{8. Symplectic manifolds}



\section{37. Symplectic structures on manifolds}



\subsection{A. Definition}



Definition.  Let $M^{2n}$ be an even-dimensional differentiable manifold.
A \emph{symplectic structure} on $M^{2n}$ is a closed nondegenerate differential 2-form
$\omega^2$ on $M^{2n}$:
$$
d\omega^2 = 0
$$
and
$$
\forall \pmb \xi \ne 0
\quad
\exists \pmb \eta:
\;
\omega^2(\pmb \xi, \pmb \eta) \ne 0
\quad
(\pmb \xi, \pmb \eta \in TM_{\pmb x}).
$$
The pair $(M^{2n}, \omega^2)$ is called
a \emph{symplectic manifold}.


Example.
$\omega^2 = \sum_{i = 1}^n dp_i \wedge dq_i$.



\subsection{B. The cotangent bundle and its symplectic structure}



Let $V$ be an $n$-dimensional differentiable manifold.
%
A 1-form on the tangent space to $V$ at a point $\mathbf x$ is called
a \emph{cotangent vector} to $V$ at $\mathbf x$.

This vector space of cotangent vectors is denoted by $T^*V_{\mathbf x}$,
called the \emph{cotangent space} to $V$ at $\mathbf x$.

The union of the cotangent spaces to the manifold at all of its points
is called the \emph{cotangent bundle}, and denoted by $T^*V$.

The set $T^*V$ has a natural structure of a differentiable manifold of
dimension $2 n$.
(Note. Each point $\mathbf x \in V$ has a $n$-dimensional vectors space;
the dimensionality of $V$ is $n$).
%
A point of $T^*V$ is a 1-form on the tangent space to $V$ at some point of $V$.
%
If $\mathbf q$ is a choice of a local coordinates for points in $V$,
then such a form is given by its $n$ components $\mathbf p$.
%
Together, the $2n$ numbers $\mathbf p, \mathbf q$ form a collection of local coordinates
for points in $T^*V$.


There is a natural projection
$f: T^*V \rightarrow V$
(note: sending every 1-form on $TV_{\mathbf x}$
to the point $\mathbf x$).
%
The projection $f$ is differentiable and surjective
(note: each $\mathbf x \in V$ has at least one preimage $T^*V$).
%
The preimage of a point $\mathbf x \in V$ under $f$
is the cotangent space $T^*V_{\mathbf x}$.


Theorem.
The cotangent bundle $T^*V$ has a natural symplectic structure.
%
In the local coordinates described above,
the symplectic structure is given by the formula
$\omega^2 = d\mathbf p \wedge d\mathbf q = dp_1 \wedge dq_1 + \cdots + dp_n \wedge dq_n$.

Proof.
Define a distinguished 1-form on $T^*V$.
%
Let $\pmb \xi \in T(T^*V)p$ be a vector tangent to the cotangent bundle at
the point $p \in T^*V_{\mathbf x}$.
[Note: this means $p$ is a 1-form with coordinates $\mathbf p$,
and we can express $\pmb \xi$ in local coordinates as $(d\mathbf p, d\mathbf q)$.]
%
The derivative $f_{\mathbf x}: T(T^*V) \rightarrow TV$
of the natural projection $f: T^*V \rightarrow V$,
which takes $\pmb \xi$ to a vector $f_{\mathbf x} \pmb \xi$ ($d\mathbf q$)
tangent to $V$ at $\mathbf x$.
%
We define a 1-form $\omega^1$ on $T^*V$ by the relation
$\omega^1(\pmb \xi) \equiv p( f_{\mathbf x}\pmb \xi )$
[Note: remember that $p$ is a 1-form so it accepts
  a tangent vector $d\mathbf q = f_{\mathbf x} \pmb \xi$
as input and returns a number.]
%
In the local coordinates described above,
this form is
$\omega^1 = \mathbf p \cdot d\mathbf q$,
and
$\omega^2 = d\omega^1$ is nondegenerate.


Remark.
Let $L = L(\mathbf q, \dot{\mathbf q})$  be a Lagrangian.
%
Then, $\dot{\mathbf q}$ is a tangent vector;
%
$\mathbf p = \partial L/\partial \dot{\mathbf q}$
is a cotangent vector.
%
The $\mathbf p, \mathbf q$ phase space of the Lagrangian system
is the cotangent of the configuration manifold.
%
The above theorem shows that the phase space of
a mechanical problem has a natural symplectic
manifold structure.



\subsection{C. Hamiltonian vector fields}



A Riemann structure on a manifold establishes an isomorphism between
the spaces of tangent vectors and 1-forms.
%
A symplectic structure establishes a similar isomorphism.


Definition.
%
To each vector $\pmb \xi$
tangent to a symplectic manifold $(M^{2n}, \omega^2)$
at point $\mathbf x$, we associate a 1-form $\omega^1_{\pmb \xi}$
on $TM_{\mathbf x}$ by the formula
$$
\omega^1_{\pmb \xi}(\pmb \eta) = \omega^2(\pmb \eta, \pmb \xi),
\quad \forall \pmb \eta \in TM_{\mathbf x}.
$$
%
The correspondence $\pmb \xi \rightarrow \omega^1_{\mathbf \xi}$
is an isomorphism between $2n$-dimensional vector spaces of vectors
and those of 1-forms.


Example.
%
For $n = 1$,
$\omega^2 = \pmb \eta \wedge \pmb \xi = \eta_1 \xi_2 - \eta_2 \xi_1$.
Then
$\omega^1_{\pmb \xi}(\pmb \eta) = (\pmb \eta, \mathbf J^{-1} \pmb \xi)$,
where $\mathbf J^{-1}$ means a rotation of $-\pi/2$.

Thus, $\omega^1_{\pmb \xi}$ is represented by the vector $\mathbf J^{-1} \pmb \xi$.
%
We will denote by $I$ the isomorphism
$I: T*M_{\mathbf x} \rightarrow TM_{\mathbf x}$
(from 1-forms to tangent vectors,
$\omega^1_{\pmb \xi} \rightarrow \pmb \xi$).
%
Now let $H$ be a function on a symplectic manifold $M^{2n}$.
%
Then $dH$ is a differential 1-form on $M$ and
for every point there is a tangent vector to $M$ associated to it.
%
In this way, we obtain a vector field $IdH$ on $M$.


Definition.
%
The vector field $IdH$ is called \emph{Hamiltonian vector field}.
%
$H$ is called the \emph{Hamiltonian function}.
%
In the above example,
the 1-form is represented by the vector
$\nabla H = (\partial H/\partial \mathbf p, \partial H/\partial q)^T$
to invert this vector to corresponding tangent vector we multiple it by $\mathbf J$
%
\[
\mathbf J \cdot \nabla H
=
\left(
  \begin{array}{ccc}
    0 & -1 \\
    1 & 0
  \end{array}
\right)
\left(
  \begin{array}{ccc}
    \frac{ \partial H }{ \partial \mathbf p } \\
    \frac{ \partial H }{ \partial \mathbf q }
  \end{array}
\right)
=
\left(
  \begin{array}{ccc}
    -\frac{ \partial H }{ \partial \mathbf q } \\
    \frac{ \partial H }{ \partial \mathbf p }
  \end{array}
\right)
=
\left(
  \begin{array}{ccc}
    \dot{\mathbf p} \\
    \dot{\mathbf q}
  \end{array}
\right).
\]





\end{document}

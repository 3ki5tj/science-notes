%\documentclass{article}
% options: aip, jcp, reprint, preprint
%\documentclass[preprint]{revtex4-1}
\documentclass[notitlepage, preprint]{revtex4-1}
%\documentclass[aip,jcp,reprint,superscriptaddress]{revtex4-1}

\usepackage{amsmath}
\usepackage{amsthm}
\usepackage{mathrsfs}
\usepackage{graphicx}
\usepackage{dcolumn}
\usepackage{bm}
%\usepackage{multirow}
\usepackage{hyperref}
\usepackage{tikz}
%\usepackage{setspace}

\tikzstyle{blackdot}=[circle,draw=black,fill=black,
                      inner sep=0pt,minimum size=1.5mm]
\tikzstyle{whitedot}=[circle,draw=black,fill=white,
                      inner sep=0pt,minimum size=1.5mm]

%\renewcommand{\theequation}{N\arabic{equation}}

\newtheorem{defn}{Definition}
\newtheorem{thrm}{Theorem}
\newtheorem{lemm}[thrm]{Lemma}
\newtheorem{prop}[thrm]{Proposition}

\newcommand{\vct}[1]{\mathbf{#1}}
\providecommand{\vr}{} % clear \vr
\renewcommand{\vr}{\vct{r}}
\newcommand{\vk}{\vct{k}}
\newcommand{\vR}{\vct{R}}
\newcommand{\dvk}{\frac{d\vk}{(2\pi)^D}}
\newcommand{\tdvk}{\tfrac{d\vk}{(2\pi)^D}}

% add a superscript ``ex''
\newcommand{\supex}[1]{ { { #1 }^{ \mathrm{ex} } } }
\newcommand{\Pex}{\supex{P}}
\newcommand{\Ptex}{P^{ \mathrm{ex} }_t}
\newcommand{\Fex}{\supex{F}}
\newcommand{\muex}{\supex{\mu}}
\newcommand{\muexxi}{\mu^{ \mathrm{ex} }_\xi}
\newcommand{\kex}{\supex{\kappa}}
\newcommand{\Chn}{\mathscr{C}}
%\newcommand{\Chn}{\mathcal{C}}
%\newcommand{\Chn}{\mathsf{C}}
\newcommand{\secref}[1]{Sec. \ref{#1}}

\newcommand{\llbra}{[\![}
\newcommand{\llket}{]\!]}





\begin{document}

%\title{On Attard's formula of chemical potential}

\begin{abstract}
We show that for a general approximate integral equation,
  Attard's formula of chemical potential
  yields a different value from
  the direct charging integral.
%
The difference is due to that the former
  assumes a special charging parameter that is
  proportional to the total correlation function,
%
  and the charging integral depends critically
  on the charging parameter for a general integral equation.
%
Concrete examples are given to illustrate the points.
\end{abstract}

\maketitle

\section{Introduction}



\subsection{Chemical potential and the charging parameter}



For a simple liquid,
the excess (non-ideal-gas) chemical potential $\muex$
can be computed from the charging integral
%
\begin{align}
  -\beta \, \muex
&=
  \rho \int_0^1 d\xi \int d\vr \, [-\beta \, \phi_\xi(r)] \, g_\xi(r)
  \label{eq:muintg}
\end{align}
%
Here,
$\phi(r)$ and $g(r)$ denote the pair potential
and the radial distribution function, respectively.
%
We are considering a system that contains a special particle 1,
whose interaction with other particles
is modified by the charging parameter, $\xi$.
%
The charging parameter, $\xi$, is 0
  when particle 1 is not interacting with other particles,
  i.e., $\phi_{\xi = 0}(r) = 0$;
it is 1.0
  when the interaction of particle 1 and other particles is assuming the full strength,
  i.e., $\phi_{\xi = 1}(r) = \phi(r)$.



Let us introduce the $f$-bond, which is defined as
\[
  f_\xi(r) \equiv \exp[-\beta \, \phi_\xi(r)] - 1.
\]
The $f$-bond vanishes at $\xi = 0$, and $f_{\xi = 1}(r) = f(r)$.
%
We can now rewrite Eq. \eqref{eq:muintg} as
\begin{align}
  -\beta \, \muex
&=
  \rho \int_0^1 d\xi \int d\vr \, (\partial_\xi f_\xi) \, y_\xi(r).
\label{eq:mudf}
\end{align}
%
where $y_\xi(r) \equiv \exp[-\beta \, \phi_\xi(r)] \, g_\xi(r)$
is the cavity distribution function.





\subsection{Graphic expansion}



It is well known that
the excess chemical potential of a homogeneous fluid
(which corresponds to the $\xi = 1$ case here)
has a well-defined graphic expansion,
which is the sum of all biconnected (irreducible)
1-root graphs weighted by the inverse symmetry number\cite{hansen}.
%
\begin{align*}
  \newcommand{\sz}{3.0mm}
  -\beta \, \muex
=
  \begin{tikzpicture}[baseline=-1.5*\sz]
    \node (r1) at (-\sz, -\sz) [whitedot]{};
    \node (r2) at ( \sz, -\sz) [blackdot]{};
    \draw (r1) -- (r2);
  \end{tikzpicture}
  +
  \frac{1}{2} \,
  \begin{tikzpicture}[baseline=-.5*\sz]
    \node (r1) at (-\sz, -\sz) [whitedot]{};
    \node (r2) at ( \sz, -\sz) [blackdot]{};
    \node (r3) at (   0,  \sz) [blackdot]{};
    \draw (r1) -- (r3) -- (r2) -- (r1);
  \end{tikzpicture}
  +
  \frac{1}{2} \,
  \begin{tikzpicture}[baseline=-.5*\sz]
    \node (r1) at (-\sz, -\sz) [whitedot]{};
    \node (r2) at ( \sz, -\sz) [blackdot]{};
    \node (r3) at ( \sz,  \sz) [blackdot]{};
    \node (r4) at (-\sz,  \sz) [blackdot]{};
    \draw (r1) -- (r4) -- (r3) -- (r2) -- (r1) (r2) -- (r4);
  \end{tikzpicture}
  +
  \frac{1}{2} \,
  \begin{tikzpicture}[baseline=-.5*\sz]
    \node (r1) at (-\sz, -\sz) [whitedot]{};
    \node (r2) at ( \sz, -\sz) [blackdot]{};
    \node (r3) at ( \sz,  \sz) [blackdot]{};
    \node (r4) at (-\sz,  \sz) [blackdot]{};
    \draw (r1) -- (r4) -- (r3) -- (r2) -- (r1) (r1) -- (r3);
  \end{tikzpicture}
  +
  \frac{1}{6} \,
  \begin{tikzpicture}[baseline=-.5*\sz]
    \node (r1) at (-\sz, -\sz) [whitedot]{};
    \node (r2) at ( \sz, -\sz) [blackdot]{};
    \node (r3) at ( \sz,  \sz) [blackdot]{};
    \node (r4) at (-\sz,  \sz) [blackdot]{};
    \draw (r1) -- (r4) -- (r3) -- (r2) -- (r4) (r1) -- (r3) (r1) -- (r2);
  \end{tikzpicture}
  + \cdots.
\end{align*}
%
This series can be adapted to the $\xi \ne 1$ case:
%
we simply replace every $f$-bond adjacent to field point 1
by the $f_\xi$-bond,
the result is denoted by $-\beta \, \muexxi$:
%
\begin{align}
  -\beta \, \muexxi
=
  \newcommand{\sz}{3.6mm}
  \newcommand{\isep}{0.1*\sz}
  \begin{tikzpicture}[baseline=-1.5*\sz]
    \node (r1) at (-\sz, -\sz) [whitedot]{};
    \node (r2) at ( \sz, -\sz) [blackdot]{}
        edge[thick, inner sep=\isep] node[anchor=north] {\scriptsize$\xi$} (r1);
  \end{tikzpicture}
  +
  \frac{1}{2} \,
  \begin{tikzpicture}[baseline=-0.5*\sz]
    \node (r1) at (-\sz, -\sz) [whitedot]{};
    \node (r2) at ( \sz, -\sz) [blackdot]{}
        edge[thick, inner sep=\isep] node[anchor=north] {\scriptsize$\xi$} (r1);
    \node (r3) at (   0,  \sz) [blackdot]{}
        edge[thick, inner sep=\isep] node[anchor=-30] {\scriptsize$\xi$} (r1);
    \draw (r3) -- (r2);
  \end{tikzpicture}
  +
  \frac{1}{2} \,
  \begin{tikzpicture}[baseline=-0.5*\sz]
    \node (r1) at (-\sz, -\sz) [whitedot]{};
    \node (r2) at ( \sz, -\sz) [blackdot]{}
        edge[thick, inner sep=\isep] node[anchor=north] {\scriptsize$\xi$} (r1);
    \node (r3) at ( \sz,  \sz) [blackdot]{};
    \node (r4) at (-\sz,  \sz) [blackdot]{}
        edge[thick, inner sep=\isep] node[anchor=east] {\scriptsize$\xi$} (r1);
    \draw (r1) -- (r4) -- (r3) -- (r2) -- (r1) (r2) -- (r4);
  \end{tikzpicture}
  +
  \frac{1}{2} \,
  \begin{tikzpicture}[baseline=-0.5*\sz]
    \node (r1) at (-\sz, -\sz) [whitedot]{};
    \node (r2) at ( \sz, -\sz) [blackdot]{}
        edge[thick, inner sep=\isep] node[anchor=north] {\scriptsize$\xi$} (r1);
    \node (r3) at ( \sz,  \sz) [blackdot]{}
        edge[thick, inner sep=\isep] node[anchor=-45] {\scriptsize$\xi$} (r1);
    \node (r4) at (-\sz,  \sz) [blackdot]{}
        edge[thick, inner sep=\isep] node[anchor=east] {\scriptsize$\xi$} (r1);
    \draw (r1) -- (r4) -- (r3) -- (r2) -- (r1) (r1) -- (r3);
  \end{tikzpicture}
  +
  \frac{1}{6} \,
  \begin{tikzpicture}[baseline=-0.5*\sz]
    \node (r1) at (-\sz, -\sz) [whitedot]{};
    \node (r2) at ( \sz, -\sz) [blackdot]{}
        edge[thick, inner sep=\isep] node[anchor=north] {\scriptsize$\xi$} (r1);
    \node (r3) at ( \sz,  \sz) [blackdot]{}
        edge[thick, inner sep=2*\isep] node[anchor=15] {\scriptsize$\xi$} (r1);
    \node (r4) at (-\sz,  \sz) [blackdot]{}
        edge[thick, inner sep=\isep] node[anchor=east] {\scriptsize$\xi$} (r1);
    \draw (r1) -- (r4) -- (r3) -- (r2) -- (r4) (r1) -- (r3) (r1) -- (r2);
  \end{tikzpicture}
  + \cdots.
  \label{eq:muexxi}
\end{align}





\subsection{Derivative of the chemical potential as a total differential}



From Eq. \eqref{eq:muexxi}, it follows that the derivative
\begin{equation}
  \partial_\xi (-\beta \, \muexxi)
  = \rho \int d\vr (\partial_\xi f_\xi) \, y_\xi(r).
\label{eq:dmu}
\end{equation}
also has a well-defined graphic representation.
%
This representation involves graphs with $\partial_\xi f_\xi$-bonds
attached to particle 1.

Further, the right-hand side of Eq. \eqref{eq:dmu}
must be a total differential of $\xi$.
%
This statement, however, is valid under the assumption that
$y_\xi(r)$ is exact.
%
It is not necessarily so for a general integral equation,
e.g., the Percus-Yevick (PY) equation\cite{percusyevick},
which yields an approximate $y_\xi(r)$ [or equivalently, $g_\xi(r)$].





\subsection{Specialness of the hypernetted-chain equation}



Nonetheless,
  the hypernetted-chain (HNC) equation\cite{
  morita1958, morita1959, morita1960, vanleeuwen1959, green1960, verlet1960, rushbrooke1960}
  is a lucky exception:
the right-hand side of Eq. \eqref{eq:dmu} is a total differential of $\xi$,
  although it includes fewer graphs than the exact sum.
%
The specialness of the HNC equation is related to the fact that
%
we have a succinct and unique formula\cite{morita1960, singer1985}
for the excess chemical potential:
\begin{equation}
  -\beta \, \muex
=
  \rho \int
  \left[ c(r) - \tfrac{1}{2} h(r) t(r) \right] \, d\vr.
\label{eq:muhnc}
\end{equation}
%
This formula can be derived from analyzing graphs included in the HNC equation,
as it is done originally\cite{morita1960}.
%
It can also be derived from algebraically transforming Eq. \eqref{eq:dmu}\cite{singer1985}
(see also Appendix \ref{sec:muhncB}).
%
The latter is possible because the right-hand side is a total differential.





\subsection{General integral equation and Attard's formula}



For a general (e.g., PY) integral equation, however,
there is usually no well-defined graphic sum for
$-\beta \, \muexxi$,
although the sum for $y_\xi(r)$ or $g_\xi(r)$ exists.
%
Consequently,
the right-hand side of Eq. \eqref{eq:dmu}
is not a total differential of $\xi$,
and the integral
depends on the detailed definition of $\xi$.




We will show below that this $\xi$-dependence
causes Attard's formula\cite{attard1991}
to produce seemingly self-contradictory results of $\muex$
for the PY equation
(as a result of different definitions of $\xi$).
%
This phenomenon can be viewed as a neat illustration that
integrating Eq. \eqref{eq:dmu} does not yield a well-defined $\xi$-independent $\muexxi$.



Conversely,
if we postulate a $\beta \, \muexxi$ with a well-defined graphic sum,
its derivative with respect to $\xi$
will not be given by Eq. \eqref{eq:dmu}.
%
Thus, the contradiction in Attard's formula,
although inconvenient, is just as inevitable as
the well-known nuisance that the virial and compressibility routes
produce different values of pressure.





\section{Evaluation of the charging integral}



\subsection{Charging parameter proportional to $f$-bond}



Let us first evaluate Eq. \eqref{eq:mudf}.
%
The most convenient $\xi$ to do so is one that satisfies
\begin{equation}
  f_\xi(r) = f(r) \, \xi.
  \label{eq:fxi}
\end{equation}
Now we can write
\begin{equation}
  y_\xi(r) = \sum_{n = 0}^\infty y^{(n)}(r) \, \xi^n,
\label{eq:yxi}
\end{equation}
where $y^{(n)}(r)$ collects all graphs
with $n$ $f$-bonds adjacent to field point 1.
%
Using Eqs. \eqref{eq:fxi} and \eqref{eq:yxi} in Eq. \eqref{eq:mudf} yields
\begin{equation}
  -\beta \, \muex
=
  \rho \sum_{n = 0}^\infty
  \frac{1}{n+1}
  \int d\vr \, f(r) \, y^{(n)}(r).
\label{eq:mu_fyseries}
\end{equation}



\subsection{$\xi$-dependence of the graphic sum of the chemical potential}



We should point out a weakness of Eq. \eqref{eq:mu_fyseries}.
%
That is, for an arbitrary $y_\xi(r)$,
the graphic sum generated by Eq. \eqref{eq:mu_fyseries}
depends on the choice of $\xi$.
%
This can be seen from the example below.

For simplicity, consider $y_\xi(r) = \exp \tau_\xi(r)$,
where $\tau_\xi(r) =
  \newcommand{\sz}{2.5mm}
  \newcommand{\isep}{0.2*\sz}
  \begin{tikzpicture}[baseline=-0.5*\sz]
    \node (r1) at (-1.2*\sz, -\sz) [whitedot]{};
    \node (r2) at ( 1.2*\sz, -\sz) [whitedot]{};
    \node (r3) at (       0,  \sz) [blackdot]{}
        edge[thick, inner sep=\isep] node[anchor=-30] {\scriptsize$\xi$} (r1);
    \draw (r3) -- (r2);
  \end{tikzpicture}$.
Then,
\[
  \newcommand{\sz}{4.0mm}
  \newcommand{\isep}{0.2*\sz}
  y_\xi(r)
=
  1
+
  \begin{tikzpicture}[baseline=-0.5*\sz]
    \node (r1) at (-1.2*\sz, -\sz) [whitedot]{};
    \node (r2) at ( 1.2*\sz, -\sz) [whitedot]{};
    \node (r3) at (       0,  \sz) [blackdot]{}
        edge[thick, inner sep=\isep] node[anchor=-30] {\scriptsize$\xi$} (r1);
    \draw (r3) -- (r2);
  \end{tikzpicture}
+
  \frac{1}{2} \;
  \begin{tikzpicture}[baseline=-0.5*\sz]
    \node (r1) at (-\sz, -\sz) [whitedot]{};
    \node (r2) at ( \sz, -\sz) [whitedot]{};
    \node (r3) at (-\sz,  \sz) [blackdot]{}
        edge[thick, inner sep=\isep] node[anchor=0] {\scriptsize$\xi$} (r1);
    \node (r4) at ( \sz,  \sz) [blackdot]{}
        edge[thick, inner sep=2*\isep] node[anchor=10] {\scriptsize$\xi$} (r1);
    \draw (r3) -- (r2) -- (r4);
  \end{tikzpicture}
%+
%  \frac{1}{6}
%  \begin{tikzpicture}[baseline=-\sz]
%    \node (r1) at (-\sz, -\sz) [whitedot]{};
%    \node (r2) at ( \sz, -\sz) [whitedot]{};
%    \node (r3) at (-\sz,  \sz) [blackdot]{}
%        edge[thick, inner sep=\isep] node[anchor=0] {\scriptsize$\xi$} (r1);
%    \node (r4) at ( \sz,  \sz) [blackdot]{}
%        edge[thick, inner sep=2*\isep] node[anchor=10] {\scriptsize$\xi$} (r1);
%    \node (r5) at (   0, -2.0*\sz) [blackdot]{}
%        edge[thick, inner sep=\isep] node[anchor=45] {\scriptsize$\xi$} (r1);
%    \draw (r3) -- (r2) -- (r4) (r2) -- (r5);
%  \end{tikzpicture}
+ \cdots,
\]
%
%
%
and Eq. \eqref{eq:mu_fyseries} becomes
\begin{equation}
  \newcommand{\sz}{4.0mm}
  \newcommand{\isep}{0.2*\sz}
  -\beta \, \muex
=
  \begin{tikzpicture}[baseline=-1.5*\sz]
    \node (r1) at (-\sz, -\sz) [whitedot]{};
    \node (r2) at ( \sz, -\sz) [blackdot]{};
    \draw (r1) -- (r2);
  \end{tikzpicture}
+
  \frac{1}{2} \;
  \begin{tikzpicture}[baseline=-0.5*\sz]
    \node (r1) at (-1.2*\sz, -\sz) [whitedot]{};
    \node (r2) at ( 1.2*\sz, -\sz) [blackdot]{};
    \node (r3) at (       0,  \sz) [blackdot]{};
    \draw (r1) -- (r3) -- (r2) -- (r1);
  \end{tikzpicture}
+
  \frac{1}{6} \;
  \begin{tikzpicture}[baseline=-0.5*\sz]
    \node (r1) at (-\sz, -\sz) [whitedot]{};
    \node (r2) at ( \sz, -\sz) [blackdot]{};
    \node (r3) at (-\sz,  \sz) [blackdot]{};
    \node (r4) at ( \sz,  \sz) [blackdot]{};
    \draw (r1) -- (r3) -- (r2) -- (r4) -- (r1) -- (r2);
  \end{tikzpicture}
%+
%  \frac{1}{24}
%  \begin{tikzpicture}[baseline=-\sz]
%    \node (r1) at (-\sz, -\sz) [whitedot]{};
%    \node (r2) at ( \sz, -\sz) [blackdot]{};
%    \node (r3) at (-\sz,  \sz) [blackdot]{};
%    \node (r4) at ( \sz,  \sz) [blackdot]{};
%    \node (r5) at (   0, -2.2*\sz) [blackdot]{};
%    \draw (r1) -- (r3) -- (r2) -- (r4) -- (r1) -- (r5) -- (r2) -- (r1);
%  \end{tikzpicture}
+ \cdots.
\label{eq:muex_model}
\end{equation}



If Eq. \eqref{eq:muex_model} \emph{were} a $\xi$-independent graphic sum,
the corresponding $\muexxi$ should be
\begin{equation}
  \newcommand{\sz}{4.0mm}
  \newcommand{\isep}{0.2*\sz}
  -\beta \, \muexxi
=
  \begin{tikzpicture}[baseline=-1.25*\sz]
    \node (r1) at (-\sz, -\sz) [whitedot]{};
    \node (r2) at ( \sz, -\sz) [blackdot]{}
        edge[thick, inner sep=\isep] node[anchor=north] {\scriptsize$\xi$} (r1);
  \end{tikzpicture}
+
  \frac{1}{2} \;
  \begin{tikzpicture}[baseline=-0.5*\sz]
    \node (r1) at (-1.2*\sz, -\sz) [whitedot]{};
    \node (r2) at ( 1.2*\sz, -\sz) [blackdot]{}
        edge[thick, inner sep=\isep] node[anchor=90] {\scriptsize$\xi$} (r1);
    \node (r3) at (       0,  \sz) [blackdot]{}
        edge[thick, inner sep=\isep] node[anchor=-30] {\scriptsize$\xi$} (r1);
    \draw (r3) -- (r2);
  \end{tikzpicture}
+
  \frac{1}{6} \;
  \begin{tikzpicture}[baseline=-0.5*\sz]
    \node (r1) at (-\sz, -\sz) [whitedot]{};
    \node (r2) at ( \sz, -\sz) [blackdot]{}
        edge[thick, inner sep=\isep] node[anchor=90] {\scriptsize$\xi$} (r1);
    \node (r3) at (-\sz,  \sz) [blackdot]{}
        edge[thick, inner sep=\isep] node[anchor=0] {\scriptsize$\xi$} (r1);
    \node (r4) at ( \sz,  \sz) [blackdot]{}
        edge[thick, inner sep=2*\isep] node[anchor=10] {\scriptsize$\xi$} (r1);
    \draw (r3) -- (r2) -- (r4);
  \end{tikzpicture}
%+
%  \frac{1}{24}
%  \begin{tikzpicture}[baseline=-\sz]
%    \node (r1) at (-\sz, -\sz) [whitedot]{};
%    \node (r2) at ( \sz, -\sz) [blackdot]{}
%        edge[thick, inner sep=0.2*\isep] node[anchor=90] {\scriptsize$\xi$} (r1);
%    \node (r3) at (-\sz,  \sz) [blackdot]{}
%        edge[thick, inner sep=\isep] node[anchor=0] {\scriptsize$\xi$} (r1);
%    \node (r4) at ( \sz,  \sz) [blackdot]{}
%        edge[thick, inner sep=2*\isep] node[anchor=10] {\scriptsize$\xi$} (r1);
%    \node (r5) at (   0, -2.2*\sz) [blackdot]{}
%        edge[thick, inner sep=\isep] node[anchor=45] {\scriptsize$\xi$} (r1);
%    \draw (r3) -- (r2) -- (r4) (r2) -- (r5);
%  \end{tikzpicture}
+ \cdots,
\label{eq:muexxi_model}
\end{equation}
%
which should hold for an \emph{arbitrary} $\xi$
(not just one that is proportional to $f_\xi$).

But differentiating Eq. \eqref{eq:muexxi_model} yields
\[
  \newcommand{\sz}{4.0mm}
  \newcommand{\isep}{0.2*\sz}
  -\partial_\xi ( \beta \, \muexxi )
=
  \begin{tikzpicture}[baseline=-1.25*\sz]
    \node (r1) at (-\sz, -\sz) [whitedot]{};
    \node (r2) at ( \sz, -\sz) [blackdot]{}
        edge[thick, dashed] (r1);
  \end{tikzpicture}
+
  \frac{1}{2} \;
  \begin{tikzpicture}[baseline=-0.5*\sz]
    \node (r1) at (-1.2*\sz, -\sz) [whitedot]{};
    \node (r2) at ( 1.2*\sz, -\sz) [blackdot]{}
        edge[thick, dashed] (r1);
    \node (r3) at (       0,  \sz) [blackdot]{}
        edge[thick, inner sep=\isep] node[anchor=-30] {\scriptsize$\xi$} (r1);
    \draw (r3) -- (r2);
  \end{tikzpicture}
+
  \frac{1}{6} \;
  \begin{tikzpicture}[baseline=-0.5*\sz]
    \node (r1) at (-\sz, -\sz) [whitedot]{};
    \node (r2) at ( \sz, -\sz) [blackdot]{}
        edge[thick, dashed] (r1);
    \node (r3) at (-\sz,  \sz) [blackdot]{}
        edge[thick, inner sep=\isep] node[anchor=0] {\scriptsize$\xi$} (r1);
    \node (r4) at ( \sz,  \sz) [blackdot]{}
        edge[thick, inner sep=2*\isep] node[anchor=10] {\scriptsize$\xi$} (r1);
    \draw (r3) -- (r2) -- (r4);
  \end{tikzpicture}
+
  \frac{1}{3} \;
  \begin{tikzpicture}[baseline=-0.5*\sz]
    \node (r1) at (-\sz, -\sz) [whitedot]{};
    \node (r2) at ( \sz, -\sz) [blackdot]{}
        edge[thick, inner sep=\isep] node[anchor=90] {\scriptsize$\xi$} (r1);
    \node (r3) at (-\sz,  \sz) [blackdot]{}
        edge[thick, inner sep=\isep] node[anchor=0] {\scriptsize$\xi$} (r1);
    \node (r4) at ( \sz,  \sz) [blackdot]{}
        edge[thick, dashed] (r1);
    \draw (r3) -- (r2) -- (r4);
  \end{tikzpicture}
%+
%  \frac{1}{24}
%  \begin{tikzpicture}[baseline=-\sz]
%    \node (r1) at (-\sz, -\sz) [whitedot]{};
%    \node (r2) at ( \sz, -\sz) [blackdot]{}
%        edge[thick, dashed] (r1);
%    \node (r3) at (-\sz,  \sz) [blackdot]{}
%        edge[thick, inner sep=\isep] node[anchor=0] {\scriptsize$\xi$} (r1);
%    \node (r4) at ( \sz,  \sz) [blackdot]{}
%        edge[thick, inner sep=2*\isep] node[anchor=10] {\scriptsize$\xi$} (r1);
%    \node (r5) at (   0, -2*\sz) [blackdot]{}
%        edge[thick, inner sep=\isep] node[anchor=45] {\scriptsize$\xi$} (r1);
%    \draw (r3) -- (r2) -- (r4) (r2) -- (r5);
%  \end{tikzpicture}
%+
%  \frac{1}{8}
%  \begin{tikzpicture}[baseline=-\sz]
%    \node (r1) at (-\sz, -\sz) [whitedot]{};
%    \node (r2) at ( \sz, -\sz) [blackdot]{}
%        edge[thick, inner sep=0.2*\isep] node[anchor=90] {\scriptsize$\xi$} (r1);
%    \node (r3) at (-\sz,  \sz) [blackdot]{}
%        edge[thick, inner sep=\isep] node[anchor=0] {\scriptsize$\xi$} (r1);
%    \node (r4) at ( \sz,  \sz) [blackdot]{}
%        edge[thick, inner sep=2*\isep] node[anchor=10] {\scriptsize$\xi$} (r1);
%    \node (r5) at (   0, -2*\sz) [blackdot]{}
%        edge[thick, dashed] (r1);
%    \draw (r3) -- (r2) -- (r4) (r2) -- (r5);
%  \end{tikzpicture}
+ \cdots,
\]
where a dashed line represents a $\partial_\xi f_\xi$-bond.
%
Clearly, this expression is not equivalent to the integral defined in Eq. \eqref{eq:dmu}
because, for a general $\xi$,
\[
  \newcommand{\sz}{4.0mm}
  \newcommand{\isep}{0.2*\sz}
  \begin{tikzpicture}[baseline=-0.5*\sz]
    \node (r1) at (-\sz, -\sz) [whitedot]{};
    \node (r2) at ( \sz, -\sz) [blackdot]{}
        edge[thick, dashed] (r1);
    \node (r3) at (-\sz,  \sz) [blackdot]{}
        edge[thick, inner sep=\isep] node[anchor=0] {\scriptsize$\xi$} (r1);
    \node (r4) at ( \sz,  \sz) [blackdot]{}
        edge[thick, inner sep=2*\isep] node[anchor=10] {\scriptsize$\xi$} (r1);
    \draw (r3) -- (r2) -- (r4);
  \end{tikzpicture}
\; \ne \;
  \begin{tikzpicture}[baseline=-0.5*\sz]
    \node (r1) at (-\sz, -\sz) [whitedot]{};
    \node (r2) at ( \sz, -\sz) [blackdot]{}
        edge[thick, inner sep=\isep] node[anchor=90] {\scriptsize$\xi$} (r1);
    \node (r3) at (-\sz,  \sz) [blackdot]{}
        edge[thick, inner sep=\isep] node[anchor=0] {\scriptsize$\xi$} (r1);
    \node (r4) at ( \sz,  \sz) [blackdot]{}
        edge[thick, dashed] (r1);
    \draw (r3) -- (r2) -- (r4);
  \end{tikzpicture}.
\]

The above example illustrates that
the graphs included in $y_\xi(r)$ has to
satisfy certain constraints
in order for Eq. \eqref{eq:mudf}
to produce a $\xi$-independent graphic sum.
%
The constraints are satisfied by the graphic sum of the exact $g(r)$,
and the corresponding partial sum in the HNC case.
%
However, they are not satisfied in a general integral equation
such as the PY one.
%
As a result, the chemical potential is $\xi$-dependent
in the latter cases.





\subsection{Charging parameter proportional to $h$-bond}



On the other hand,
one can algebraically transform Eq. \eqref{eq:mudf}
to (see Appendix \ref{sec:muhncB})
%
\begin{equation}
  -\beta \, \muex
=
  \rho \int d\vr
  \left[ c(r) - B(r) - \tfrac{1}{2} h(r) \, t(r)
  - \int_0^1 d\xi \, h_\xi(r) \, \partial_\xi B_\xi(r) \right],
\label{eq:muhncB}
\end{equation}
where $B(r) \equiv \log y(r) - t(r)$ is the bridge function.

This expression is most readily evaluated
by choosing a \emph{different} $\xi$ such that
\begin{equation}
  h_{\xi}(r) = h(r) \, \xi.
  \label{eq:hxi}
\end{equation}
We assume that such a $\xi$ exists
as $h(r) \approx f(r)$ in the low density limit.

Then, by writing
\begin{equation}
  B_{\xi}(r) = \sum_{n = 2}^\infty B^{(n)}_{\xi}(r) \, {\xi}^{\, n},
\label{eq:Bxi}
\end{equation}
we get\cite{attard1991}
\begin{equation}
  -\beta \, \muex
=
  \rho \int d\vr
  \left[ c(r) - B(r) - \frac{1}{2} h(r) \, t(r)
   - \sum_{n = 2}^\infty \, \frac{n}{n+1} h(r) \, B^{(n)}(r) \right].
\label{eq:muhncB_hBseries}
\end{equation}
This is Attard's formula for the excess chemical potential.




\subsection{Different charging parameter, different result}



We emphasize that although
Eq. \eqref{eq:muhncB} is equivalent to Eq. \eqref{eq:mudf},
Eq. \eqref{eq:muhncB_hBseries} is valid
only for the special $\xi$ defined in Eq. \eqref{eq:hxi}.
%
If we use the $\xi$ defined in Eq. \eqref{eq:fxi}
to evaluate Eq. \eqref{eq:muhncB},
the result would be different.
%
Below, we give an example to illustrate the point.

Let us consider the integral of
\begin{equation}
  \int_0^1 d\xi \int d\vr \, h_\xi(r) \, \partial_\xi B_\xi(r),
  \label{eq:hBint}
\end{equation}
for two fictitious $h_\xi(r)$ and $B_\xi(r)$:
\begin{align*}
  h_\xi(r)
&=
  \newcommand{\sz}{3.0mm}
  \newcommand{\isep}{0.2*\sz}
  \begin{tikzpicture}[baseline=-1.25*\sz]
    \node (r1) at (-\sz, -\sz) [whitedot]{};
    \node (r2) at ( \sz, -\sz) [whitedot]{}
        edge[thick, inner sep=\isep] node[anchor=90] {\scriptsize$\xi$} (r1);
  \end{tikzpicture}
  +
  \begin{tikzpicture}[baseline=-0.5*\sz]
    \node (r1) at (-1.2*\sz, -\sz) [whitedot]{};
    \node (r2) at ( 1.2*\sz, -\sz) [whitedot]{}
        edge[thick, inner sep=\isep, blue!50!black] node[anchor=90] {\scriptsize$\xi$} (r1);
    \node (r3) at (   0*\sz,  \sz) [blackdot]{}
        edge[thick, inner sep=\isep] node[anchor=0] {\scriptsize$\xi$} (r1)
        edge[thick, inner sep=\isep, dotted, red!70!black] (r2);
  \end{tikzpicture}\;,
\\
  B_\xi(r)
&=
  \newcommand{\sz}{5.0mm}
  \newcommand{\isep}{0.05*\sz}
  -\frac{1}{2} \;
  \begin{tikzpicture}[baseline=-0.1*\sz]
    \node (r1) at (-3.0*\sz,   0*\sz) [whitedot]{};
    \node (r2) at ( 1.5*\sz,   0*\sz) [whitedot]{};
    \node (r3) at (   0*\sz, 1.5*\sz) [blackdot]{}
        edge[thick, inner sep=\isep] node[anchor=-80] {\footnotesize$h_\xi$} (r1)
        edge[thick, inner sep=\isep, green!30!black]
          node[anchor=-135] {\footnotesize$c$} (r2);
    \node (r4) at (   0*\sz,-1.5*\sz) [blackdot]{}
        edge[thick, inner sep=\isep] node[anchor=80] {\footnotesize$h_\xi$} (r1)
        edge[thick, inner sep=\isep, green!30!black]
          node[anchor=135] {\footnotesize$c$} (r2);
  \end{tikzpicture}\;.
\end{align*}
Note that here we have assumed
$B_\xi(r) = -t_\xi^2(r)/2$,
with $t_\xi(r)$ satisfying the Ornstein-Zernike (OZ) relation:
\begin{equation}
  t_\xi(r) = \rho (h_\xi * c)(r).
  \label{eq:ozxi}
\end{equation}
%
Then
\begin{align}
I
\equiv
\rho \int_0^1 d\xi
  \int d\vr \, h_\xi(r) \, \partial_\xi B_\xi(r)
=
  \newcommand{\sz}{5.0mm}
  \newcommand{\isep}{0.05*\sz}
  - \int_0^1 d\xi \;
  \begin{tikzpicture}[baseline=-0.1*\sz]
    \node (r1) at (-3.0*\sz,   0*\sz) [whitedot]{};
    \node (r2) at ( 2.0*\sz,   0*\sz) [blackdot]{}
        edge[thick, inner sep=\isep]
          node[anchor=90] {\footnotesize$h_\xi$} (r1);
    \node (r3) at (   0*\sz, 1.5*\sz) [blackdot]{}
        edge[thick, inner sep=3*\isep]
          node[anchor=-80] {\footnotesize$\partial_\xi h_\xi$} (r1)
        edge[thick, inner sep=\isep, green!30!black]
          node[anchor=-135] {\footnotesize$c$} (r2);
    \node (r4) at (   0*\sz,-1.5*\sz) [blackdot]{}
        edge[thick, inner sep=\isep]
          node[anchor=80] {\footnotesize$h_\xi$} (r1)
        edge[thick, inner sep=\isep, green!30!black]
          node[anchor=135] {\footnotesize$c$} (r2);
  \end{tikzpicture}\;.
  \label{eq:hBintg_example}
\end{align}


Now let us evaluate Eq. \eqref{eq:hBintg_example} by
the $\xi$ defined by Eq. \eqref{eq:hxi},
which satisfies $\partial_\xi h_\xi = h$.
%
So
\begin{align}
  I^{(h)}
&=
  \newcommand{\sz}{5.0mm}
  \newcommand{\isep}{0.05*\sz}
  -\frac{1}{3} \;
  \begin{tikzpicture}[baseline=-0.1*\sz]
    \node (r1) at (-3.0*\sz,   0*\sz) [whitedot]{};
    \node (r2) at ( 2.0*\sz,   0*\sz) [blackdot]{}
        edge[thick, inner sep=\isep]
          node[anchor=90] {\footnotesize$h$} (r1);
    \node (r3) at (   0*\sz, 1.5*\sz) [blackdot]{}
        edge[thick, inner sep=3*\isep]
          node[anchor=-80] {\footnotesize$h$} (r1)
        edge[thick, inner sep=\isep, green!30!black]
          node[anchor=-135] {\footnotesize$c$} (r2);
    \node (r4) at (   0*\sz,-1.5*\sz) [blackdot]{}
        edge[thick, inner sep=\isep]
          node[anchor=80] {\footnotesize$h$} (r1)
        edge[thick, inner sep=\isep, green!30!black]
          node[anchor=135] {\footnotesize$c$} (r2);
  \end{tikzpicture}
  \notag \\
&=
  \newcommand{\sz}{2.5mm}
  \newcommand{\isep}{0.1*\sz}
  -\frac{1}{3} \;
  \begin{tikzpicture}[baseline=-0.1*\sz]
    \node (r1) at (-3.0*\sz,   0*\sz) [whitedot]{};
    \node (r2) at ( 2.0*\sz,   0*\sz) [blackdot]{}
        edge[thick, inner sep=\isep] (r1);
    \node (r3) at (   0*\sz, 1.5*\sz) [blackdot]{}
        edge[thick, inner sep=3*\isep] (r1)
        edge[thick, inner sep=\isep, green!30!black] (r2);
    \node (r4) at (   0*\sz,-1.5*\sz) [blackdot]{}
        edge[thick, inner sep=\isep] (r1)
        edge[thick, inner sep=\isep, green!30!black] (r2);
  \end{tikzpicture}
  -\frac{2}{3} \;
  \begin{tikzpicture}[baseline=-0.1*\sz]
    \node (r1) at (-3.0*\sz,   0*\sz) [whitedot]{};
    \node (r2) at ( 2.0*\sz,   0*\sz) [blackdot]{}
        edge[thick, inner sep=\isep] (r1);
    \node (r3) at (   0*\sz, 1.5*\sz) [blackdot]{}
        edge[thick, inner sep=3*\isep, blue!50!black] (r1)
        edge[thick, inner sep=\isep, green!30!black] (r2);
    \node (r4) at (   0*\sz,-1.5*\sz) [blackdot]{}
        edge[thick, inner sep=\isep] (r1)
        edge[thick, inner sep=\isep, green!30!black] (r2);
    \node (r5) at (-1.3*\sz, 2.8*\sz) [blackdot]{}
        edge[thick, inner sep=3*\isep] (r1)
        edge[thick, inner sep=\isep, dotted, red!70!black] (r3);
  \end{tikzpicture}
  -\frac{1}{3} \;
  \begin{tikzpicture}[baseline=-0.1*\sz]
    \node (r1) at (-3.0*\sz,   0*\sz) [whitedot]{};
    \node (r2) at ( 2.0*\sz,   0*\sz) [blackdot]{}
        edge[thick, inner sep=\isep] (r1);
    \node (r3) at (   0*\sz, 1.5*\sz) [blackdot]{}
        edge[thick, inner sep=3*\isep, blue!50!black] (r1)
        edge[thick, inner sep=\isep, green!30!black] (r2);
    \node (r4) at (   0*\sz,-1.5*\sz) [blackdot]{}
        edge[thick, inner sep=\isep, blue!50!black] (r1)
        edge[thick, inner sep=\isep, green!30!black] (r2);
    \node (r5) at (-1.3*\sz, 2.8*\sz) [blackdot]{}
        edge[thick, inner sep=3*\isep] (r1)
        edge[thick, inner sep=\isep, dotted, red!70!black] (r3);
    \node (r6) at (-1.3*\sz,-2.8*\sz) [blackdot]{}
        edge[thick, inner sep=3*\isep] (r1)
        edge[thick, inner sep=\isep, dotted, red!70!black] (r4);
  \end{tikzpicture}
  %
  %
  %
  -\frac{1}{3} \;
  \begin{tikzpicture}[baseline=-0.1*\sz]
    \node (r1) at (-3.0*\sz,   0*\sz) [whitedot]{};
    \node (r2) at ( 2.0*\sz,   0*\sz) [blackdot]{}
        edge[thick, inner sep=\isep, blue!50!black] (r1);
    \node (r3) at (   0*\sz, 2.5*\sz) [blackdot]{}
        edge[thick, inner sep=3*\isep] (r1)
        edge[thick, inner sep=\isep, green!30!black] (r2);
    \node (r4) at (   0*\sz,-2.5*\sz) [blackdot]{}
        edge[thick, inner sep=\isep] (r1)
        edge[thick, inner sep=\isep, green!30!black] (r2);
    \node (r9) at ( 0.0*\sz, 1.0*\sz) [blackdot]{}
        edge[thick, inner sep=\isep] (r1)
        edge[thick, inner sep=\isep] (r2);
  \end{tikzpicture}
  -\frac{2}{3} \;
  \begin{tikzpicture}[baseline=-0.1*\sz]
    \node (r1) at (-3.0*\sz,   0*\sz) [whitedot]{};
    \node (r2) at ( 2.0*\sz,   0*\sz) [blackdot]{}
        edge[thick, inner sep=\isep, blue!50!black] (r1);
    \node (r3) at (   0*\sz, 2.5*\sz) [blackdot]{}
        edge[thick, inner sep=3*\isep, blue!50!black] (r1)
        edge[thick, inner sep=\isep, green!30!black] (r2);
    \node (r4) at (   0*\sz,-2.5*\sz) [blackdot]{}
        edge[thick, inner sep=\isep] (r1)
        edge[thick, inner sep=\isep, green!30!black] (r2);
    \node (r5) at (-1.5*\sz, 3.3*\sz) [blackdot]{}
        edge[thick, inner sep=3*\isep] (r1)
        edge[thick, inner sep=\isep, dotted, red!70!black] (r3);
    \node (r9) at ( 0.0*\sz, 1.0*\sz) [blackdot]{}
        edge[thick, inner sep=\isep] (r1)
        edge[thick, inner sep=\isep] (r2);
  \end{tikzpicture}
  -\frac{1}{3} \;
  \begin{tikzpicture}[baseline=-0.1*\sz]
    \node (r1) at (-3.0*\sz,   0*\sz) [whitedot]{};
    \node (r2) at ( 2.0*\sz,   0*\sz) [blackdot]{}
        edge[thick, inner sep=\isep, blue!50!black] (r1);
    \node (r3) at (   0*\sz, 2.5*\sz) [blackdot]{}
        edge[thick, inner sep=3*\isep, blue!50!black] (r1)
        edge[thick, inner sep=\isep, green!30!black] (r2);
    \node (r4) at (   0*\sz,-2.5*\sz) [blackdot]{}
        edge[thick, inner sep=\isep, blue!50!black] (r1)
        edge[thick, inner sep=\isep, green!30!black] (r2);
    \node (r5) at (-1.5*\sz, 3.3*\sz) [blackdot]{}
        edge[thick, inner sep=3*\isep] (r1)
        edge[thick, inner sep=\isep, dotted, red!70!black] (r3);
    \node (r6) at (-1.5*\sz,-3.3*\sz) [blackdot]{}
        edge[thick, inner sep=3*\isep] (r1)
        edge[thick, inner sep=\isep, dotted, red!70!black] (r4);
    \node (r9) at ( 0.0*\sz, 1.0*\sz) [blackdot]{}
        edge[thick, inner sep=\isep] (r1)
        edge[thick, inner sep=\isep] (r2);
  \end{tikzpicture}
  \;.
  \label{eq:hBintg_example_h}
\end{align}


But if we evaluate the same integral by the $\xi$ defined by Eq. \eqref{eq:fxi}, we get
%
\begin{align}
  I^{(f)}
&=
  \newcommand{\sz}{2.5mm}
  \newcommand{\isep}{0.1*\sz}
  -\frac{1}{3} \;
  \begin{tikzpicture}[baseline=-0.1*\sz]
    \node (r1) at (-3.0*\sz,   0*\sz) [whitedot]{};
    \node (r2) at ( 2.0*\sz,   0*\sz) [blackdot]{}
        edge[thick, inner sep=\isep] (r1);
    \node (r3) at (   0*\sz, 1.5*\sz) [blackdot]{}
        edge[thick, inner sep=3*\isep] (r1)
        edge[thick, inner sep=\isep, green!30!black] (r2);
    \node (r4) at (   0*\sz,-1.5*\sz) [blackdot]{}
        edge[thick, inner sep=\isep] (r1)
        edge[thick, inner sep=\isep, green!30!black] (r2);
  \end{tikzpicture}
  -\frac{3}{4} \;
  \begin{tikzpicture}[baseline=-0.1*\sz]
    \node (r1) at (-3.0*\sz,   0*\sz) [whitedot]{};
    \node (r2) at ( 2.0*\sz,   0*\sz) [blackdot]{}
        edge[thick, inner sep=\isep] (r1);
    \node (r3) at (   0*\sz, 1.5*\sz) [blackdot]{}
        edge[thick, inner sep=3*\isep, blue!50!black] (r1)
        edge[thick, inner sep=\isep, green!30!black] (r2);
    \node (r4) at (   0*\sz,-1.5*\sz) [blackdot]{}
        edge[thick, inner sep=\isep] (r1)
        edge[thick, inner sep=\isep, green!30!black] (r2);
    \node (r5) at (-1.3*\sz, 2.8*\sz) [blackdot]{}
        edge[thick, inner sep=3*\isep] (r1)
        edge[thick, inner sep=\isep, dotted, red!70!black] (r3);
  \end{tikzpicture}
  -\frac{2}{5} \;
  \begin{tikzpicture}[baseline=-0.1*\sz]
    \node (r1) at (-3.0*\sz,   0*\sz) [whitedot]{};
    \node (r2) at ( 2.0*\sz,   0*\sz) [blackdot]{}
        edge[thick, inner sep=\isep] (r1);
    \node (r3) at (   0*\sz, 1.5*\sz) [blackdot]{}
        edge[thick, inner sep=3*\isep, blue!50!black] (r1)
        edge[thick, inner sep=\isep, green!30!black] (r2);
    \node (r4) at (   0*\sz,-1.5*\sz) [blackdot]{}
        edge[thick, inner sep=\isep, blue!50!black] (r1)
        edge[thick, inner sep=\isep, green!30!black] (r2);
    \node (r5) at (-1.3*\sz, 2.8*\sz) [blackdot]{}
        edge[thick, inner sep=3*\isep] (r1)
        edge[thick, inner sep=\isep, dotted, red!70!black] (r3);
    \node (r6) at (-1.3*\sz,-2.8*\sz) [blackdot]{}
        edge[thick, inner sep=3*\isep] (r1)
        edge[thick, inner sep=\isep, dotted, red!70!black] (r4);
  \end{tikzpicture}
  %
  %
  %
  -\frac{1}{4} \;
  \begin{tikzpicture}[baseline=-0.1*\sz]
    \node (r1) at (-3.0*\sz,   0*\sz) [whitedot]{};
    \node (r2) at ( 2.0*\sz,   0*\sz) [blackdot]{}
        edge[thick, inner sep=\isep, blue!50!black] (r1);
    \node (r3) at (   0*\sz, 2.5*\sz) [blackdot]{}
        edge[thick, inner sep=3*\isep] (r1)
        edge[thick, inner sep=\isep, green!30!black] (r2);
    \node (r4) at (   0*\sz,-2.5*\sz) [blackdot]{}
        edge[thick, inner sep=\isep] (r1)
        edge[thick, inner sep=\isep, green!30!black] (r2);
    \node (r9) at ( 0.0*\sz, 1.0*\sz) [blackdot]{}
        edge[thick, inner sep=\isep] (r1)
        edge[thick, inner sep=\isep] (r2);
  \end{tikzpicture}
  -\frac{3}{5} \;
  \begin{tikzpicture}[baseline=-0.1*\sz]
    \node (r1) at (-3.0*\sz,   0*\sz) [whitedot]{};
    \node (r2) at ( 2.0*\sz,   0*\sz) [blackdot]{}
        edge[thick, inner sep=\isep, blue!50!black] (r1);
    \node (r3) at (   0*\sz, 2.5*\sz) [blackdot]{}
        edge[thick, inner sep=3*\isep, blue!50!black] (r1)
        edge[thick, inner sep=\isep, green!30!black] (r2);
    \node (r4) at (   0*\sz,-2.5*\sz) [blackdot]{}
        edge[thick, inner sep=\isep] (r1)
        edge[thick, inner sep=\isep, green!30!black] (r2);
    \node (r5) at (-1.5*\sz, 3.3*\sz) [blackdot]{}
        edge[thick, inner sep=3*\isep] (r1)
        edge[thick, inner sep=\isep, dotted, red!70!black] (r3);
    \node (r9) at ( 0.0*\sz, 1.0*\sz) [blackdot]{}
        edge[thick, inner sep=\isep] (r1)
        edge[thick, inner sep=\isep] (r2);
  \end{tikzpicture}
  -\frac{2}{6} \;
  \begin{tikzpicture}[baseline=-0.1*\sz]
    \node (r1) at (-3.0*\sz,   0*\sz) [whitedot]{};
    \node (r2) at ( 2.0*\sz,   0*\sz) [blackdot]{}
        edge[thick, inner sep=\isep, blue!50!black] (r1);
    \node (r3) at (   0*\sz, 2.5*\sz) [blackdot]{}
        edge[thick, inner sep=3*\isep, blue!50!black] (r1)
        edge[thick, inner sep=\isep, green!30!black] (r2);
    \node (r4) at (   0*\sz,-2.5*\sz) [blackdot]{}
        edge[thick, inner sep=\isep, blue!50!black] (r1)
        edge[thick, inner sep=\isep, green!30!black] (r2);
    \node (r5) at (-1.5*\sz, 3.3*\sz) [blackdot]{}
        edge[thick, inner sep=3*\isep] (r1)
        edge[thick, inner sep=\isep, dotted, red!70!black] (r3);
    \node (r6) at (-1.5*\sz,-3.3*\sz) [blackdot]{}
        edge[thick, inner sep=3*\isep] (r1)
        edge[thick, inner sep=\isep, dotted, red!70!black] (r4);
    \node (r9) at ( 0.0*\sz, 1.0*\sz) [blackdot]{}
        edge[thick, inner sep=\isep] (r1)
        edge[thick, inner sep=\isep] (r2);
  \end{tikzpicture}
  \;.
  \label{eq:hBintg_example_f}
\end{align}
Note that in Eq. \eqref{eq:hBintg_example_f},
the denominator of the coefficient before each graph
is equal to the number $f$-bonds adjacent to field point 1
[cf. the $1/(n+1)$ factor in Eq. \eqref{eq:mu_fyseries}].


This example clearly shows that \emph{generally}
the value of the charging integral Eq. \eqref{eq:hBint}
depends critically on the choice of $\xi$.
%
As a result, the value from Eq. \eqref{eq:muhncB_hBseries}
under the assumption of Eq. \eqref{eq:hxi},
would be different from that from Eq. \eqref{eq:mu_fyseries}
under the assumption of Eq. \eqref{eq:fxi}.
%
This point is illustrated below.





\section{PY equation}



Now let us show that Eqs. \eqref{eq:mu_fyseries} and \eqref{eq:muhncB_hBseries}
produce contradictory results, when they are used on the PY equation.

In the PY equation,
every graph in the expansion of $y(r) = 1 + t(r)$
is a polygon graph.
That is, it is possible to
put all vertices of the graph on a convex polygon
without producing any intersecting edge.
%
Thus, Eq. \eqref{eq:mu_fyseries} permits
no graph like
\[
G_3
=
  \newcommand{\sz}{3.6mm}
  \begin{tikzpicture}[baseline=-0.5*\sz]
    \node (r1) at (-1.5*\sz, -\sz) [whitedot]{};
    \node (r2) at ( 1.5*\sz, -\sz) [whitedot]{};
    \node (r3) at ( 1.5*\sz,  \sz) [blackdot]{};
    \node (r4) at (   0*\sz,  \sz) [blackdot]{};
    \node (r5) at (-1.5*\sz,  \sz) [blackdot]{};
    \draw (r1) -- (r3) -- (r2) -- (r4) -- (r1) -- (r5) -- (r2) -- (r1);
  \end{tikzpicture} \; .
\]

Now we shall show that $G_3$ exists in the expansion of Eq. \eqref{eq:muhncB_hBseries},
which will yield an immediate contradiction.
%
First, the bridge function in the PY equation is given by
\[
B_\xi(r) = \log[1 + t_\xi(r)] - t_\xi(r) = -t_\xi^2(r)/2 + t_\xi^3(r)/3 - \cdots.
\]
Each $t_\xi(r)$ can be further expanded by Eq. \eqref{eq:ozxi},
which gives a linear relation between $t_\xi$ and $h_\xi$.
%
As a result,
the term $-t_\xi^2(r)/2$ belongs to $B^{(2)} \, \xi^2$,
and
the term $t_\xi^3(r)/3$ belongs to $B^{(3)} \, \xi^3$.



Now $G_3$ can occur in Eq. \eqref{eq:muhncB_hBseries}
only in the $n = 2$ and $n = 3$ terms of the sum.
%
For $B^{(2)}$ and $B^{(3)}$,
the corresponding graphs in $h(r)$ to form $G_3$ must be
$  \newcommand{\sz}{2mm}
  \begin{tikzpicture}[baseline=-0.5*\sz]
    \node (r1) at (-1*\sz, -\sz) [whitedot]{};
    \node (r2) at ( 1*\sz, -\sz) [whitedot]{};
    \node (r3) at ( 0*\sz,  \sz) [blackdot]{};
    \draw (r1) -- (r3) -- (r2) -- (r1);
  \end{tikzpicture}$,
and
$  \newcommand{\sz}{2mm}
  \begin{tikzpicture}[baseline=-1.5*\sz]
    \node (r1) at (-1*\sz, -\sz) [whitedot]{};
    \node (r2) at ( 1*\sz, -\sz) [whitedot]{};
    \draw (r1) -- (r2);
  \end{tikzpicture}$,
respectively.
%
Thus, the overall coefficient before $G_3$
in the expansion of Eq. \eqref{eq:muhncB_hBseries}
is given by
\[
  -\frac{2}{3} \cdot \left( -\frac{1}{2} \right)
  -\frac{3}{4} \cdot \left( \frac{1}{3}  \right)
= \frac{1}{12} \ne 0.
\]

We have thus shown that Eqs. \eqref{eq:mu_fyseries}
and \eqref{eq:muhncB_hBseries}
yield different values of chemical potential
for the PY equation.




\section{Self-consistent integral equations}



We can certainly fix the above $\mu$-inconsistency
by adding some variable term,
e.g., $\lambda \, t^2(r)$, to $y(r)$,
and then setting $\lambda$ to enforce the $\mu$-consistency
of Eqs. \eqref{eq:mu_fyseries} and \eqref{eq:muhncB_hBseries}.
%
But by doing so,
can we obtain a more accurate integral equation?
%
Some fleeting attempts suggest a negative answer,
at least for the three-dimensional hard-sphere fluid.
%
The reason is the following.
%
The HNC equation naturally achieves the $\mu$-consistency
of Eqs. \eqref{eq:mu_fyseries} and \eqref{eq:muhncB_hBseries}.
%
But unfortunately, the virial coefficient from the HNC equation
(we should use the virial-route value)
$B_4^{v, \mathrm{HNC}} = 0.4453 B_2^3$
does not agree well with the true value $B_4 = 0.2869 B_2^3$.



An alternative is to enforce the self-consistency of
the virial coefficients from Eq. \eqref{eq:muintg}
and those from the compressibility or virial route.
%
For $B_4$ of the three-dimensional hard-sphere fluid,
we found that, for the PY equation,
$B_4^{\mu, \mathrm{PY}} = B_4^{c, \mathrm{HNC}} = 0.2092 B_2^3$,
$B_4^{v-\mu, \mathrm{PY}} = B_4^{v, \mathrm{HNC}} = 0.4453 B_2^3$,
and
$B_4^{c-\mu, \mathrm{PY}} = 0.2731 B_2^3$,
respectively.
%
These figures are inferior to the pressure-consistent (compressibility and virial) result
$B_4^{c-v, \mathrm{PY}} = 0.2824 B_2^3$.
%
Thus, more work are to be done in this direction.


\appendix



\section{\label{sec:muhncB} Derivation of Eq. \eqref{eq:muhncB}}


Here, we derive Eq. \eqref{eq:muhncB} from Eq. \eqref{eq:mudf}.
%
The manipulation is rather standard\cite{singer1985}.
%
We start from Eq. \eqref{eq:dmu},
which is the differential form of Eq. \eqref{eq:mudf}.
%
Integration by parts yields
\[
(\partial_\xi f_\xi) \, y_\xi
=
(\partial_\xi e_\xi) \, y_\xi
= \partial_\xi ( e_\xi \, y_\xi )
- e_\xi \, \partial_\xi y_\xi,
\]
where $e_\xi = f_\xi + 1$.
%
Since $y_\xi = \exp( t_\xi + B_\xi )$,
\[
  \partial_\xi y_\xi = y_\xi \, \partial_\xi( t_\xi + B_\xi).
\]
and by using $e_\xi \, y_\xi = g_\xi$, we get
\begin{align*}
(\partial_\xi f_\xi) \, y_\xi
&= \partial_\xi g_\xi
- g_\xi \, \partial_\xi (t_\xi + B_\xi)
\\
&= \partial_\xi h_\xi
- \partial_\xi (t_\xi + B_\xi)
- h_\xi \, \partial_\xi (t_\xi + B_\xi),
\end{align*}


Finally, by integrating over $\vr$, we get
\[
\int d\vr \, (\partial_\xi f_\xi) \, y_\xi
=
\int d\vr \, \left\{
  \partial_\xi (c_\xi - B_\xi - \tfrac{1}{2} h_\xi \, t_\xi)
- h_\xi \, \partial_\xi B_\xi \right\},
\]
where we have used the fact that
$t_\xi$ and $h_\xi$ are linearly related by
Eq. \eqref{eq:ozxi}, such that
\[
\int d\vr \, h_\xi \, \partial_\xi t_\xi
= \int d\vr \, \frac{1}{2} h_\xi \, t_\xi,
\]

Finally, integrating over $\xi$
and multiplying the integral by $\rho$
yields Eq. \eqref{eq:muhncB}.
%
By setting $B = 0$, we get Eq. \eqref{eq:muhnc}.



\bibliography{../liquid}
%\bibliographystyle{alpha}
\end{document}


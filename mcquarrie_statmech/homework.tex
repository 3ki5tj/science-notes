\documentclass[reprint]{revtex4-1}

\usepackage{amsmath}
\usepackage[usenames,dvipsnames,svgnames,table]{xcolor}
\usepackage{hyperref}

\hypersetup{
    colorlinks,
    linkcolor={red!30!black},
    citecolor={blue!50!black},
    urlcolor={blue!80!black}
}

\numberwithin{equation}{section}

\begin{document}
\title{Homework problems for McQuarrie's Statistical Mechanics, \\
  based on Dr. B. M. Pettitt's lectures}
%\author{\vspace{-10ex}}
%\date{\vspace{-10ex}}
\maketitle

\tableofcontents

\section{Notes}

In this section, we collect a few results that are helpful in problem solving.

\subsection{Euler's theorem}

For a single-component system, we have
the following useful equation
\begin{equation}
  E = T \, S - p \, V + \mu \, N.
\label{eq:Euler}
\end{equation}
%
We can show this from McQuarrie's Eq. (1-60),
$$
G = \sum_i \mu_i \, N_i
$$
But by definition $G = E - TS + pV$, so
%
\begin{equation}
  E = T \, S - p \, V + \sum_i \mu_i \, N_i.
\label{eq:Euler_multi}
\end{equation}
%
We are using the one component case.


\section{Lecture 1: Mechanics vs. Statistical Mechanics}

\subsection{Problem 1-5}

\subsection{Problem 1-14}

\subsection{Problem 1-28}

\subsection{Problem 1-30}


\section{Lecture 2: Probability and Distributions}

\subsection{Problem 1-41}

\subsection{Problem 1-44}

\subsection{Problem 1-46}

\subsection{Programming problem}

\paragraph*{Problem}

Write a program and draw $10$ random numbers between $-1$ and $+1$,
add them together and divide by $10$ and save it.
Do this $100$ times, $1000$ times and $100,000$ times.
Plot the distributions as histograms using only $10$ points on the $x$-axis.


\section{Lecture 3. Thermodynamics and Entropy}

\subsection{Problem 1-52}

\subsection{Problem 1-59}


\section{Lecture 4: Second and Third Laws}

\subsection{Problem 1-51}

\subsection{Problem 2-6}

\subsection{Problem 2-8}

\section{Lecture 5: Idea of Ensembles}

\subsection{Programming problem}

Write a program to estimate $\pi$ from Monte Carlo integration.
%
Pick points randomly on a unit square.
%
Some points will lie inside a circle inscribed in the square,
some outside.
%
Use the ratio of the hits inside to the total number of trials
to estimate the area of the circle and therefore $\pi$.
How does it converge with the number of attempts.

\section{Lecture 6: The Principles of A Priori Probabilities}

\subsection{Problem 2-11}

\subsection{Problem 2-13}

\section{Lecture 7: Partition Functions and the Use of Ensembles}

\subsection{Problem 2-16}

\subsection{Problem 2-17}

\section{Lecture 8: The Canonical Ensemble}

\subsection{Problem 3-4}

\subsection{Problem 3-6}

\section{Lecture 9: Grand Canonical Ensemble}

\subsection{Problem 3-15}

\subsection{Problem 3-17}

\subsection{Problem 3-24}

\section{Lecture 10: Fluctuations}

\subsection{Problem 3-22}

\paragraph*{Problem}

Show that in the grand-canonical ensemble,
$$
\sigma_E^2 = k \, T^2 \, C_V
  + \left( \frac{\partial \overline E } { \partial \overline N} \right)_{T, V} \sigma_N^2.
$$

\paragraph*{Corrections.}

There is clearly a typo here, since the units do not match.
%
The most likely correction is
\begin{equation}
  \sigma_E^2 = k \, T^2 \, C_V
  + \left( \frac{\partial \overline E } { \partial \overline N} \right)_{T, V} \sigma_N^2 \, \textcolor{red}{\times \mu}.
  \label{eq:prob3-22a}
\end{equation}
%
Another possible correction is
%
\begin{equation}
  \sigma_E^2 = k \, T^2 \, C_{V,\textcolor{blue}{N}}
  + \left( \frac{\partial \overline E } { \partial \overline N} \right)_{T, V}^{\textcolor{blue}{2}} \sigma_N^2.
  \label{eq:prob3-22b}
\end{equation}
%
Note that in the second correction,
$C_V$ is interpreted as the heat capacity in the canonical ensemble,
so it is less likely.

\paragraph*{Solution.}

We will show Eq. \eqref{eq:prob3-22a} first.
The grand partition function is defined as
\begin{equation}
\Xi
=
  \sum_{N} Q(N, V, T) \, z^N
=
  \sum_{N, E} \Omega(N, V, E) \, e^{-\beta \, E + \mu^* \, N}
,
  \label{eq:Xi_def}
\end{equation}
%
where we have defined the reduced chemical potential $\mu^* = \beta \, \mu = \ln z$.
%
As $\ln \Omega$ corresponds to the entropy $S/k$,
$\ln Q$ corresponds to the Helmholtz free energy $S/k - \beta \, E = -\beta \, A$,
the free energy corresponding to $\ln \Xi$
is the so-called grand potential:
%
\begin{equation}
  \ln \Xi
  = \frac{S}{k} - \beta \, E + \mu^* N
  = \beta \, p \, V
  .
  \label{eq:lnXi}
\end{equation}
Here, the last step follows from Euler's theorem, Eq. \eqref{eq:Euler}.
%
By differentiating Eq. \eqref{eq:Xi_def}, we get
%
\begin{align}
  \overline E &= -\frac{ \partial \ln \Xi } { \partial \beta }
  ,
  \label{eq:dXidbeta}
  \\
  \overline N &= \frac{ \partial \ln \Xi } { \partial \mu^* }
  ,
  \label{eq:dXidmu}
\end{align}
%
or
\begin{equation}
  d \ln \Xi
  =
  \frac{ \partial \ln \Xi } { \partial \beta } d\beta
  +
  \frac{ \partial \ln \Xi } { \partial \mu^* } d\mu^*
  = -\overline E \, d\beta + \overline N \, d\mu^*
  .
  \notag
  %\label{eq:firstlawXi}
\end{equation}
%
We can compare this to the first law of thermodynamics,
$dE = T \, dS - p \, dV + \mu \, dN$,
which after some Legrendre transforms, can be written as
%
\begin{equation}
  d(\beta \, p \, V) = -E \, d\beta + N \, d\mu^* + \beta \, p \, dV
  .
  \notag
  %\label{eq:firstlaw}
\end{equation}
This means that we should interpret $E$ and $N$ in the first law
as the ensemble averages, $\overline E$ and $\overline N$, respectively.
%
We can further differentiate Eqs. \eqref{eq:dXidbeta} and \eqref{eq:dXidmu}
%
\begin{align}
  \sigma_E^2
  &=
  -\left( \frac{ \partial \overline E } { \partial \beta } \right)_{\mu^*, V}
  = \frac{ \partial^2 \ln \Xi } { \partial \beta^2 }
  ,
  \label{eq:varE}
  \\
  \sigma_N^2
  &= \left( \frac{ \partial \overline N } { \partial \mu^* } \right)_{\beta, V}
  = \frac{ \partial^2 \ln \Xi } { \partial \mu^{*2} }
  \label{eq:varN}
  ,
  \\
  \left( \frac{ \partial \overline E } { \partial \mu^* } \right)_{\beta, V}
  &= -\frac{ \partial^2 \ln \Xi } { \partial \mu^* \, \partial \beta }
  = - \left( \frac{ \partial \overline N } { \partial \beta } \right)_{\mu^*, V}
  \label{eq:covEN}
  .
\end{align}

Now the heat capacity in the grand-canonical ensemble can be defined as
%
\begin{equation}
= \left( \frac{ T \, \partial S } { \partial T } \right)_{\mu^*, V}
= -\beta \left( \frac{ \partial S } { \partial \beta } \right)_{\mu^*, V}
= -\frac{1}{T} \left( \frac{ \partial (S/k) } { \partial \beta } \right)_{\mu^*, V}.
\label{eq:CV_def}
\end{equation}
%
Using Eq. \eqref{eq:lnXi}, we have
\begin{align}
-T \, C_V
  &= \left( \frac{ \partial (\ln \Xi + \beta \, \overline E - \mu^* \, \overline N) }
  { \partial \beta } \right)_{\mu^*, V}.
  \notag
  \\
  &= \frac{ \partial \ln \Xi } { \partial \beta }
  + \overline E
  + \beta \left( \frac{ \partial \overline E } { \partial \beta } \right)_{\mu^*, V}
  - \mu^* \left( \frac{ \partial \overline N } { \partial \beta } \right)_{\mu^*, V}
  \notag
  \\
  &=
  - \beta \, \sigma_E^2
  + \mu^* \left( \frac{ \partial \overline E } { \partial \mu^* } \right)_{\beta, V}
  ,
  \notag
\end{align}
%
where we have used Eqs. \eqref{eq:dXidbeta}, \eqref{eq:varE}, and \eqref{eq:covEN}
in the last step.
%
Further using Eq. \eqref{eq:varN}, we have
\begin{equation}
T \, C_V
=
\beta \, \sigma_E^2
+
\mu^* \,
\frac{ \left( \partial \overline E / \partial \mu^* \right)_{\beta, V} }
     { \left( \partial \overline N / \partial \mu^* \right)_{\beta, V} }
\, \sigma_N^2
.
\label{eq:TCV}
\end{equation}
Finally, if we interpret
$$
\frac{ \left( \partial \overline E / \partial \mu^* \right)_{\beta, V} }
     { \left( \partial \overline N / \partial \mu^* \right)_{\beta, V} }
\to
\left(
  \frac{ \partial \overline E }
       { \partial \overline N }
\right)_{T, V}
,
$$
and multiply both sides of Eq. \eqref{eq:TCV} by $k T = 1/\beta$,
we recover Eq. \eqref{eq:prob3-22a}.

\paragraph*{Second interpretation.}


About the second interpretation, Eq. \eqref{eq:prob3-22b},
we will use the advanced technique of transforming thermodynamic variables
(i.e. chain rule),
so it is okay to ignore it if it sounds unfamiliar.
%
Similar to Eq. \eqref{eq:CV_def},
the heat capacity in the canonical ensemble is defined as
\begin{equation}
C_{V, N}
= -\frac{1}{T} \left( \frac{ \partial S^* } { \partial \beta } \right)_{V, N},
  \label{eq:CVN_def}
\end{equation}
where we have introduced $S^* = S/k$ for convenience.
%
To be clear, the heat capacity in the grand-canonical ensemble
will be denoted as
\begin{equation}
C_{V, \mu^*} = -\frac{1}{T} \left( \frac{ \partial S^* } { \partial \beta } \right)_{V, \mu^*}.
  \label{eq:CVmu_def}
\end{equation}
%
Using the technique of determinants for transforming variables, we have
\begin{align*}
  \left( \frac{ \partial S^* } {\partial \beta } \right)_{\mu^*}
  &=
  \frac{ \partial (S^*, \mu^*) } { \partial (\beta, \mu^*) }
  =
  \left.
  \frac{ \partial (S^*, \mu^*) } { \partial (\beta, N) }
  \middle/
  \frac{ \partial (\beta, \mu^*) } { \partial (\beta, N) }
  \right.
  \\
  &=
  \left( \frac{ \partial S^* } {\partial \beta } \right)_{N}
  -
  \left.
  \left( \frac{ \partial \mu^* } {\partial \beta } \right)_{N}
  \left( \frac{ \partial S^* } {\partial N } \right)_{\beta}
  \middle/
  \left( \frac{ \partial \mu^* } {\partial N } \right)_{\beta}
  \right.
\end{align*}
From the first law, we have
$$
d(S^* - \beta \, E) = -E \, d\beta -\mu^* dN + \beta p \, dV
,
$$
%
so
\begin{align*}
  \left( \frac{ \partial \mu^* } {\partial \beta } \right)_{N}
  &=
  \left( \frac{ \partial E } {\partial N } \right)_{\beta}
  ,
  \\
  \left( \frac{ \partial S^* } {\partial N } \right)_{\beta}
  &=
  \left( \frac{ \partial (S^* - \beta \, E) } {\partial N } \right)_{\beta}
  +
  \left( \frac{ \partial ( \beta \,  E) } {\partial N } \right)_{\beta}
  \\
  &=
  -\mu^*
  +
  \beta \, \left( \frac{ \partial E } {\partial N } \right)_{\beta}
  .
\end{align*}
Using the above two expressions, we get
\begin{align*}
  \left( \frac{ \partial S^* } {\partial \beta } \right)_{\mu^*}
  &=
  \left( \frac{ \partial S^* } {\partial \beta } \right)_{N}
  +
  \mu^* \,
  \left( \frac{ \partial E } {\partial N } \right)_{\beta}
  \left( \frac{ \partial N } {\partial \mu^* } \right)_{\beta}
  \\
  &-
  \beta \,
  \left( \frac{ \partial E } {\partial N } \right)_{\beta}^2
  \left( \frac{ \partial N } {\partial \mu^* } \right)_{\beta}
  .
\end{align*}
%
Dividing both sides by $-1/(k T) = -\beta$,
and using Eqs. \eqref{eq:CVN_def}, \eqref{eq:CVmu_def}, and \eqref{eq:varN},
we get
$$
  k T^2 C_{V, \mu^*}
  +
  \mu \,
  \left( \frac{ \partial E } {\partial N } \right)_{\beta}
  \sigma_N^2
  =
  k T^2 C_{V, N}
  +
  \left( \frac{ \partial E } {\partial N } \right)_{\beta}^2
  \sigma_N^2
  .
$$
If we interpret
$\left( \partial E / \partial N \right)_{\beta}$
as
$\left( \partial \overline E / \partial \overline N \right)_{T, V}$,
we recover Eq. \eqref{eq:prob3-22b}.
Note that we have implicitly assumed that $V$ is fixed in the above derivation.


\subsection{Problem 3-23}

\section{Lecture 11: Quantum vs. Classical Statistics}

\section{Lecture 12: Ideal Monatomic Gas}

\subsection{Derive the thermodynamic potential for $(\mu, p, T)$}

\paragraph*{Problem.}

Derive the thermodynamic potential for the $(\mu, p, T)$ ensemble.


\paragraph*{Solution 1.}

The answer is that no such thermodynamic potential exists.

If there were a thermodynamic potential for the $(\mu, p, T)$ ensemble,
the independent variables would be $\beta$, $\beta \, p$ and $\beta \, \mu$.
%
The partition function would look be
\begin{align*}
  \Psi
  &=
  \sum_V \Xi(N, V, \beta) \, e^{-\beta \, p \, V}
  \\
  &=
  \sum_V \sum_N Q(N, V, \beta) \, e^{\beta \, \mu \, N} \, e^{-\beta \, p \, V}
  \\
  &=
  \sum_V \sum_N \sum_E \Omega(N, V, E) \, e^{-\beta \, E}
  \, e^{\beta \, \mu \, N} \, e^{-\beta \, p \, V}
  .
\end{align*}
%
Now let us map the partition functions to thermodynamic potentials.
%
Recall that the logarithm, $\ln \Omega$,
corresponds to the entropy, $S/k$.

The canonical partition function is
$Q = \sum_E \Omega \, e^{-\beta \, E}$,
and its logarithm corresponds to
$$
\ln Q = /k - \beta \, E = -\beta \, A,
$$
where $A = E - T \, S$ is the Helmholtz free energy.

Similarly, the grand canonical partition function is
$\Xi = \sum_N Q \, e^{\beta \, \mu \, N}$,
and its logarithm corresponds to
$$
\ln \Xi = \ln Q + \beta \, \mu \, N = -\beta \, (A - \mu \, N).
$$
But from Euler's theorem, Eq. \eqref{eq:Euler}, we know that
$$
A - \mu N = E - T S - \mu N = -p \, V.
$$
Thus,
\begin{equation}
  \ln \Xi = \beta \, p \, V.
\label{eq:lnXi_bpV}
\end{equation}

Finally, for the thermodynamic potential, $\Psi$,
we would have
$$
\ln \Psi = \ln \Xi - \beta \, p \, V.
$$
But from Eq. \eqref{eq:lnXi_bpV},
we simply get zero,
which means that such a thermodynamic potential does not exist.

\paragraph*{Solution 2.}

There is actually a simpler explanation.
%
A thermodynamic potential must be an extensive quantity,
which means that it depends linearly with the system size.
%
But $\mu$, $p$, and $T$ are all intensive quantities.
%
So there is no way to construct an extensive quantity
from the three variables.


\subsection{Problem 5-9}


\subsection{Problem 5-14}


\section{Lecture 13: Ideal Diatomic Gas}

\subsection{Problem 6-19}

\subsection{Show that when $kT \gg h\nu$, $q_\mathrm{vib} = kT/(h\nu)$.}

Show that when $kT \gg h\nu$, $q_\mathrm{vib} = kT/(h\nu)$.


\section{Lecture 14: Classical Statistical Mechanics}

\subsection{Problem 7-15}

\subsection{Problem 7-16}

\paragraph{Hint: use 7-15.}

\subsection{Problem 7-18}

\section{Lecture 15: Classical Mechanics and Phase Space}

\subsection{Problem 7-6}

\paragraph{Hint: see 2-16, i.e. $q \sim \sinh(E)$.}

\subsection{Problem 7-17}

\section{Lecture 16: Classical Phase Space}

\section{Lecture 17: Crystals, Glasses, and Polymers}

\section{Lecture 18: Imperfect Gases}

\section{Lecture 19: Classical Monatomic liquids}

\section{Lecture 20: Density Distributions and Correlation Functions}

\section{Lecture 21: Evaluating Correlation Functions}

\section{Lecture 22: Equilibria and Multi-Component Systems}

\section{Lecture 23: Thermodynamic Perturbation Theory}

\section{Lecture 24: Electrolyte Solutions}

\end{document}


\documentclass[12pt]{article}
\usepackage{amsmath}
\usepackage{tikz}
\usepackage{url}
\begin{document}


\newcounter{problem}[section]
\newenvironment{problem}[1]
{
  \refstepcounter{problem}\par\bigskip
  \textbf{\large Problem~\theproblem. #1}
  \par\medskip
}
{ \medskip }




\section{Problem Set 3: Simulated tempering}


\begin{problem}{An Analogy}


Which of the following resembles simulated tempering
and the framework of expanded ensemble most?
%
Can you identify temperature and configuration there?

\begin{enumerate}
  \item
    A person browsing an electronic dictionary,
    which lists several meanings of a word,
    with synonyms to some particular meanings given as hyper-links.

  \item
    An online price comparison tool that lists and compares
    the prices from different websites and sorts them
    in ascending order.

  \item
    A runner on the treadmill doing interval training,
    in which the speed varies periodically from slow to fast
    in each interval.

  \item
    A science-fiction movie, in which a person can travel
    to a parallel universe, only if his/her counterpart in
    that universe is transported to this one.
\end{enumerate}


\end{problem}


\pagebreak

\section{Answer keys}


\begin{problem}{An Analogy}

\begin{enumerate}
  \item
    This is my favorite answer.
    Each word ``samples'' a few meanings.
    So the word corresponds to a distinct distribution,
    labeled by the ``temperature,''
    and the meaning space is the configuration space.
    %
    When two words share a similar meaning,
    one can perform a ``temperature'' transition.

  \item
    It resembles the temperature transition mechanics,
    if the heat-bath rule is used.
    %
    But it may not be a good analogy for the expanded ensemble framework.

  \item
    In simulated tempering,
    the temperature changes stochastically not periodically.

  \item
    This is more like a swap in replica exchange or parallel tempering.
\end{enumerate}

\end{problem}


\end{document}

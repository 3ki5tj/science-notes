\documentclass[12pt]{article}
\usepackage{amsmath}
\usepackage{tikz}
\usepackage{url}
\begin{document}


\section{Problem Set 1: Transition matrices}

\subsection{Getting familiar with transition matrices}

%This exercise helps develop some familiarity with transition matrices.

Consider the following a matrix,
\begin{equation}
  \mathbf T
  =
  \left(
    \begin{array}{ccc}
    \times  &   0.4   &   0       \\
    1       &   0     &   \times  \\
    0       & \times  &   0
    \end{array}
  \right)
  .
\end{equation}
It has a few missing elements marked by ``$\times$.''

\begin{enumerate}
  \item
  Fill the missing elements using the properities of a transition matrix:
  $$
    T_{ij} \ge 0 \qquad \mathrm{and} \qquad \sum_{i = 1}^N T_{ij} = 1.
  $$

  \item
  Draw the transition diagram of nodes and arrows.

  \item
  Find the eigenvalues and right eigenvectors.
  \small{
    For the eigenvalues,
    it can be done with a pencil and a piece of paper
    by computing the roots of the cubic equation
    $$
    \det (\mathbf T - \lambda \, \mathbf I) = 0.
    $$
    But if this is too much,
    we have Wolfram Alpha, \url{www.wolframalpha.com}.
    Go there and type something like
    \texttt{
      Eigensystem[\{\{0.4,0.5\},\{0.6,0.5\}\}]}.
  }

  \item
  Identify the eigenvector that corresponds to the eigenvalue $\lambda = 1$.
  Verify that its elements share the same sign.
  Try to normalize it to the stationary distribution, $\mathbf p^s$,
  such that all elements sum to $1.0$.

  \item
  \label{prob1:eigneg1}
  Verify that all eigenvalues satisfy $|\lambda| \le 1$.
  If there is an eigenvalue with $\lambda = -1$,
  can we say something about it?

  \item
  Verify that the transition matrix satisfies detailed balance:
  $$
    T_{ij} \, p^s_j = T_{ji} \, p^s_i,
  $$
  for any $i$ and $j$.

  \item
  For each right eigenvector, $\mathbf v$,
  compute the corresponding vector, $\mathbf u$,
  as
  $$
    u_i = \frac { v_i } { p^s_i },
  $$
  verify that every $\mathbf u$
  constructed this way is a left eigenvalue of $\mathbf T$
  with the same eigenvalue.

  \item
  Verify that $\mathbf u_i \cdot \mathbf v_j = 0$
  for any $i \ne j$.

  \item
  \label{prob1:coarsegraining}
  Construct a coarse-grained transition matrix, $\mathbf S$,
  of two states by merging the first and the third states
  in $\mathbf T$. If we called the merged state $M$, then
  %
  \begin{equation}
  S_{M2} = T_{12} + T_{32},
  \qquad \mathrm{and} \qquad
  S_{2M} =
  \frac{ T_{21} \, p^s_1 + T_{23} \, p^s_3 }
       { p^s_1 + p^s_3 }
  .
  \label{eq:coarse-graining}
  \end{equation}
  %
  Figure out $S_{22}$ and $S_{MM}$.

  \item
  Compute the eigenvalues of $\mathbf S$
  and compare them with those of $\mathbf T$.

  \item
  Draw the transition diagram of $\mathbf S$.
  What can we say about this?
  Is the process strictly random?
  Is the Perron-Frobenius theorem still applicable?
  Reconsider part \ref{prob1:eigneg1}.

  \item
  Consider the attenuated transition matrix
  $$
  \mathbf T'
  =
  \frac 1 2
  \left(
    \mathbf I
    +
    \mathbf T
  \right)
  =
  \left(
    \begin{array}{ccc}
      0.5   &   0.2   &   0     \\
      0.5   &   0.5   &   0.5   \\
      0     &   0.3   &   0.5
    \end{array}
  \right)
  .
  $$
  Show that the eigenvectors of $\mathbf T$
  are also the eigenvectors of $\mathbf T'$.
  What are the eigenvalues of $\mathbf T'$?


\end{enumerate}



\subsection{Coarse-graining}



This is a generalization of the coarse graining process
in part \ref{prob1:coarsegraining} of the previous problem.
%
Each coarse-graining
can be down by successively
merging two states into one,
one step at a time.
%
But in some step, we may have more than
three states in total.
%
What are the generalized version of Eq. \eqref{eq:coarse-graining}?
%
\begin{enumerate}
  \item
  Show that the coarse-grained matrix satisfies
  the three basic rules of a transition matrix:
  (i) $S_{ij} \ge 0$,
  (ii) $\sum_i S_{ij} = 1$,
  (iii) $\sum_j S_{ij} \, P^s_j = P^s_i$.
  Here, $\mathbf P^s$ is the distribution
  of the coarse-grained not the original states.

  \item
  If the original matrix satisfies detailed balance,
  show that the coarse-grained matrix also satisfies
  detailed balance.
\end{enumerate}


\section{Problem Set 2: Monte Carlo simulations}

\subsection{Binomial random number}



This is a programming task.
%
Using Markov-chain Monte Carlo,
generate a random number $i$ that satisfies the binomial distribution:
$$
p_i = c \, {N \choose i}
\qquad
0 \le i \le N
,
$$
where $c$ is a normalization factor.

Actually, $N$ does not have to be an integer.
Modify your code such that it may work for a non-integral $N$,
say $N = 10.5$.

Hint. We do not have to worry about $c$.
Only the ratios matter in a Monte Carlo simulation, and
$$
\frac{ p_{i+1} } { p_i } = \frac{ N - i } { i + 1 },
\qquad
\frac{ p_{i-1} } { p_i } = \frac{ i } { N - i + 1}.
$$

A sample C code is here: \url{http://ideone.com/CgL1FI}.



\subsection{Autocorrelations}

[Data from a Monte Carlo or molecular dynamcis trajectory are correlated.
  The redundancy $g$ because of the correlation
  is related to the integral autocorrelation time $\tau$
  by the following formula
  \begin{equation}
    g = 1 + 2 \, \tau.
  \label{eq:1p2T}
  \end{equation}
  This exercise is a demonstration of this formula.
]

Alice flips a coin every day.
%
If the result is a head, she writes $+1$ on her notebook;
otherwise, she writes $-1$.
%
At the end of the year, she adds the numbers up.

Bob does pretty much the same thing, except that he is lazier.
%
So every day there is two thirds chance that he does not flip the coin
and simply copies yesterday's result.
%
At the end of the year, Bob also adds the numbers up.

\begin{enumerate}
  \item
  Show that on average Bob's end-of-year number
  is roughly $\sqrt 5$ as large as Alice's
  in magnitude (no matter the sign).
  %
  In other words, Bob's numbers has
  a statistical redundancy of $g = 5$.

  \item
  Show that the autocorrelation time of Bob's numbers
  is $\tau = 2$.

  \item
  Verify Eq. \eqref{eq:1p2T}.
\end{enumerate}

Here is a demonstration:
\url{http://ideone.com/hH7U6Y}.


\end{document}

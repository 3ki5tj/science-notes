\documentclass[11pt]{article}
\usepackage{amsmath}
\usepackage[width=7.0in, height=10.0in]{geometry}



\begin{document}

\title{ Velocity rescaling thermostat }
\author{ \vspace{-10ex} }
\date{ \vspace{-10ex} }
\maketitle



The velocity rescaling thermostat is a modern and popular thermostat,
which is widely used in molecular dynamics (MD) simulations.
%
Here, a ``thermostat'' is a piece of computer code
that helps the MD program to sample a canonical distribution.
%
We will explain how the thermostat works in this document.



\section{Langevin equation}



We start with a simple exercise,
that is, to solve the equation of motion
of an over-damped harmonic oscillator
under a random noise:
%
\begin{equation}
  \dot x = -\gamma x + \sqrt{ 2 \gamma } \sigma \xi,
  \label{eq:ho_langevin}
\end{equation}
where,
$\dot x = dx/dt$,
$\gamma$ and $\sigma$ are two constants,
and $\xi$ is a unit Gaussian white noise:
\begin{align}
  \langle \xi(t) \xi(t') \rangle
=
  \delta(t - t').
\end{align}
Equation \eqref{eq:ho_langevin} is the Langevin equation.

Since Eq. \eqref{eq:ho_langevin} involves a random noise,
it is meaningless to discuss a particular trajectory.
%
Instead, we can imagine an \emph{ensemble} with the same initial condition,
and then study the averaged trajectory
and its variance at different time frames.
%
This is to say that we should talk about the distribution
of $x$, $\rho(x, t)$ as a function of $t$.
%
At time $t = 0$, we can assume that the distribution
is focused at $x = x_0$:
%
\begin{equation}
  \rho(x, t = 0) = \delta(x - x_0).
  \label{eq:rho_t0}
\end{equation}



\section{Fokker-Planck equation}



We now give a useful theorem without giving a proof.
%
If $x$ follows the Langevin equation
\begin{equation}
  \dot x = -v(x) + \sqrt{ 2 B(x) } \, \xi,
  \label{eq:x_langevin}
\end{equation}
then the distribution $\rho(x, t)$ satisfies
\begin{equation}
\frac { \partial \rho } { \partial t }
=
\frac { \partial } { \partial x }
\left\{
  v(x) \, \rho
  +
\frac { \partial  } { \partial x } \Bigl[ B(x) \rho \Bigr]
\right\}.
\label{eq:x_fp}
\end{equation}

An immediate consequence of Eq. \eqref{eq:x_fp} is that
the stationary distribution, $\rho^*$, satisfies
\begin{equation*}
  \frac { d } { d x } \bigl[ B(x) \rho^*(x) \bigr]
=
  - v(x) \, \rho^*(x)
\end{equation*}
or
\begin{equation}
  \frac { d } { d x } \log\bigl[ B(x) \rho^*(x) \bigr]
=
  - \frac{ v(x) }{ B(x) }.
\label{eq:rhost_fp}
\end{equation}

Equation \eqref{eq:rhost_fp} is useful.
%
If we know $v(x)$ and $B(x)$,
we can then find out the stationary distribution $\rho^*(x)$
by solving Eq. \eqref{eq:rhost_fp}.
%
Conversely,
if we wish to design a dynamic equation of $x$
that ultimately converges to some predefined distribution $\rho^*(x)$.
%
Then, $v(x)$ and $B(x)$ are constrained by Eq. \eqref{eq:rhost_fp},
although we still have the freedom to choose a convenient $B(x)$.



\section{Stationary solution of the over-damped oscillator}



Now let us apply Eq. \eqref{eq:rhost_fp} to Eq. \eqref{eq:ho_langevin}.
%
We identify that
$v(x) = \gamma x$,
and
$B(x) = \gamma \sigma^2$.
%
Thus, the stationary distribution, $\rho^*(x)$,
\[
  \frac { d } { d x } \log\bigl[ \gamma \sigma^2 \rho^*(x) \bigr]
=
  - \frac{ x }{ \sigma^2 }.
\]
This equation is readily solved,
\[
  \rho^*(x)
=
  - \frac{ x^2 }{ 2 \sigma^2 }
  + \mathrm{const.},
\]
which,
after the normalization $\int \rho^*(x) \, dx = 1$,
yields
\begin{align}
\rho^*(x)
=
\frac{ 1 } { \sqrt{ 2 \pi \sigma^2 } }
\exp\left(
  -\frac{ x^2 } { 2 \sigma^2 }
\right).
\label{eq:ho_rhost}
\end{align}



\section{General solution of the over-damped oscillator}



Now let us solve the time-dependent distribution, $\rho(x, t)$.
%
We shall assume the initial condition given by Eq. \eqref{eq:rho_t0},
%
and solve Eq. \eqref{eq:x_fp}.

Inspired by the stationary distribution, Eq. \eqref{eq:ho_rhost},
we shall use the trial solute
%
\begin{align}
\rho(x, t)
=
\frac{ 1 } { \sqrt{ 2 \pi b(t) } }
\exp \left\{
  -\frac{ [x - c(t)]^2 } { 2 b(t) }
\right\}.
\label{eq:ho_rhot_trial}
\end{align}
%
We expect that $c(t) \rightarrow 0$, and $b(t) \rightarrow \sigma^2$
in the limit of $t \rightarrow \infty$.

Plugging Eq. \eqref{eq:ho_rhot_trial} into Eq. \eqref{eq:x_fp},
we get
\begin{multline}
  \frac{ 1 } { \sqrt{ 2 \pi b } }
  \exp \left[
    - \frac{ (x - c)^2 } { 2 b }
  \right]
  \left\{
    \left[
      \frac { (x - c)^2 } { b } - 1
    \right]
    \frac{ 1 } { 2 b }
    \frac{ d b } { d t }
    +
    \frac { x - c } { b }
    \frac { d c } { d t }
  \right\} \\
=
  \frac{ 1 } { \sqrt{ 2 \pi b } }
  \exp \left[
    - \frac{ (x - c)^2 } { 2 b }
  \right]
  \left\{
    \left[
      \frac{ (x - c)^2 } { b } - 1
    \right]
    \, \gamma \,
    \left(\frac{ \sigma^2 } b - 1\right)
    -
    \frac{ (x - c) } { b } \, \gamma \, c
  \right\}.
\end{multline}
%
Thus, for the trial solution to be valid,
we must have
\begin{align*}
\frac{ 1 } { 2 b }
\frac{ d b } { d t }
&=
\gamma \,
\left(\frac{ \sigma^2 } b - 1\right),
\\
\frac { d c } { d t }
&=
-\gamma \, c.
\end{align*}
Solving the above two equations, we get
\begin{align*}
b &= \sigma^2 [1 - \exp( - 2 \gamma t)],
\\
c &= c(0) \exp(-\gamma t) = x_0 \, \exp(-\gamma t).
\end{align*}
%
Thus, the solution is
%
\begin{align}
\rho(x, t)
=
\frac{ 1 } { \sqrt{ 2 \pi \sigma^2 (1 - e^{-2\gamma t}) } }
\exp \left[
  -\frac{ ( x - x_0 \, e^{-\gamma t} )^2 } { 2 \sigma^2 (1 - e^{-2\gamma t}) }
\right].
\label{eq:ho_rhot}
\end{align}
Another way to put this is that
the solution at time $t$ is given by
a Gaussian centered at $x_0 \, e^{-\gamma t}$,
with a variance of $\sigma^2 (1 - e^{-2\gamma t})$.



\begin{thebibliography}{100}

\bibitem{singer}
  Giovanni Bussi, Davide Donadio and Michele Parrinello,
  ``Canonical sampling through velocity rescaling,''
  J. Chem. Phys., Vol. 126, 014101,
  2007.

\end{thebibliography}



\end{document}

\documentclass[11pt]{article}
\usepackage{amsmath}
\usepackage[width=7.0in, height=10.0in]{geometry}



\begin{document}



\section{Harmonic oscillator chain under the open boundary condition }



The set of equations we want to solve is
\begin{equation}
\begin{split}
M \ddot x_1 &= -K x_1 + K x_2 \\
M \ddot x_2 &= \hphantom{-}  K x_1 - 2 K x_2   + K x_3 \\
            &\hphantom{-} \vdots \\
M \ddot x_n &= \hphantom{-K x_1 -2}        K x_{n-1}  - K x_n.
\end{split}
\label{eq:eos}
\end{equation}


We will use the following trial solution:
\begin{align}
  x_i = \cos[a (i - 1) + b - \omega t],
  \label{eq:trialsol}
\end{align}
where the parameters $a$, $b$ and $\omega$ are to be determined.
Note that the solution can be multiplied by an arbitrary constant $c$
without changing its nature.



Substituting Eq. \eqref{eq:trialsol} in Eq. \eqref{eq:eos} yields
\begin{align}
-M \omega^2 x_1 &= -K x_1 + K x_2 \label{eq:eos1} \\
                &\hphantom{-} \vdots \notag \\
-M \omega^2 x_i &= \hphantom{-}  K x_{i-1} - 2 K x_i + K x_{i+1}
  \label{eq:eosi} \\
                &\hphantom{-} \vdots \notag \\
-M \omega^2 x_n &= \hphantom{-K x_1 -2}        K x_{n-1}  - K x_n.
  \label{eq:eosn}
\end{align}
Now that we have get rid of the time dependence,
we shall set $t$ to 0 below.

To determine parameters,
we first use Eq. \eqref{eq:trialsol} in Eq. \eqref{eq:eosi}:
\[
-M \omega^2 \cos\phi
=
K \cos(\phi - a)
-2 K \cos\phi
+K \cos(\phi + a),
\]
where $\phi = (i-1)a + b$.
Using $\cos(\phi \pm a) = \cos\phi \cos a \mp \sin\phi \sin a$, we get
\begin{align}
\omega
=
\sqrt \frac { K (2 - 2 \cos a) } M
=
2 \sqrt \frac K M \left| \sin \frac a 2 \right|.
\label{eq:omega_a}
\end{align}


Next, we use Eq. \eqref{eq:trialsol} in Eq. \eqref{eq:eos1}:
\begin{equation}
-M \omega^2 \cos b
=
-K \cos b
+K \cos(b + a).
\label{eq:a_b_step1}
\end{equation}
Further, by using Eq. \eqref{eq:omega_a} in Eq. \eqref{eq:a_b_step1},
we get
\begin{equation*}
(1 - \cos a) \cos b
= \sin a \sin b.
\end{equation*}
or, as
$1 - \cos a = 2 \sin^2(a/2)$
and
$\sin a = 2 \sin(a/2) \, \cos(a/2)$,
\begin{equation*}
\tan \left( \frac a 2 \right)
= \tan b,
\end{equation*}
which means
\begin{equation}
b = \frac a 2 + m \pi,
\label{eq:a_b}
\end{equation}
with $m$ being an arbitrary integer.



Similarly, we can use Eq. \eqref{eq:trialsol} in Eq. \eqref{eq:eosn}
\begin{equation*}
-M \omega^2 \cos \phi_n
=
K \cos(\phi_n - a)
-K \cos \phi_n,
\end{equation*}
where $\phi_n = (n - 1) a + b$.
%
Again, by using Eq. \eqref{eq:omega_a}, we get
\[
(\cos a - 1) \cos \phi_n = \sin \phi_n \sin a.
\]
or
\[
\tan \left( \frac a  2 \right) = -\tan \phi_n,
\]
which means
\begin{align}
m' \pi - \frac a 2 = \phi_n = (n - 1) a + b.
\label{eq:a_phin}
\end{align}


Now the three equations \eqref{eq:omega_a}, \eqref{eq:a_b} and \eqref{eq:a_phin}
determines the three unknowns $a$, $b$ and $\omega$.
%
Using Eq. \eqref{eq:a_b} in Eq. \eqref{eq:a_phin} yields
\begin{equation*}
a = \frac{ p \, \pi } n,
\end{equation*}
where $p = m' - m$ is an integer.
%
The frequency $\omega$ is then given by \eqref{eq:omega_a}.
\begin{equation}
\omega_p
=
2 \sqrt \frac K M
\left|
\sin \left( \frac {p \, \pi} {2 \, n} \right)
\right|.
\label{eq:freq}
\end{equation}
where we have attached the subscript $p$ to the frequency $\omega$.
%
In the case of $n = 3$ oscillators, we have
$\omega_0 = 0$,
$\omega_1 = \sqrt{K/M}$,
and
$\omega_2 = \sqrt{3K/M}$.


So, the final solution, for mode $p$, is
\begin{align}
  x_i^{(p)}
=
  c_p \cos\left[
  \frac{ (2 i - 1) p \, \pi } { 2 n }
  -2 \sqrt \frac K M
  \left|
  \sin \left( \frac {p \, \pi} {\mathbf 2 \, n} \right)
  \right| t
  \right],
  \label{eq:obcsol}
\end{align}
where $c_p$ is an arbitrary constant.
%
The general solution is a linear combination of different modes.



\section{Periodic boundary condition}



Let us do the same problem under the periodic boundary condition (PBC).
%
The only difference is that Eqs. \eqref{eq:eos1} and \eqref{eq:eosn}
are now replaced by Eq. \eqref{eq:eosi}
with $i = 1$ and $i = n$, respectively.
%
Thus Eq. \eqref{eq:omega_a} remains valid;
but Eqs. \eqref{eq:a_b} and \eqref{eq:a_phin}
are now replaced by the condition that
\[
  x_{n+i} = x_i.
\]
This is possible only if
\begin{equation*}
  a = \frac{ 2 \pi p } { n },
\end{equation*}
or, equivalently,
\begin{equation}
\omega_p^\mathrm{(PBC)}
=
2 \sqrt \frac K M
\left|
\sin \left( \frac {p \, \pi} {n} \right)
\right|,
\label{eq:freq_pbc}
\end{equation}
%
and, the final solution, for mode $p$, is
\begin{align}
  x_i^{(p), \mathrm{PBC}}
=
  c'_p \cos\left[
  \frac{ 2 (i - 1) p \, \pi } { n }
  + b
  -2 \sqrt \frac K M
  \left|
  \sin \left( \frac {p \, \pi} {n} \right)
  \right| t
  \right],
  \label{eq:obcsol}
\end{align}
where $c'_p$ is an arbitrary constant,
and $b$ is arbitrary.
%


Comparing Eq. \eqref{eq:freq_pbc} with Eq. \eqref{eq:freq},
we note the missing factor of 2 in the frequency.
%
This means that the PBC case misses
half of the frequencies.
%
However, in the PBC case,
each frequency corresponds to two independent modes,
one sine mode and one cosine mode;
and any linear combination of the two modes
is a valid solution as well.
%
Mathematically, it corresponds to the fact that
$\cos b$ and $\sin b$ serve as the coefficients of combination:
\[
  \cos(\phi + b) = \cos b \, \cos \phi - \cos b \, \cos \phi.
\]
This degeneracy is absent in the open boundary condition (OBC) case;
consequently, the OBC frequencies are finer.


\end{document}

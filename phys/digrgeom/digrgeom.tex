\documentclass{article}
\usepackage{amsmath}
\usepackage{bm}
\usepackage[usenames,dvipsnames]{xcolor}
\usepackage{tikz}
\usepackage{hyperref}

\hypersetup{
    colorlinks,
    linkcolor={red!30!black},
    citecolor={blue!50!black},
    urlcolor={blue!80!black}
}



\begin{document}

\title{}
\author{ \vspace{-10ex} }
\date{ \vspace{-10ex} }
\maketitle


\section{\S 3.4 A digression on geometry}


\subsection{\S 3.4.1 Some geometry}

\paragraph{Contravariant and covariant vectors.}

Let us fix a point $\mathbf x$.
$$
df
= \sum_i \frac{\partial f}{\partial x^i} dx^i
= (\nabla f, d\mathbf x).
$$
Now under a coordinate transform,
if a vector behaves like $d\mathbf x$
then it is a contravariant vector, or a vector field.

If a vector behaves like $\nabla f$,
then it is a covariant vector,
which characterize a 1-form.
%
That is, we generalize a vector that behaves like
$(\partial_1 f, \dots, \partial_n f)$
to a general covariant vector
$(\alpha_1, \dots, \alpha_n)$;
this vector corresponds to a 1-form
$\alpha_1 \, dx^1 + \cdots + \alpha_n \, dx^n$.
Note in this definition,
$dx^1, \cdots, dx^n$ only
are placeholders to accept the components
of an input tangent vector $d\mathbf x = (dx_1, \cdots, dx_n)$.



\paragraph{Dirac notation.}

The \emph{value} of a 1-form is denoted by the Dirac notation
$$
\alpha(X)
=
\langle \alpha, X \rangle
=
\sum_i \alpha_i \, X^i.
$$
Particularly, a total differential is a 1-form
$$
df(X) = \sum_i \partial_i f \, X^i
= ( \nabla f, \mathbf X ) = \langle df, \mathbf X \rangle.
$$


\subsubsection{Lie derivative}

The above expression can also be expressed as
$$
L_{\mathbf X} f,
$$
where
$$
L_{\mathbf X} \equiv \sum_i X^i \partial_i
$$
is the differentiation along the field,
or the \emph{Lie derivative} along $\mathbf X$.

The Lie derivative of a scalar yields a scalar.
The Lie derivative of a vector field (contravariant vector) yields a vector field (contravariant).
The Lie derivative of a 1-form (covariant vector) yields a 1-form (covariant vector).
%
The Lie derivative of any object yields an object of the same kind.


We also introduce the notation
$$
L_{\mathbf X} f \equiv \mathbf X(f).
$$

\paragraph{Lie derivative applied to a total differential}

Lie derivative applied to a total differential
$$
L_{\mathbf X} dx^i
= L_{\mathbf X} (x^i_B - x^i_A)
= (L_{\mathbf X} x^i)_B
- (L_{\mathbf X} x^i)_A
= d (L_{\mathbf X} x^i).
$$
where $\mathbf x_B = \mathbf x_A + d\mathbf x$.
%
Generally, for a total differential $df$
$$
L_{\mathbf X} df
= L_{\mathbf X} (f(\mathbf x_B) - f(\mathbf x_A))
= (L_{\mathbf X} f)_B
- (L_{\mathbf X} f)_A
= d (L_{\mathbf X} f).
$$
This means that we can freely exchange the order of
the Lie derivative and $d$.

\paragraph{Leibniz rule}

Let us apply the Lie derivative to a scalar $\langle \omega, \mathbf X\rangle$:
$$
L_{\mathbf Y} \langle\omega, \mathbf X\rangle
=
\langle L_{\mathbf Y}\omega, \mathbf X\rangle
+
\langle \omega, L_{\mathbf Y} \mathbf X \rangle.
$$
It can be shown that
$$
L_{\mathbf Y} \mathbf X
= L_{\mathbf Y} L_{\mathbf X} - L_{\mathbf X} L_{\mathbf Y}
\equiv L_{[\mathbf X, \mathbf Y]}.
$$
Thus, we can show the Jacobi identity:
$$
[[\mathbf X, \mathbf Y], \mathbf Z]
+[[\mathbf Y, \mathbf Z], \mathbf X]
+[[\mathbf Z, \mathbf X], \mathbf Y] = 0.
$$


\subsection{Euler-Lagrange equations}


In mechanics $\mathbf x \rightarrow (\mathbf q, \dot{\mathbf q}$,
and $dx^i \rightarrow q^\alpha, \dot q^\alpha$.
%
For a Lagrangian $L=L(q^\alpha, \dot q^\alpha)$,
define a 1-form,
$$
\theta_L = \frac{ \partial L } { \partial \dot q^\alpha } d q^\alpha.
$$
This 1-form only has components for $q^\alpha$ but not those for $\dot q^\alpha$.

Consider the vector field for time evolution
$$
\pmb \Delta =
\frac{d}{dt} (\mathbf q, \dot{\mathbf q})
=
(\dot{\mathbf q}, \ddot{\mathbf q})
.
$$
Then
$$
L_{\pmb \Delta} f = \frac{df}{dt}.
$$


$$
\begin{aligned}
L_{\pmb \Delta} \theta_L
&=
\sum_\alpha
  \left(L_{\pmb \Delta} \frac{ \partial L } { \partial {\dot q}^\alpha } \right) d q^\alpha
+
  \frac{ \partial L } { \partial \dot q^\alpha } ( L_{\pmb \Delta} d q^\alpha )
\\
&=
\sum_\alpha
  \frac{d}{dt}\left( \frac{ \partial L } { \partial {\dot q}^\alpha} \right) d q^\alpha
+
  \frac{ \partial L } { \partial {\dot q}^\alpha } d {\dot q}^\alpha
\\
&=
\sum_\alpha
  \frac{ \partial L } { \partial q^\alpha} d q^\alpha
+
  \frac{ \partial L } { \partial {\dot q}^\alpha } d {\dot q}^\alpha
= dL
\end{aligned}
$$
This is the coordinates-independent form of the Euler-Lagrange equations.
\begin{equation}
L_{\pmb \Delta} \theta_L = dL
\end{equation}
This equation contains multiple components, so it consists of multiple equations.

Proof of
$
L_{\mathbf Y} \mathbf X
=
L_{\mathbf Y} L_{\mathbf X} - L_{\mathbf X} L_{\mathbf Y}.
$

Key steps.
$$
\begin{aligned}
(L_{\mathbf Y} L_{\mathbf X} - L_{\mathbf X} L_{\mathbf Y})f
&=
L_{\mathbf Y} \langle df, \mathbf X\rangle
- \langle d(L_{\mathbf Y} f), \mathbf X \rangle
\\
&=
\langle L_{\mathbf Y} df, \mathbf X\rangle
+\langle df, L_{\mathbf Y} \mathbf X\rangle
- \langle d(L_{\mathbf Y} f), \mathbf X \rangle
\\
&=
\langle df, L_{\mathbf Y} \mathbf X\rangle.
\end{aligned}
$$
Compare this to
$$
\begin{aligned}
L_{\mathbf Z} f = \langle df, \mathbf Z \rangle.
\end{aligned}
$$


Then
$$
L_{\mathbf Y} \mathbf X (f g)
=
f (L_{\mathbf Y} \mathbf X g)
+
g (L_{\mathbf Y} \mathbf X f).
$$
There is a theorem that says that if a map $\phi$ satisfies
$$
\phi (f g)
=
f \phi g
+
g \phi f,
$$
then $\phi$ defines a vector field $\mathbf X$
with $\phi = L_{\mathbf X}$.


\end{document}

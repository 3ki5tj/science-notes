\documentclass{article}
\usepackage{amsmath}
\usepackage{amsthm}
\usepackage{amsfonts}
\usepackage{chngcntr}
\usepackage{caption}
\usepackage{bm}
\usepackage[usenames,dvipsnames]{xcolor}
\usepackage{tikz}

\begin{document}

\title{Brownian dynamics with a colored noise}
\author{ \vspace{-10ex} }
\date{ \vspace{-10ex} }
\maketitle

\subsection*{Problem}

Consider the Langevin
\begin{equation}
\dot x = -\alpha \, x + \sqrt{\Gamma} \, \xi,
\label{eq:Langevin_colorednoise}
\end{equation}
where $\xi$ is a colored Gaussian noise satisfying
$\langle \xi(t) \, \xi(t') \rangle = e^{-\gamma \, |t-t'|}$.
What is the Fokker-Planck equation for $x$?

\subsection*{Solution.}

\paragraph{Interpretation as Ornstein-Uhlenbeck process.}

Note that the stationary distribution of $\xi$ is
\begin{equation}
P(\xi) = e^{-\frac{1}{2}\xi^2}/\sqrt{2\pi}.
\label{eq:Pxi}
\end{equation}
%
The idea [van Kampen (page 243)] is
to treat $\xi$ as a Ornstein-Uhlenbeck process
(i.e., the position of a harmonic oscillator under
a white-noise Brownian dynamics).
%
So
\begin{equation}
  \begin{aligned}
    \dot x &= -\alpha \, x + \sqrt{\Gamma} \, \xi, \\
    \dot \xi &= -\gamma \, \xi + \sqrt{2 \gamma} \, d W(t),
  \end{aligned}
  \label{eq:ho_colorednoise_x}
\end{equation}
where $W(t)$ is the Wiener process (so $dW(t)$ is a white noise).

Below we will show that
if initially $x$ and $\xi$ are independent
and the distribution of $\xi$ is given by Eq. \eqref{eq:Pxi},
the marginal distribution $P(x, t)$ satisfies
\begin{equation}
\frac{ \partial P } { \partial t } = \frac{ \partial } { \partial x } (x P)
+
\frac{ 1 - e^{-(\alpha+\gamma) t} } {\alpha + \gamma} \Gamma
\frac{ \partial^2 P } { \partial x^2 }.
\label{eq:ho_colorednoise_Px}
\end{equation}

\paragraph{Distribution $P(x, t)$ is a Gaussian.}

The distribution $P(x, \xi, t)$ for
Eq. \eqref{eq:ho_colorednoise_x} satisfies
$$
\frac{ \partial P(x, \xi, t) } { \partial t }
=
\frac{ \partial } { \partial x }
\left[
  \left(\alpha \, x - \sqrt{\Gamma} \, \xi \right) \, P(x, \xi, t)
\right]
+
\frac{ \partial } { \partial \xi }
\left[
  (\gamma \, \xi ) \, P(x, \xi, t)
\right]
+
\gamma
\frac{ \partial^2 } { \partial \xi^2} P(x, \xi, t).
$$
Since it is a linear Fokker-Planck equation,
the solution is a joint Gaussian distribution
for $x$ and $\xi$.
%
Further, the deterministic part for $x$
is unaffected by the colored noise $\xi$.
%
So the marginal distribution of $P(x, t)$
must adopt the form of
\begin{equation}
\frac{ \partial P(x, t) } { \partial t }
=
\frac{ \partial } { \partial x }
\left[
  \alpha \, x \, P(x, t)
\right]
+
B(t)
\frac{ \partial^2 } { \partial x^2} P(x, t),
\label{eq:ho_colorednoise_Px1}
\end{equation}
where $B(t)$
is a function to be determined
(which would be a constant if the noise were white).

\paragraph{Determination of the variance.}

By multiplying Eq. \eqref{eq:ho_colorednoise_Px1} with $x^2$ and integrating,
we get
%
\begin{align}
\frac{ \partial \langle x^2 \rangle } { \partial t }
&=
\int x^2 \frac{ \partial [(\alpha \, x) \, P(x, t) ] } { \partial x} \, dx
+ B(t) \int x^2 \frac{ \partial^2 P(x, t) } { \partial x^2 } \, dx
\notag
\\
&=
-\int 2 \, x \, (\alpha \, x) \, P(x, t) \, dx
- 2 \, B(t) \int x \frac{ \partial P(x, t) } { \partial x } \, dx
\notag
\\
&=
- 2 \, \alpha \, \langle x^2 \rangle
+ 2 \, B(t).
\label{eq:dxxdt_Langevin}
\end{align}

We will find $B(t)$ below.

\paragraph{Method 1}

The explicit solution of Eq. \eqref{eq:Langevin_colorednoise} is
(assuming $x(0) = 0$),
%
\begin{equation}
  x(t) = \sqrt{\Gamma} \int_0^t \xi(\tau) \, e^{-\alpha(t-\tau)}  \, d\tau.
\label{eq:xt_solution}
\end{equation}
%
Multiplying \eqref{eq:Langevin_colorednoise} by $2 \, x$ and averaging, we get
%
\begin{align}
  \frac{ d \langle x^2 \rangle } { d t }
  &=
  -2\alpha \langle x^2 \rangle
  + 2 \, \sqrt\Gamma
  \langle x(t) \, \xi(t) \rangle
  \notag
  \\
  &=
  -2\alpha \langle x^2 \rangle
  + 2 \, \Gamma
  \int_0^t \langle \xi(\tau) \, \xi(t) \rangle e^{-\alpha(t- \tau} \, d\tau
  \notag
  \\
  &=
  -2\alpha \langle x^2 \rangle
  + 2 \, \Gamma
  \int_0^t e^{-\gamma(t - \tau)} e^{-\alpha(t - \tau} \, d\tau
  \notag
  \\
  &=
  -2\alpha \langle x^2 \rangle
  + 2 \frac{ \Gamma } { \gamma + \alpha }
  \left(
    1 - e^{-(\gamma + \alpha) \, t}
  \right).
\end{align}
%
Compare this to \eqref{eq:dxxdt_Langevin},
we conclude
\begin{equation}
B(t) =
  \frac{ \Gamma } { \gamma + \alpha }
  \left(
    1 - e^{-(\gamma + \alpha) \, t}
  \right).
  \label{eq:Bt_colorednoise}
\end{equation}
So the Fokker-Planck equation for $P(x, t)$ is given by
Eq. \eqref{eq:ho_colorednoise_Px}.



\paragraph{Method 2}

The following method use Laplace transform.
%
Laplace transforming \eqref{eq:dxxdt_Langevin} yields
\begin{equation}
\mathcal L[ B(t) ](s) = \frac{2 \, \alpha + s}{2} \, \mathcal L[\langle x^2 \rangle](s).
\label{eq:Laplace_B_xx}
\end{equation}

On the other hand,
by \eqref{eq:xt_solution}, the variance is given by
$$
\begin{aligned}
\langle x^2 \rangle
&=
\int_0^t\int_0^t \langle \xi(\tau) \, \xi(\tau') \rangle
e^{-\alpha(t-\tau)-\alpha(t'-\tau')} \, d\tau \, d\tau'
\\
&=
\Gamma \int_0^t\int_0^t
e^{-\gamma|\tau - \tau'|-\alpha(t-\tau)-\alpha(t'-\tau')} \, d\tau \, d\tau'
\\
&=
  \frac{\Gamma }{\alpha\,(\gamma + \alpha) }
  +
  \frac{2 \, \Gamma \, e^{-(\gamma + \alpha) \, t} }{(\gamma - \alpha) (\gamma + \alpha) }
  -
  \frac{\Gamma \, e^{-2 \, \alpha \, t} }{(\gamma - \alpha) \, \alpha }.
\end{aligned}
$$
So, the Laplace transform is
$$
\mathcal L[\langle x^2 \rangle](s)
=
  \frac{ \Gamma } { \alpha \, (\gamma + \alpha) \, s }
  +
  \frac{ 2 \, \Gamma } { (\gamma - \alpha) \, (\gamma + \alpha) \, (\gamma + \alpha + s) }
  -
  \frac{ \Gamma } { (\gamma - \alpha ) \, \alpha \, (2 \, \alpha + s) }.
$$
Using this in \eqref{eq:Laplace_B_xx}, we get
$$
\mathcal L[B(t)](s) =
  \frac{ \Gamma }{ \gamma + \alpha }
  \left( \frac{1}{s} - \frac{1}{ \gamma + \alpha + s } \right),
$$
which means \eqref{eq:Bt_colorednoise}.

\paragraph{Discussion.}

This example is instructive,
it shows that if the random noise is colored.
%
The variance term $B(t)$ in the Fokker-Planck equation
depends non-trivially on the deterministic force through $\alpha$.
%
Thus, for a non-harmonic force, the generally solution
can be hard to find.


\end{document}

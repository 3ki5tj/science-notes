\documentclass{article}
\usepackage{amsmath}
\begin{document}

\section{Chaiken PCMP, Critical dimensions of phase transition}

%I am trying to follow the mean-field argument for that the fluctuation is unimportant in a phase transition for dimensions $d \ge 4$ in
%[Principles of Condensed Matter Physics]
%(http://www.amazon.com/Principles-Condensed-Matter-Physics-Chaikin/dp/0521794501/) by Chaikin and Lubensky.
The book gives two interesting and important results of phase transitions.
%
The first result is that
fluctuation is unimportant in a phase transition for dimensions $d \ge 4$,
which means that the mean-field theory works almost perfectly for $d \ge 4$.
%
The second result is that
fluctuation is too vehement to permit a phase transition for $d \le 2$.

These arguments are profound.
Below we shall derive them from simple arguments of a mean-field theory,
which were simplified from those in the book.


\subsection{Mean-field theory and phase transition}



The theory of phase transition is built on the idea of minimizing
a free-energy functional of a field $\phi(x)$:
%
\begin{align}
F
&= \int_V \left[
\frac{1}{2}r \, \phi^2(\mathbf x) + c (\nabla \phi(\mathbf x))^2
+ u \, \phi^4(\mathbf x)
\right] \, d^d \mathbf x,
\label{eq:Fx}
\end{align}
where $c, u > 0$.

In the mean-field theory, we assume a uniformly field.
Then the term $\nabla \phi = \mathbf 0$, and
\begin{align}
F
&= V \left(
\frac{1}{2}r \, \phi^2 + u \, \phi^4
\right),
\end{align}
where $V$ is the volume.


If $r > 0$, the minimal of $F$ is achieved at $\phi = 0$.

If $r < 0$, the minimal of $F$ is achieved at a nonzero $\phi = \sqrt{4u/|r|}$,
which is the solution of
$$
\partial F/\partial \phi = V \, (r\phi + 4 u \phi^3) = 0.
$$



The theory is that $r$ changes sign around the critical temperature $T = T_c$.
%
Thus the behaviors for $T > T_c$ and $T < T_c$ are qualitatively different.


Next, let us see how the fluctuation term modifies the picture.



\subsection{Expansion of small fluctuation}



Assuming that the fluctuation is small, we can write
$\phi(\mathbf x) = \langle \phi \rangle + \delta \phi(\mathbf x)$
and
\begin{align}
F
&\approx
V \left(
\tfrac{1}{2} r \, \langle \phi \rangle^2 + u \, \langle \phi \rangle^4
\right)
\notag
\\
&+
\int_V \left[
\tfrac{1}{2} r \, \delta \phi^2(\mathbf x)
+ c (\nabla \delta \phi(\mathbf x))^2
+ 6 u \, \langle \phi \rangle^2 \delta \phi^2(\mathbf x)
+ u \, \delta \phi^4(\mathbf x)
\right] \, d^d \mathbf x. \quad
\label{eq:F_fluc1}
\end{align}
The mean-field theory searches the minimum of the first term, which gives
\begin{align}
  \langle \phi \rangle^2 = \frac{|r|}{4\,u} = - \frac{r}{4\,u},
\label{eq:phi_mf}
\end{align}
for a negative $r$ under the critical temperature.
Then we want to see when it is valid to drop the second term
involving the integral.


Using \eqref{eq:phi_mf} in \eqref{eq:F_fluc1},
we simplify the fluctuation term as
\begin{align}
F_\mathrm{fluc}
&=
\int_V \left[
|r| \, \delta \phi^2(\mathbf x)
+ c (\nabla \delta \phi(\mathbf x))^2
+ u \, \delta \phi^4(\mathbf x)
\right] \, d^d \mathbf x. \quad
\label{eq:F_fluc2}
\end{align}



\subsection{When the fluctuation is unimportant}


Particularly, we wish to see if
\begin{align}
\langle \delta\phi^2(\mathbf x) \rangle
< \langle \phi \rangle^2,
\label{eq:phicmp}
\end{align}
around the critical temperature, where $r \approx 0$, and $\langle \phi \rangle \approx 0$.
We wish to find $\langle \delta^2 \phi(\mathbf x) \rangle$
and compare it with the value given in Eq. \eqref{eq:phi_mf}.


\subsection{Scaling argument for the upper critical dimension}

This can be done by normalizing the fluctuation part of the free energy with dimensionless coordinates and field
\begin{align}
\mathbf X &= \frac{ \mathbf x }{ \xi }, \\
\Phi(\mathbf X) &= \frac{ \delta \phi(\mathbf x) }{A},
\end{align}
where $\xi$ is the coherent length.
%
Then we get,
$$
\begin{aligned}
F_\mathrm{fluc}
&=
\int_{V/\xi^d} \left[
|r| \, A^2\, \xi^d \Phi^2(\mathbf X)
+ c \, A^2 \, \xi^{d-2} \left(
\frac{ d \Phi(\mathbf X) }{ d \mathbf X} \right)^2
+ u \, A^4 \, \xi^d \, \Phi^4(\mathbf X)
\right] \, d^d \mathbf X.
\end{aligned}
$$

Since $\mathbf X$ and $\Phi(\mathbf X)$ are supposed to be dimensionless,
the coefficients have to be dimensionless as well.
Now we have two parameters $A$ and $\xi$,
but three coefficients, so one of them must be negligible.

It is reasonable to assume that
the magnitude of the fluctuation $A$ is small,
so the $A^4$ term is much smaller than the $A^2$ term.
Under this assumption,
we set the first two coefficients to unity, and
\begin{align}
|r| \, A^2 \, \xi^d &= 1, \\
c \, A^2 \, \xi^{d-2} &= 1, \\
u \, A^4 \, \xi^d &< 1
\label{eq:drop_uphi4}
\end{align}
Solving the first two equations for $A$ and $\xi$, we get
\begin{align}
\xi &= \sqrt{\frac{c}{|r|}},
\label{eq:xi_scaling}
\\
A^2 &= \frac{1}{c \, \xi^{d-2}} = c^{-d/2} \, |r|^{d/2-1},
\label{eq:A2_scaling}
\end{align}
Using $A^2$ from Eq. \eqref{eq:A2_scaling} for $\langle \delta^2 \phi(\mathbf x) \rangle$
and Eq. \eqref{eq:phi_mf} for $\langle \phi \rangle^2$, we get
\begin{align}
4 \, u \, c^{-\frac{d}{2}} \, |r|^{\frac{d-4}{2}} < 1.
\label{eq:upper_condition}
\end{align}
Since around the critical temperature, $r \rightarrow 0$,
Eq. \eqref{eq:phicmp} is always satisfied if $d > 4$,
and fluctuation is unimportant.

But for $d < 4$, this \eqref{eq:upper_condition}
won't be satisfied at the critical temperature, with $r \rightarrow 0$.



\subsection{Dropping the $\phi^4$ term}


In the above, we have dropped the $u \, \delta\phi^4$ term is valid.
To see if this is valid,
we have to see condition \eqref{eq:drop_uphi4} holds.
Using \eqref{eq:A2_scaling} and \eqref{eq:phi_mf}, we get
$$
u \, c^{-d} \, |r|^{d - 2}
\left( \frac{c}{|r|} \right)^{d/2}
=
u c^{-d/2} |r|^{d/2 - 2}
< 1.
$$
which is the same as \eqref{eq:upper_condition}.
%
This means that the condition given by \eqref{eq:upper_condition}
is the same as the ability of dropping the $u \, \phi^4$ term.
%
For $d < 4$, \eqref{eq:upper_condition} breaks around the critical temperature,
and we cannot drop this term.


\subsection{Self-consistent approximation}

In an attempt of fixing the problem,
we shall not ignore the $\phi^4$ term,
but replace it by $6 \, \langle \phi^2 \rangle^2$.


\subsection{Lower critical dimension}


Let us turn to the lower critical dimension.
From \eqref{eq:A2_scaling},
if $d < 2$ we see that $A^2$ diverges as $r\rightarrow 0$.
This

\end{document}

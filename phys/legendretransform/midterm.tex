\documentclass{article}

\usepackage{amsmath}
\usepackage[usenames,dvipsnames,svgnames,table]{xcolor}
\usepackage{hyperref}

\hypersetup{
    colorlinks,
    linkcolor={red!30!black},
    citecolor={blue!50!black},
    urlcolor={blue!80!black}
}



\begin{document}
\tableofcontents
\title{Mid-term notes}
\author{\vspace{-10ex}}
\date{\vspace{-10ex}}
%\maketitle

\section{Problem 3-22. Fluctuation in grand-canonical ensemble}

\paragraph{Problem 3-22.}
Show that in the grand-canonical ensemble,
$$
\sigma_E^2 = k \, T^2 \, C_V
  + \left( \frac{\partial \overline E } { \partial \overline N} \right)_{T, V} \sigma_N^2.
$$

There is clearly a typo here, since the units do not match.
%
The most likely correction is
\begin{equation}
  \sigma_E^2 = k \, T^2 \, C_V
  + \left( \frac{\partial \overline E } { \partial \overline N} \right)_{T, V} \sigma_N^2 \, \textcolor{red}{\mu}.
  \label{eq:prob3-22a}
\end{equation}
%
Another possible correction is
%
\begin{equation}
  \sigma_E^2 = k \, T^2 \, C_{V,\textcolor{blue}{N}}
  + \left( \frac{\partial \overline E } { \partial \overline N} \right)_{T, V}^{\textcolor{blue}{2}} \sigma_N^2.
  \label{eq:prob3-22b}
\end{equation}
%
Note that in the second correction,
$C_V$ is interpreted as the heat capacity in the canonical ensemble,
so it is less likely.

\paragraph{Solution.}

We will show Eq. \eqref{eq:prob3-22a} first.
The grand partition function is defined as
\begin{equation}
\Xi
=
  \sum_{N} Q(N, V, T) \, z^N
=
  \sum_{N, E} \Omega(N, V, E) \, e^{-\beta \, E + \mu^* \, N}
,
  \label{eq:Xi_def}
\end{equation}
%
where we have defined the reduced chemical potential $\mu^* = \beta \, \mu = \ln z$.
%
As $\ln \Omega$ corresponds to the entropy $S/k$,
$\ln Q$ corresponds to the Helmholtz free energy $S/k - \beta \, E = -\beta \, A$,
the free energy corresponding to $\ln \Xi$
is the so-called grand potential:
%
\begin{equation}
  \ln \Xi
  = \frac{S}{k} - \beta \, E + \mu^* N
  = \beta \, p \, V
  .
  \label{eq:lnXi}
\end{equation}
Here, the last step follows from Euler's theorem,
i.e. $E = T \, S - p \, V + \mu \, N$.\footnote{
  From McQuarrie Eq. (1-60),
  $$
  G = \sum_i \mu_i \, N_i
  $$
  But by definition $G = E - TS + pV$, so
  $$
  E = T \, S - p \, V + \sum_i \mu_i \, N_i.
  $$
  We are using the one component case.
}
%
By differentiating Eq. \eqref{eq:Xi_def}, we get
%
\begin{align}
  \overline E &= -\frac{ \partial \ln \Xi } { \partial \beta }
  ,
  \label{eq:dXidbeta}
  \\
  \overline N &= \frac{ \partial \ln \Xi } { \partial \mu^* }
  ,
  \label{eq:dXidmu}
\end{align}
%
or
\begin{equation}
  d \ln \Xi
  =
  \frac{ \partial \ln \Xi } { \partial \beta } d\beta
  +
  \frac{ \partial \ln \Xi } { \partial \mu^* } d\mu^*
  = -\overline E \, d\beta + \overline N \, d\mu^*
  .
  \notag
  %\label{eq:firstlawXi}
\end{equation}
%
We can compare this to the first law of thermodynamics,
$dE = T \, dS - p \, dV + \mu \, dN$,
which after some Legrendre transforms, can be written as
%
\begin{equation}
  d(\beta \, p \, V) = -E \, d\beta + N \, d\mu^* + \beta \, p \, dV
  .
  \notag
  %\label{eq:firstlaw}
\end{equation}
This means that we should interpret $E$ and $N$ in the first law
as the ensemble averages, $\overline E$ and $\overline N$, respectively.
%
We can further differentiate Eqs. \eqref{eq:dXidbeta} and \eqref{eq:dXidmu}
%
\begin{align}
  \sigma_E^2
  &=
  -\left( \frac{ \partial \overline E } { \partial \beta } \right)_{\mu^*, V}
  = \frac{ \partial^2 \ln \Xi } { \partial \beta^2 }
  ,
  \label{eq:varE}
  \\
  \sigma_N^2
  &= \left( \frac{ \partial \overline N } { \partial \mu^* } \right)_{\beta, V}
  = \frac{ \partial^2 \ln \Xi } { \partial \mu^{*2} }
  \label{eq:varN}
  ,
  \\
  \left( \frac{ \partial \overline E } { \partial \mu^* } \right)_{\beta, V}
  &= -\frac{ \partial^2 \ln \Xi } { \partial \mu^* \, \partial \beta }
  = - \left( \frac{ \partial \overline N } { \partial \beta } \right)_{\mu^*, V}
  \label{eq:covEN}
  .
\end{align}

Now the heat capacity in the grand-canonical ensemble can be defined as
%
\begin{equation}
= \left( \frac{ T \, \partial S } { \partial T } \right)_{\mu^*, V}
= -\beta \left( \frac{ \partial S } { \partial \beta } \right)_{\mu^*, V}
= -\frac{1}{T} \left( \frac{ \partial (S/k) } { \partial \beta } \right)_{\mu^*, V}.
\label{eq:CV_def}
\end{equation}
%
Using Eq. \eqref{eq:lnXi}, we have
\begin{align}
-T \, C_V
  &= \left( \frac{ \partial (\ln \Xi + \beta \, \overline E - \mu^* \, \overline N) }
  { \partial \beta } \right)_{\mu^*, V}.
  \notag
  \\
  &= \frac{ \partial \ln \Xi } { \partial \beta }
  + \overline E
  + \beta \left( \frac{ \partial \overline E } { \partial \beta } \right)_{\mu^*, V}
  - \mu^* \left( \frac{ \partial \overline N } { \partial \beta } \right)_{\mu^*, V}
  \notag
  \\
  &=
  - \beta \, \sigma_E^2
  + \mu^* \left( \frac{ \partial \overline E } { \partial \mu^* } \right)_{\beta, V}
  ,
  \notag
\end{align}
%
where we have used Eqs. \eqref{eq:dXidbeta}, \eqref{eq:varE}, and \eqref{eq:covEN}
in the last step.
%
Further using Eq. \eqref{eq:varN}, we have
\begin{equation}
T \, C_V
=
\beta \, \sigma_E^2
+
\mu^* \,
\frac{ \left( \partial \overline E / \partial \mu^* \right)_{\beta, V} }
     { \left( \partial \overline N / \partial \mu^* \right)_{\beta, V} }
\, \sigma_N^2
.
\label{eq:TCV}
\end{equation}
Finally, if we interpret
$$
\frac{ \left( \partial \overline E / \partial \mu^* \right)_{\beta, V} }
     { \left( \partial \overline N / \partial \mu^* \right)_{\beta, V} }
\to
\left(
  \frac{ \partial \overline E }
       { \partial \overline N }
\right)_{T, V}
,
$$
and multiply both sides of Eq. \eqref{eq:TCV} by $k T = 1/\beta$,
we recover Eq. \eqref{eq:prob3-22a}.

\paragraph{Second interpretation.}


About the second interpretation, Eq. \eqref{eq:prob3-22b},
we will use the advanced technique of transforming thermodynamic variables
(i.e. chain rule),
so it is okay to ignore it if it sounds unfamiliar.
%
Similar to Eq. \eqref{eq:CV_def},
the heat capacity in the canonical ensemble is defined as
\begin{equation}
C_{V, N}
= -\frac{1}{T} \left( \frac{ \partial S^* } { \partial \beta } \right)_{V, N},
  \label{eq:CVN_def}
\end{equation}
where we have introduced $S^* = S/k$ for convenience.
%
To be clear, the heat capacity in the grand-canonical ensemble
will be denoted as
\begin{equation}
C_{V, \mu^*} = -\frac{1}{T} \left( \frac{ \partial S^* } { \partial \beta } \right)_{V, \mu^*}.
  \label{eq:CVmu_def}
\end{equation}
%
Using the technique of determinants for transforming variables, we have
\begin{align*}
  \left( \frac{ \partial S^* } {\partial \beta } \right)_{\mu^*}
  &=
  \frac{ \partial (S^*, \mu^*) } { \partial (\beta, \mu^*) }
  =
  \left.
  \frac{ \partial (S^*, \mu^*) } { \partial (\beta, N) }
  \middle/
  \frac{ \partial (\beta, \mu^*) } { \partial (\beta, N) }
  \right.
  \\
  &=
  \left( \frac{ \partial S^* } {\partial \beta } \right)_{N}
  -
  \left.
  \left( \frac{ \partial \mu^* } {\partial \beta } \right)_{N}
  \left( \frac{ \partial S^* } {\partial N } \right)_{\beta}
  \middle/
  \left( \frac{ \partial \mu^* } {\partial N } \right)_{\beta}
  \right.
\end{align*}
From the first law, we have
$$
d(S^* - \beta \, E) = -E \, d\beta -\mu^* dN + \beta p \, dV
,
$$
%
so
\begin{align*}
  \left( \frac{ \partial \mu^* } {\partial \beta } \right)_{N}
  &=
  \left( \frac{ \partial E } {\partial N } \right)_{\beta}
  ,
  \\
  \left( \frac{ \partial S^* } {\partial N } \right)_{\beta}
  &=
  \left( \frac{ \partial (S^* - \beta \, E) } {\partial N } \right)_{\beta}
  +
  \left( \frac{ \partial ( \beta \,  E) } {\partial N } \right)_{\beta}
  =
  -\mu^*
  +
  \beta \, \left( \frac{ \partial E } {\partial N } \right)_{\beta}
  .
\end{align*}
Using the above two expressions, we get
\begin{align*}
  \left( \frac{ \partial S^* } {\partial \beta } \right)_{\mu^*}
  &=
  \left( \frac{ \partial S^* } {\partial \beta } \right)_{N}
  +
  \mu^* \,
  \left( \frac{ \partial E } {\partial N } \right)_{\beta}
  \left( \frac{ \partial N } {\partial \mu^* } \right)_{\beta}
  -
  \beta \,
  \left( \frac{ \partial E } {\partial N } \right)_{\beta}^2
  \left( \frac{ \partial N } {\partial \mu^* } \right)_{\beta}
  .
\end{align*}
%
Dividing both sides by $-1/(k T) = -\beta$,
and using Eqs. \eqref{eq:CVN_def}, \eqref{eq:CVmu_def}, and \eqref{eq:varN},
we get
$$
  k T^2 C_{V, \mu^*}
  +
  \mu \,
  \left( \frac{ \partial E } {\partial N } \right)_{\beta}
  \sigma_N^2
  =
  k T^2 C_{V, N}
  +
  \left( \frac{ \partial E } {\partial N } \right)_{\beta}^2
  \sigma_N^2
  .
$$
If we interpret
$\left( \partial E / \partial N \right)_{\beta}$
as
$\left( \partial \overline E / \partial \overline N \right)_{T, V}$,
we recover Eq. \eqref{eq:prob3-22b}.
Note that we have implicitly assumed that $V$ is fixed in the above derivation.


\section{Tsallis entropy: application of Lagrange multiplier}

\paragraph{Problem.}
[This problem extends Problems 1-51 and 2-6 in McQuarrie].

Unlike the usual definition of entropy,
$S = -k \sum_i P_i \, \ln P_i$,
in this problem we wish to explore the idea of Tsallis entropy,
which is defined as\footnote{
  The Tsallis entropy is reduced to the usual entropy in the limit of $q \to 1$.
  To see this, we note that
  \begin{align*}
  P_i^q
    &= P_i \, P_i^{q-1} = P_i \, e^{(q - 1) \, \ln P_i}
    \approx P_i \, \left[ 1 + (q - 1) \, \ln P_i \right]
    .
  \end{align*}
  %
  The last step holds because when $x = (q-1) \, \ln P_i$ is small,
  $$
  e^x = 1 + x + \frac{x^2}{2!} + \frac{x^3}{3!} + \cdots \approx 1 + x.
  $$
  %
  Then summing over $i$ yields
  \begin{equation}
  \sum_i P_i^q = \sum_i P_i  + (q - 1) \sum_i P_i \, \ln P_i
  = 1 + (q - 1) \sum_i P_i \, \ln P_i
    ,
    \notag %\label{eq:pqsum}
  \end{equation}
  where we have used Eq. \eqref{eq:psum} in the second step.
  Using this expression in Eq. \eqref{eq:Sq_def} yields
  $$
  \lim_{q \to 1} S_q =
  \frac{k}{q - 1} \left( 1 - 1 - (q-1) \sum_i P_i \, \ln P_i \right)
  = -k \sum_i P_i \, \ln P_i.
  $$
}
\begin{equation}
S_q = \frac{k}{q - 1} \left( 1 - \sum_i P_i^q \right).
  \label{eq:Sq_def}
\end{equation}

If we impose two constraints (as in Problem 2-6)
%
\begin{align}
  \sum_i P_i = 1, \label{eq:psum} \\
  \sum_i E_i \, P_i = E, \label{eq:pEsum}
\end{align}
%
Show that the equilibrium distribution is
$$
P_i \propto (E_0 - \, E_i)^{1/(q-1)},
$$
for some value of $E_0$.\footnote{
  This is distribution is not complete artificial.
  %
  The potential energy distribution in the microcanonical ensemble is
  $$
  P_i(U) \propto (E_\mathrm{tot} - U)^{\frac{N_f}{2} - 1},
  $$
  where
  $E_\mathrm{tot}$ is the total energy, and
  $N_f$ is the number of degrees of freedom (usually $3N$ in three dimensions).
  %
  Thus, the distribution of the potential energy, $U$,
  maximizes the Tsallis entropy for $q = N_f/(N_f - 2)$.
}


\paragraph{Solution.}

To find the equilibrium distribution
we need to find the numbers $P_1, P_2, \dots$
that maximize the Tsallis entropy, $S_q$.
%
Using the method of Lagrange multiplier, we define the target function
as
$$
F(P_1, P_2, \dots)
= S_q(P_1, P_2, \dots) - \lambda_1 \sum_i P_i - \lambda_2 \sum_i E_i \, P_i,
$$
with $\lambda_1$ and $\lambda_2$ being two multipliers
for the two constraints, Eqs. \eqref{eq:psum} and \eqref{eq:pEsum}, respectively.
%
To maximize $F$, we require
$$
\frac{ \partial F } { \partial P_i } = 0,
$$
for each $i$, which leads to
$$
\frac{ \partial S_q } { \partial P_i } - \lambda_1 - \lambda_2 \, E_i = 0.
$$
This condition leads to
$$
P_i = \left[ \frac{q - 1} { k \, q} (-\lambda_1 - \lambda_2 \, E_i) \right]^{1/(q-1)}.
$$
Finally, we can redefine $-\lambda_1/\lambda_0$ as $E_0$.




\section{A note on Legendre transform}

In pages 14-17, McQuarrie has given a lucid explanation on Legendre transform.\footnote{
  Another explanation can be found in Arnold's Mathematical Methods of Classical Mechanics, pages 61-64.
}
%
Here is a piece of personal understanding.
%
Legendre transform is more readily understood
through the calculus technique of integration by parts.
%
If we have a hard integral,
$$A = \int p(x) \, dx,$$
but we totally know how to compute
$$G = \int x(p) \, dp,$$
where $x(p)$ is the inverse function of $p(x)$,
then we can use integration by parts:
%
$$
A = \int p(x) \, dx = x p - \int x(p) \, dp = x p - G.
$$
That is, we can compute the hard integral, $A$, through
the easier one, $G$.

For example, for $p(x) = \sin^{-1}(x)$, the integral
$$
A = \int \sin^{-1}(x) \, dx
$$
is intimidating.
%
But since $x = \sin(p(x))$,
the integral
$$
G = \int x \, dp = \int \sin(p) \, dp = -\cos p
$$
is rather simple. So we calculate the original integral as
$$
A = x \, p - G = x p + \cos p = x \, \sin^{-1}(x) + \sqrt{1-x^2}.
$$

Here is another analogy in physical experiment.
%
Suppose we want to measure the Helmholtz free energy of a balloon.
%
We know that
%
\begin{equation}
  dA = -S\,dT - p \, dV + \mu dN,
  \label{eq:firstlaw1}
\end{equation}
%
Assuming the temperature, $T$,
and the number of particles in the balloon, $N$, are fixed,
all we need to do is to design an experiment to calculate the integral
$$
-\int p \, dV,
$$
which requires us to change the volume of the balloon.
%
But this is a challenging task, and
even if this can be done, we have to measure the pressure of balloon,
which is also difficult.
%
A easier task is to place the balloon in an environment
under a known pressure and measure the volume of the balloon.
%
For example, we can push the balloon under water,
where the pressure is known to be $p_0 + \rho_w \, g \, h$.
Here, $p_0$ is the atmospheric pressure, $\rho_w$ is the mass density of water,
$g$ is the gravity of Earth, and $h$ is the vertical distance from the water surface.
%
The volume of the balloon can be readily measured from the amount of displaced water.
%
In other words, we can compute the integral
\begin{equation}
\int V \, dp.
\label{eq:Vdp}
\end{equation}
%
If we apply Legendre transform or
integration by parts on Eq. \eqref{eq:firstlaw1},
%
\begin{equation}
  dG = d(A + pV) = -S\,dT + V \, dp + \mu dN
  ,
  \notag
  %\label{eq:firstlaw2}
\end{equation}
%
then we find that the integral \eqref{eq:Vdp} corresponds to
the change of the Gibbs free energy, $G$.
%
Once we have determined $\Delta G$
from a series of balloon-dipping experiment,
the change of the Helmholtz free energy, $\Delta A$,
can be readily found from $\Delta G - \Delta( p \, V)$.





\end{document}


\documentclass{article}
\usepackage{amsmath}
\usepackage{bm}
\usepackage[usenames,dvipsnames]{xcolor}
\usepackage{tikz}
\usepackage{hyperref}
\usepackage{url}
\hypersetup{
    colorlinks,
    linkcolor={red!30!black},
    citecolor={blue!50!black},
    urlcolor={blue!80!black}
}



\begin{document}

\title{Notes for Quantum mechanics in your face}
\author{ Sidney Coleman }
\date{}
\maketitle

I found an interesting set of lectures on physics in this
\url{https://www.physics.harvard.edu/events/videos}
{video archive}.

Particularly I enjoyed the
\url{http://media.physics.harvard.edu/video/?id=SidneyColeman\_QMIYF}
{lecture by Sidney Coleman}
on the nature of quantum mechanics.
Below are my notes.

\section{Validity of the quantum mechanics}

This section gives a clean demonstration of Bell's theorem without inequalities (from Mermin 1990) for the spooky interactions faster than the speed of light.  This is a thought experiment that shows definitely different outcomes from classical mechanics and quantum mechanics.

We first prepare a state
$|\psi \rangle = \frac{1}{\sqrt{2}} \left(|+++\rangle - |---\rangle\right)$
in our lab.
We then send the first, second and third spins to,
respectively, stations A, B and C for measurements.
%
The stations are far apart so they cannot interfere one another.
%
To be precise, Coleman adds that the differences between the events
of the measurements at A, B and C are space-like intervals,
and by the Lorentz invariant and causality,
it is impossible to establish causal relations
among the events at A, B and C.

Each station (A or B or C) can \emph{independently} decide
whether to measure the $x$ or $y$ component of spin.
%
Then the measurement of $M_1 = \sigma^A_x \sigma^B_y \sigma^C_y$
(if A decides to measure $x$, B and C $y$)
yields $+$ for our $|\psi \rangle$.
This is because $\sigma_x|\pm\rangle = |\mp\rangle$,
and $i\sigma_y|\pm\rangle = \mp|\mp\rangle$, and
$$
\begin{aligned}
\sigma^A_x \sigma^B_y \sigma^C_y
 | + + + \rangle
&= 1\cdot (i) \cdot (i) | - - -\rangle = - | - - -\rangle \\
\sigma^A_x \sigma^B_y \sigma^C_y | - - -\rangle
&= 1\cdot (-i) \cdot (-i) | + + +\rangle = - | + + +\rangle \\
\sigma^A_x \sigma^B_y \sigma^C_y (| + + + \rangle - | - - - \rangle)
&= -| - - - \rangle + | + + + \rangle \\
&= | + + + \rangle - | - - - \rangle.
\end{aligned}
$$
This shows $\sigma^A_x \sigma^B_y \sigma^C_y |\psi\rangle = |\psi \rangle$,
and $|\psi\rangle$ is an eigenstate of the measurement.
In other words,
if we keep sending the stations the same state $|\psi\rangle$,
and A keeps measuring $x$, B and C $y$,
the product of the results from the three stations is always $+1$.


The same result holds $M_2 = \sigma^A_y \sigma^B_x \sigma^C_y$,
and $M_3 = \sigma^A_y \sigma^B_y \sigma^C_x$.
Therefore, we conclude that
if we keep sending the stations the same state $|\psi\rangle$,
and if one and only one of the three stations measures $x$,
the result must always be $+1$.
%
We shall also add that $M_1$, $M_2$ and $M_3$ mutually commute,
so they can be determined simultaneously.


Similarly,
if the three stations all measure $x$,
the result must always be $-1$.
That is, for $M_4 = \sigma^A_x \sigma^B_x \sigma^C_x$
we have $\sigma^A_x \sigma^B_x \sigma^C_x|\psi\rangle = -|\psi\rangle$.
This is the prediction from quantum mechanics.


From the view point of classical mechanics,
we only assume that the lab sends \emph{some unknown information}
to the three stations.
But each lab decides \emph{deterministically} from this information
what the result should be, given its independent choice of measuring $x$ or $y$.
%
This appears to be a generous condition.
However, it turns out that,
through the patterns shown from the results,
the information sent by the lab can be partially deduced,
yielding nontrivial predictions (just like solving a logic puzzle).
In our case, the observed pattern is that
if one and only one of the three stations measures $x$,
the product of the results from the three stations is $+1$.
Our deduction from this observation is
that when all three stations measure $x$
the result is also $+1$
(see the four-page paper, Mermin 1990, also below for a brief version).

But now we see that there is a difference between
the predictions from classical mechanics and quantum mechanics,
for the measurement of $\sigma^A_x \sigma^B_x \sigma^C_x$.
Thus this experiment will give a decisive answer
whether quantum mechanics or classical mechanics is correct.
Of course, quantum mechanics wins.


Below we furnish some details that are present in the paper
to better understand the setup.


\subsection{Classical measurements}

Classically, our lab can only send each station information
through a lookup table instead of a spin in each time.
%
The station only decides what to ``measure'',
but produces the result by looking up the corresponding
value from the table received from the lab.
%
For example, if the lab sends $(+, -)$ to A, it instructs A that
if you measure $x$, yield the result of $+$, if you measure $y$, yield $-$.
The lab also needs to do the same thing to B and C.
So each time, the lab must prepare a $3 \times 2$ lookup table. For example
$$
\begin{array}{ccc}
A: \\
B: \\
C:
\end{array}
\left(\begin{array}{ccc}
+ & - \\
- & + \\
- & +
\end{array}\right)
$$
Then the measurement of
$\sigma^A_x \sigma^B_x \sigma^C_x$
will be $(+) (-) (-) = +$;
the measurement of $\sigma^A_x \sigma^B_y \sigma^C_x$
will be $(+) (+) (-) = -$.


The lookup table is only variable.
In our case, it has six cells, and hence $2^6 = 64$ possibilities.
Now let us try to eliminate a few impossible cases.

Since $M_1 = \sigma^A_x \sigma^B_y \sigma^C_y$ always yields $+$,
we are certain that the product of the cells at $(1, 1)$, $(2, 2)$, and $(3, 2)$ is $+$.
This shows that our matrix only take one of the follow forms
$$
\begin{array}{ccc}
A: \\
B: \\
C:
\end{array}
\left(\begin{array}{ccc}
+ &   \\
  & + \\
  & +
\end{array}\right)
\left(\begin{array}{ccc}
+ &   \\
  & - \\
  & -
\end{array}\right)
\left(\begin{array}{ccc}
- &   \\
  & + \\
  & -
\end{array}\right)
\left(\begin{array}{ccc}
- &   \\
  & - \\
  & +
\end{array}\right)
$$
Continuing the same analyses for $M_2$ and $M_3$ yields eight possibilities
$$
\begin{aligned}
\begin{array}{ccc}
A: \\
B: \\
C:
\end{array}
\left(\begin{array}{ccc}
+ & + \\
+ & + \\
+ & +
\end{array}\right)
\left(\begin{array}{ccc}
+ & + \\
- & - \\
- & -
\end{array}\right)
\left(\begin{array}{ccc}
- & + \\
- & + \\
+ & -
\end{array}\right)
\left(\begin{array}{ccc}
- & + \\
+ & - \\
- & +
\end{array}\right)
\\
\begin{array}{ccc}
A: \\
B: \\
C:
\end{array}
\left(\begin{array}{ccc}
+ & - \\
- & + \\
- & +
\end{array}\right)
\left(\begin{array}{ccc}
+ & - \\
+ & - \\
+ & -
\end{array}\right)
\left(\begin{array}{ccc}
- & - \\
+ & + \\
- & -
\end{array}\right)
\left(\begin{array}{ccc}
- & - \\
- & - \\
+ & +
\end{array}\right)
\end{aligned}
$$
There is something in common of the eight choices,
when all three stations measure $x$,
the result is always $+1$ (just multiple the cells in the first column).




\subsection{References}

\begin{enumerate}
\item N. D. Mermin, Quantum mysteries revisited, Am. J. Phys. 58 8 731-734 (1990).  Four-page article that gives the simplest version of the EPR experiment.
\item N. D. Mermin, Is the moon there when nobody looks? Reality and the quantum theory, Physics Today, April 38-47 (1985)  earlier version.
\item Daniel M. Greenberger, Michael A. Horne, Abner Shimony, and Anton Zeilinger, Bell's theorem without inequalities, Am. J. Phys. 58 (12) 1131-1143, (1990).
\end{enumerate}

\section{On the nature of measurements in quantum mechanics}

This section addresses the nature of quantum mechanical measurement.
According to the Copenhagen interpretation,
a measurement involves the collapse of the wave function,
and hence is nonlinear and time irreversible.
This is in strong contradiction of the nature of Schr\"odinger's equation,
which is linear and reversible.

\subsection{(N. Mart, 1930) Analogy of cloud chamber}

If we release an ionized particle in the spherical-symmetric s-wave state in a cloud chamber, the ionized particle will make a straight line.  This is puzzling.  How come a spherical symmetric wave function becomes a straight line, which is obviously not spherical symmetrical?

Let $|C\rangle$ is the state of the system, $L$ be the projection operator for a linear track, such that $L | C \rangle = | C \rangle$ if the state $|C\rangle$ corresponds to a straight line, or $L|C \rangle = 0$.  If the particle starts with a state $|\Phi_{\vec k}\rangle$ with momentum $\vec k$, and the chamber with state $| C_0\rangle$, then the initial state is $| \Phi_{\vec k}, C_0 \rangle$ will develop to a straight line along $\vec k$, $| \Psi_{\vec k} \rangle$. If we apply the projection operator $L$ on it, it remains the same,
$$
L| \Psi_{\vec k} \rangle = | \Psi_{\vec k} \rangle
$$


Now the clever part is if we integrate the initial state over the solid angle, the resulting state is a linear combination of straight lines
$$
\int d\Omega_{\vec k} | \Phi_{\vec k}, C_0 \rangle
\rightarrow
|\Psi_C\rangle \equiv \int d\Omega_{\vec k} | \Psi_{\vec k} \rangle.
$$
Now by the linearity of the projection operator $L$, we have $L |\Psi_C\rangle = |\Psi_C\rangle$, for each component of $|\Psi_C\rangle$ remains unchanged under the operator $L$, and so is the linear combination. This means $|\Psi_C\rangle$ behaves like a straight line.  But the left hand side is a spherically symmetrical wave function like the s-wave!

Mart said the problem is that people think [the solution of] the Schr\"odinger equation as a wave in the three dimensional space rather than a wave instead of a wave in the multiple dimensional space.  Coleman put it more profoundly, the problem is that people think the particle as a quantum mechanical system, but the cloud chamber as a classical-mechanical system.  If you think of the particle and cloud chamber as an entangled system, there is no problem.

\subsection{Why do we think we have definite outcomes after measurements?}

The above analogy can be generalized to Everett's many-worlds interpretation (MWI).
When a measurement happens, we think we have entered a state
that is associated with a certain outcome (like a straight line in the above analogy),
but in reality there are other parallel worlds
that are associated with different outcomes.
If we limit our mind into a definite outcome,
we have to admit that the wave function has collapsed
and the measurement represents an irreversible and nonlinear operation.
But if we include other possible worlds into the picture,
this wave function of many worlds follows
a unitary time evolution given by Schr\"odinger's equation.

The problem here is we refuse to think the observer as a quantum mechanical system.

\subsection{References}
\begin{enumerate}
\item H. Everett III, ``Relative state'' formulation of quantum mechanics, Rev. Mod. Phys. 29 434 (1957).
\item J. Hartle, Quantum Mechanics of Individual Systems, J. Am. Phys. 36 204 (1968).
\end{enumerate}

\section{Problem of probability}


The part shows that quantum mechanics challenges our definition of randomness.

\subsection{Is a sequence random?}

Given a sequence, can we tell if it is random?
If a sequence is infinite,
we can compute the average and the correlation functions,
from there we can perhaps judge if the sequence is random.

Consider a sequence of $+1$ and $-1$.
We can compute the average and the correlation functions
and if all of them are zeros,
we can perhaps say if the sequence is random.
Particularly, for the average
$$
\overline{ \sigma } = \lim_{n \rightarrow \infty} \frac{1}{N} \sum_{i = 1}^N \sigma_i.
$$
and for the correlation function,
$$
\overline{ \sigma_i \sigma_{i+a} } = \lim_{n \rightarrow \infty} \frac{1}{N} \sum_{i = 1}^N \sigma_i \sigma_{i+a}.
$$

But this classical definition fails for quantum mechanics.
Consider the sidewise state
$|\psi\rangle = \frac{1}{\sqrt 2}(|+\rangle + |-\rangle)$,
and for a sequence of replication of such states,
$|\Psi\rangle = |\psi^1\rangle \cdots |\psi^N\rangle$,
let us define an average $\overline{ \sigma_z } = \frac{1}{N} \sum_{i = 1}^N \sigma^i_z$.
So
$$
\overline{ \sigma_z } |\Psi\rangle
= \frac{1}{N} \sum_{r=1}^N \sigma^r_z |\Psi\rangle.
$$

Let us compute $\|\overline{ \sigma_z } |\Psi\rangle \|^2$.
For different spins $\langle \Psi|\sigma^r_z \sigma^s_z | \Psi \rangle = 0$,
for the same spins $\langle \Psi|\sigma^r_z \sigma^r_z | \Psi \rangle = 1$.
$$
\|\overline{ \sigma_z } |\Psi\rangle \|^2
=
\frac{1}{N^2} \sum_{r = 1}^N \langle \Psi|\sigma^r_z \sigma^r_z | \Psi \rangle = \frac{1}{N}
 \rightarrow 0. \qquad (N\rightarrow \infty)
$$
A similar argument can be done for correlation functions.
So Coleman claims that by classical measures this sequence is random.
But quantum mechanically it reads $|x \rangle \cdots |x\rangle$,
which is obviously not random! Aha!


\section{Miscellaneous quotes}

Paraphrasing.

\begin{enumerate}
\item
Every successful physical theory swallows its predecessor alive.
For example, in statistical mechanics swallows thermodynamics, the concept of entropy or heat is explained in terms of molecular motion. It is not the other way around.

\item
As Picasso said it doesn't have to be a masterpiece for you get the idea.

\item
People say that the reduction of the wave packet occurs because it \emph{looks like}
that the reduction of the wave packet occurs.
\end{enumerate}




\end{document}


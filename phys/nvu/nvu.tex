\documentclass{article}
\usepackage{amsmath}
\usepackage{bm}
\usepackage[usenames,dvipsnames]{xcolor}
\usepackage{tikz}
\usepackage{hyperref}

\hypersetup{
    colorlinks,
    linkcolor={red!30!black},
    citecolor={blue!50!black},
    urlcolor={blue!80!black}
}



\begin{document}

\title{NVU dynamics}
\author{ \vspace{-10ex} }
\date{ \vspace{-10ex} }
\maketitle


The commonly-used Verlet molecular dynamics (MD) integrator is given by
\begin{equation}
\mathbf r_{n+1} =
2 \mathbf r_n - \mathbf r_{n-1} + \frac{\Delta t^2}{m} \, \mathbf f_n,
\end{equation}
where $n$ is the index of time step.
This dynamics approximately conserves the total energy.
It is also symplectic (area preserving),
so there is little long time drift in the total energy.

Interestingly, it is recently proposed a similar dynamics,
call $NVU$ dynamics, which approximately conserves
the potential energy.
\begin{equation}
\mathbf r_{n+1}
= 2 \, \mathbf r_n - r_{n-1}
- \frac{ 2 \, (\mathbf r_{n} - \mathbf r_{n - 1}) \cdot \mathbf f_n }
       { \mathbf f_n \cdot \mathbf f_n }
  \mathbf f_n.
\label{eq:nvu}
\end{equation}

The $NVU$ dynamics has several nice properties.
\begin{enumerate}
\item
The step size is constant.
$$
|\mathbf r_{n+1} - \mathbf r_n| = |\mathbf r_n - \mathbf r_{n-1}|.
$$


\item
\begin{equation}
(\mathbf r_{n+1} - \mathbf r_n) \cdot \mathbf f_n
=
(\mathbf r_{n-1} - \mathbf r_n) \cdot \mathbf f_n.
\label{eq:rfsame}
\end{equation}


\item
Reversibility. Using \eqref{eq:rfsame}, we have
$$
\mathbf r_{n-1} = 2 \, \mathbf r_n - \mathbf r_{n+1}
- 2 \frac{ ( \mathbf r_n - \mathbf r_{n+1} ) \cdot \mathbf f_n }
{ \mathbf f_n \cdot \mathbf f_n } \, \mathbf f_n,
$$
which shows that the time-reversed version is exactly the same.


\item
From \eqref{eq:rfsame}, we can show that
the displacement over the two steps is perpendicular to the force:
\begin{equation}
(\mathbf r_{n+1} - \mathbf r_{n-1}) \cdot \mathbf f_n = 0.
\end{equation}
Geometrically,
we can show that the vectors
$\mathbf r_{n+1} - \mathbf r_{n-1}$,
$-\frac{ 2 (\mathbf r_n - \mathbf r_{n-1}) \cdot \mathbf f_n } { \mathbf f_n \cdot \mathbf f_n } \mathbf f_n$,
and
$2 ( \mathbf r_n - \mathbf r_{n-1})$
form a right triangle.


\item
The dynamics is not symplectic, and the potential energy is only approximately conserved.

\end{enumerate}


\section{References}

T. S. Ingebrigtsen et al., J. Chem. Phys. 135 104101 (2011).

\end{document}

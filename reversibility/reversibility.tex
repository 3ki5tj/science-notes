\documentclass{article}
\usepackage{amsmath}
\usepackage{bm}
\usepackage[usenames,dvipsnames]{xcolor}
\usepackage{tikz}
\usepackage{hyperref}

\hypersetup{
    colorlinks,
    linkcolor={red!30!black},
    citecolor={blue!50!black},
    urlcolor={blue!80!black}
}



\begin{document}

\title{Reversibility of molecular dynamics integrators}
\author{ \vspace{-10ex} }
\date{ \vspace{-10ex} }
\maketitle

\section{Introduction}

For anyone who has done some molecular dynamics (MD) simulations,
the word ``reversibility'' should sound familiar.
%
This is word associated with MD integrators,
i.e., numerical algorithms of integrating Newton's equation,
$\mathbf f = m \ddot{\mathbf x}$
such as Verlet, velocity Verlet, leapfrog, etc.
%
Reversibility appears to be a desirable quantity
of fancy MD integrators,
and the above-mentioned integrators are reversible
(Runge-Kutta is not).
%
At least, these facts were all what I cared to know about reversibility
for a long time.
%
We also know that MD integrators are approximate,
that is, generally we cannot exactly integrate Newton's equation,
and the precision is associated with
the size of the MD time step, $\Delta t$.
%
So, for something that is supposed to be approximate,
why should one ever care about a fancy quality, ``reversibility''?
%



This note is intended to make sense of the word ``reversibility.''
%
I wish to discuss in \emph{very informal} terms,
1) what ``reversibility'' is,
2) why it is important for MD integrators,
3) the accuracy of the reversibility (is it accurate up to $\Delta t$,
$\Delta t^2$, etc.?)
and
4) how to tell if an MD integrator is reversible.
%
The discussion below is rather shallow,
it only summarizes what I know so far.


\section{What is reversibility?}

Newton's equation, written in terms of $dx/dt$ and $dv/dt$,
%
$$
\begin{aligned}
\frac{dx}{dt} &= v, \\
\frac{dv}{dt} &= \frac{f(x)}{m}
\end{aligned}
\label{eq:newton}
$$
%
is \emph{time reversible}. That is, if we change variables with
$v = -u$ and $t = -\tau$, the resulting equation
%
\begin{align}
\frac{dx}{d\tau} &= u, \\
\frac{du}{d\tau} &= \frac{f(x)}{m}
\end{align}
%
is identical to \eqref{eq:newton}.


Generally, anything, like an MD integrator, that is invariant under
$t \rightarrow -t$ and $v \rightarrow -v$
is called time reversible.
%
Note that strictly the second transformation rule
$v \rightarrow -v$
is itself implied by $t \rightarrow -t$,
because under timer reversal
$v = dx/dt \rightarrow dx/d(-t) = -v$.


\section{Why is reversibility important for MD integrators?}


For MD integrators, reversibility gives some insurance of stability.
%
It is a common problem for MD practitioners that
an MD simulation can blow up with a large time step.
%
Now by using a reversible integrator, we can often alleviate the problem a bit.
%
Below is a very non-rigorous hand-waving argument (that would probably drive
any mathematician crazy).
%
Reversibility means a symmetry between the forward and backward time evolution.
%
If we start from an average-looking initial point, $\mathbf z_0 = (x_0, v_0)$,
the result of time evolution of $\mathbf z_0$ after a period of $t$,
denoted by $\mathbf z(t)$, would \emph{roughly} be the same
as the result of backward time evolution starting from
$\bar {\mathbf z}_0 = (x_0, -v_0)$.
%
If $\mathbf z(t)$ blows up, then so does $\bar{ \mathbf z}(-t)$.
%
Now looking at the trajectory of $\mathbf z(t)$,
we found that our starting point lies near a special point,
the minimum, of the trajectory.
%
But we have assumed that $\mathbf z_0$ is not special,
so the trajectory should not blow up.



\section{Reversibility is an accurate measure}



Although MD integrators have accuracy limited by $\Delta t$,
reversibility is an exact measure,
at least up to the accuracy permissible by floating numbers
of the computer.
%
To put the two on scales, an MD step is of the order of magnitude of $0.002$,
where as the relative accuracy of floating-point numbers
is about $10^{-7}$ (single-precision) or $10^{-14}$ (double-precision).

For an MD integrator,
reversibility means that if an initial state $(x_i, v_i)$
undergoes $n$ MD steps reaching a final state $(x_f, v_f)$,
then $n$ MD steps using the same integrators
starting from $(x_f, -v_f)$ will yield $(x_i, -v_i)$.



Below we shall discuss a few MD integrators.
%
We refer to the pair $\mathbf z = (x, v)$ as the state variable,
and denote its time reversal by $\bar{\mathbf z} = (x, -v)$.



\section{Leapfrog integrator}


The leapfrog integrator is one of the simplest and most popular MD integrators.
The MD package GROMACS, for example, currently uses leapfrog
as its default MD integrator.
%
It contains two sub-steps:
\begin{align}
x &\rightarrow x + v \Delta t
\tag{X}
\label{eq:x}
\\
v &\rightarrow v + \frac{f(x)}{m} \Delta t.
\tag{V}
\label{eq:v}
\end{align}

It is important to realize that for the leapfrog integrator,
each sub-step is reversible,
but the composite step is not.
Below is a more elaborated analysis.

\subsection{Reversibility of the $x$ half-step}

To see why step \eqref{eq:x} is reversible,
we observe that this step delivers the state $\mathbf z_1 = (x_1, v_1)$
to $\mathbf z_2 = (x_2, v_2)$, where
\begin{equation}
x_2 = x_1 + v_1 \Delta t.
\label{eq:x1_x2}
\end{equation}
Now if we start with $\bar{\mathbf z}_2 = (x_2, -v_1)$,
the condition of reversibility requires that the same procedure
to reproduce $\bar{\mathbf z}_1 = (x_1, -v_1)$.
%
This is indeed the case, because step \eqref{eq:x}, applied on $(x_2, -v_1)$
gives
\begin{equation}
y = x_2 + (-v_1) \Delta t,
\end{equation}
and after a comparison with step \eqref{eq:x1_x2}, we see that $y = x_1$,
so the result after a backward step is $(x_1, -v_1) = \bar{\mathbf z}_1$.

\subsection{Reversibility of the $v$ half-step}

Similarly, step \eqref{eq:v} is reversible.
This step changes the state from $\mathbf z_2 = (x_2, v_1)$
to $\mathbf z_3 = (x_2, v_2)$, where
\begin{equation}
v_2 = v_1 + \frac{f(x_2)}{m} \Delta t.
\label{eq:v1_v2}
\end{equation}
Now if we start from $\bar{\mathbf z}_3 = (x_2, -v_2)$,
can we recover $\bar{\mathbf z}_2 = (x_2, -v_1)$ in the end
(reversibility)?
%
The answer is yes.
%
By applying step \eqref{eq:v} to $(x_2, -v_2)$, we get
\begin{equation}
u = (-v_2) + \frac{f(x_2)}{m} \Delta t,
\end{equation}
and according to \eqref{eq:v1_v2}, $u = -v_1$.
%
Thus, the result of the backward step yields
$(x_2, -v_1) = \bar{\mathbf z}_2$ as required.

\subsection{Irreversibility of the composite $x$-$v$ step}

Now the bad news is that if we treat the combination
of \eqref{eq:x} and \eqref{eq:v} as a composite map
(i.e., a black box),
the composite map is \emph{not} reversible.
%
Continuing from the above argument,
we see that the composite map transforms ${\mathbf z}_1 = (x_1, v_1)$
to $\mathbf z_3 = (x_2, v_2)$.
If the composite map were reversible,
then starting from $\bar{\mathbf z}_3 = (x_2, -v_2)$,
we should be able to recover $\bar{\mathbf z}_1 = (x_1, -v_1)$.
%
But this is untrue as shown below.
First, applying step \eqref{eq:x} on $\bar{\mathbf z}_3$, we get
a state of $\bar{\mathbf z}_4 = (y, -v_2)$ with
$$
\begin{aligned}
y &= x_2 + (-v_2) \, \Delta t \\
&= x_1 + (v_1 - v_2) \, \Delta t,
\end{aligned}
$$
where we have used \eqref{eq:x1_x2} on the second line.
Since $v_1 \ne v_2$, we see that $y \ne x_1$.
%
Next, by applying step \eqref{eq:v} on $\bar{\mathbf z}_4$,
we get a state $\bar{\mathbf z}_5 = (y, u)$, with
$$
\begin{aligned}
u &= -v_2  + \frac{f(y)}{m} \, \Delta t \\
  &= -v_1  + \frac{f(y) - f(x_2)}{m} \, \Delta t
\end{aligned}
$$
where we have used \eqref{eq:v1_v2} on the second line.
Since $y \ne x_2$, $u \ne -v_1$.
This means the end result $\bar{\mathbf z}_5 \ne \bar{\mathbf z}_1$.


\subsection{Reversibility after a half-step shift}

But why do people still call the leapfrog integrator
is time reversible?  It turns out a step of \eqref{eq:x}
followed by a step of \eqref{eq:v} can be exactly reversed
by a step of \eqref{eq:v} followed by a step of \eqref{eq:x}
(with reversed order)!
%
To understand the reversed ordering,
it helps to think about matrix multiplication.  The inverse matrix
of the product $\mathbf{V X} $ is $\mathbf{X^{-1} V^{-1}}$,
not $\mathbf{V^{-1} X^{-1}}$!

Similarly to inverse a chain of steps
$$
v\mbox{ half-step}
\rightarrow
x\mbox{ half-step}
\rightarrow \cdots \rightarrow
v\mbox{ half-step}
\rightarrow
x\mbox{ half-step},
$$
we need
$$
x\mbox{ half-step}
\rightarrow
v\mbox{ half-step}
\rightarrow \cdots \rightarrow
x\mbox{ half-step}
\rightarrow
v\mbox{ half-step},
$$
The two sequences do look similar,
the only difference is that one starts with a $v$-step
and the other starts with an $x$-step.
%
In other words,
to reverse the leapfrog integrator,
we need to shift a half-step first.
%
This is an inconvenience of the leapfrog integrator.
%
Below we shall show that the velocity Verlet integrator
does not have this inconvenience.


\section{Velocity-Verlet integrator}


The velocity Verlet integrator contains three sub-steps:

\begin{align}
v &\rightarrow v + \frac{f(x)}{2\,m} \Delta t
\tag{vv-V1}
\label{eq:vv_v1}
\\
x &\rightarrow x + v \Delta t
\tag{vv-X}
\label{eq:vv_x}
\\
v &\rightarrow v + \frac{f(x)}{2\,m} \Delta t.
\tag{vv-V2}
\label{eq:vv_v2}
\end{align}

Starting from ${\mathbf z}_1 = (x_1, v_1)$, we have
\begin{align}
v_2 &= v_1 + \frac{f(x_1)}{2m} \Delta t, \\
x_2 &= x_1 + v_2 \, \Delta t, \\
v_3 &= v_2 + \frac{f(x_2)}{2m} \Delta t,
\end{align}
which lands at the state ${\mathbf z}_4 = (x_2, v_3)$.

To reverse it, we start from $\bar{\mathbf z}_4 = (x_2, -v_3)$,
\begin{align}
u_1 &= (-v_3) + \frac{f(x_2)}{2m} \Delta t = -v_2 , \\
y &= x_2 + u_1 \, \Delta t = x_1, \\
u_2 &= u_1 + \frac{f(y)}{2m} \Delta t = -v_2 + \frac{f(x_1)}{2m} \Delta t = -v_1,
\end{align}
which means the final state is
$(y, u_2) = (x_1, -v_2) = \bar{\mathbf z}_1$.
This shows that velocity Verlet integrator is reversible without
the need of shifting a half step.

\end{document}

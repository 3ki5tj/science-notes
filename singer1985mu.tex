\documentclass[10pt]{article}
\usepackage{amsmath}
\begin{document}



\newcommand{\vct}[1]{\mathbf{#1}}
\newcommand{\vr}{\vct{r}}
\newcommand{\vrN}{\mathbf{r}^N}
\newcommand{\vrn}{\mathbf{r}^n}
\newcommand{\plam}{\partial_\lambda}



\title{Singer-Chandler formula for the chemical potential}
\date{}
\maketitle



\section{Basic derivation}

Here we give a detailed derivation of the Singer-Chandler formula
  for the chemical potential\cite{singer}.
For simplicity, we assume a simple classical liquid,
  with a single solute molecule, denoted by the subscript $u$,
  and a one-atom solvent denoted by the subscript $v$ below.
Then, the chemical potential of adding the solute into
  the system is given by
%
\begin{equation}
\beta \mu_u
  =
  \rho_u \int d\vr
  \left[
    - c_{uv}(\vr)
    + \frac{1}{2} h_{uv}^2(\vr)
    - \frac{1}{2} h_{uv}(\vr) \, c_{uv}(\vr)
  \right],
  \label{eq:singer}
\end{equation}
%
where $\rho_u$ is the solvent density,
  $h_{uv}(\vr) = g_{uv}(\vr) - 1$ is the total correlation function
  [with $g_{uv}(\vr)$ being the radial distribution function],
and $c_{uv}(\vr)$ is the direct correlation function.
We refer the reader to Ref. \cite{hansen}
  for the basic concepts.
%



\subsection{Thermodynamic integration}

We start with the thermodynamic integration formula.
If the potential energy
\[
  U(\vrN, \lambda) = U_0(\vrN) + \lambda U_1(\vrN)
\]
depends on the parameter $\lambda$,
then the partition function
\[
  Z(\lambda) = \int d\vrN \exp[-\beta U(\vrN, \lambda)].
\]
satisfies
\begin{align*}
  \frac{ Z'(\lambda) }{Z(\lambda)}
  &=
    -\beta \int d\vrN U_1(\vrN, \lambda)
    \left\{
      \frac{ \exp
             \left[
             -\beta U(\vrN, \lambda)
             \right]}
      { Z(\lambda) }
    \right\} \\
  &=
    -\beta
    \left\langle
      U_1(\vrN, \lambda)
    \right\rangle_\lambda.
\end{align*}
In the final step, note that the term in the braces
  is the normalized Boltzmann weight,
  and the integral is equal to an ensemble average
  $\langle \dots \rangle_\lambda$.
%
Equivalently, in terms of the free energy $\beta F = -\log Z$,
\begin{equation}
  \beta F'(\lambda)
  = \beta
    \left\langle
      U_1(\vrN, \lambda)
    \right\rangle_\lambda.
\end{equation}

Let the first atom be the solute $u = 0$,
  the next $n = N - 1$ atoms be the solvent atoms.
Also,
  let $U_0(\vrn)$ be the potential energy of the $n$ solvent atoms
  and $U_1(\vrN)$ be the interaction between the solute atom $u$
  and the solvent atoms $v = 1, 2, \dots, n$.
By changing $\lambda$ from 0 to 1, we gradually turn on
  the solute-solvent interaction.
%
So the chemical potential $\mu_u = dF/dN$ is
\begin{align}
  \mu_u
  =
  F(1) - F(0)
  =
  \int_0^1 F'(\lambda) \, d\lambda
  =
  \int_0^1
    \left\langle
      U_1(\vrN, \lambda)
    \right\rangle_\lambda d\lambda.
  \label{eq:muti0}
\end{align}




In a simple liquid with pairwise interactions,
  $U_1(\vrN) = \sum_{v = 1}^{n} \phi(\vr_{uv})$,
  where $\phi(\vr_{uv})$ means the pair potential
  between a solvent atom and a solute atom
  (which may be different from the solvent-solvent
  interaction).
So
\begin{equation}
    \left\langle
      U_1(\vrN)
    \right\rangle_\lambda
    =
    n
    \left\langle
      \phi(\vr_{u,1})
    \right\rangle_\lambda
    =
    \rho_u \int \phi(\vr_{u1}) \, g_{uv}(\vr_{u1}, \lambda) \, d\vr_{u1},
  \label{eq:uphig}
\end{equation}
%
where in the second step, we have evaluated the average
pairwise energy in terms of the radial distribution function
$g_{uv}(\vr, \lambda)$.
Using \eqref{eq:uphig} in \eqref{eq:muti0} yields
\begin{equation}
    \mu_u
    =
    \rho_u \int_0^1 d\lambda \,
      \int \phi(\vr_{u1}) \, g_{uv}(\vr_{u1}, \lambda) \, d\vr_{u1}.
  \label{eq:muti}
\end{equation}
Our objective below is to evaluate
$\phi(\vr) \, g(\vr, \lambda)$
by means of the integral equation.




\subsection{Integral equation and the closure}

First,
we define the total correlation function $h(\vr)$ as
\begin{equation}
  h(\vr) \equiv g(\vr) - 1.
  \label{eq:hr}
\end{equation}
%
%
%
Next, the cavity function $y_{uv}(\vr)$ for a solute-solvent pair
\begin{equation}
  y_{uv}(\vr) \equiv e^{ \beta \lambda \phi(\vr) } g_{uv}(\vr).
  \label{eq:yr}
\end{equation}
%
%
%
The direct correlation function $c(\vr)$
is defined in the Ornstein-Zernike relation:
\begin{equation}
  h(\vr) = c(\vr) + \rho \, [c * h](\vr),
  \label{eq:cr}
\end{equation}
where $[c*h](\vr) \equiv \int_{\vct{s}'} c(\vr') \, h(\vr - \vr') \, d\vr'$
  (convolution).
We will give a broken-down version among solvents and/or solutes later.
%
%
%
Finally, we define
\begin{equation}
  t(\vr) \equiv h(\vr) - c(\vr).
  \label{eq:tr}
\end{equation}
%
%
%
And $y(\vr)$ is approximated in the HNC closure as
\begin{equation}
  y(\vr) \approx \exp\left[ t(\vr) \right].
  \label{eq:hnc}
\end{equation}
Since we have five variables
  $g(\vr)$, $h(\vr)$, $c(\vr)$, $t(\vr)$ and $y(\vr)$,
the above five equations permit a solution of $g(\vr)$.



From \eqref{eq:yr}, we have [dropping ``$(\vr)$'' below for simplicity]
\[
  \plam y_{uv}
  =
  \beta \phi \, e^{ \beta \lambda \phi } \, g_{uv}
    +
    e^{ \beta \lambda \phi } \, \plam g_{uv}.
\]
From \eqref{eq:hnc}, we have $\plam y = y \, \plam t$;
and by \eqref{eq:hr}, $\plam g = \plam h$, So
\[
  y_{uv} \, \plam t = e^{\beta \lambda \phi }
  ( \beta \phi g_{uv} + \plam h_{uv} ).
\]
Then by multiplying $e^{-\beta \lambda \phi}$ to both sides,
and using
\eqref{eq:yr}
as $y_{uv} \, e^{-\beta \lambda \phi} \rightarrow 1 + h_{uv}$,
\eqref{eq:tr}
as $\plam t_{uv} \rightarrow \plam h_{uv} - \plam c_{uv}$,
we have
\begin{equation}
  \beta \phi \, g_{uv}
  =
  - \plam c_{uv} + h_{uv} \, \plam h_{uv} - c_{uv} \, \plam h_{uv}.
  \label{eq:hc1}
\end{equation}



Finally, we want to write the right side \eqref{eq:hc1} as
a total differential $\plam(\dots)$.
%
Obviously $h_{uv} \plam h_{uv} = \plam ( h_{uv}^2/2 )$.
%
We argue below that $h_{uv} \plam c_{uv} = \plam ( h_{uv} c_{uv} /2) $
in an \emph{average} sense.
Let us look at \eqref{eq:cr} more carefully.
%
For the solute-solvent pair, we have
\[
  h_{uv} = c_{uv} + \rho_v \, c_{uv}*h_{vv}
  + \rho_u \, c_{uu}*h_{uv},
\]
where $\rho_v$ and $\rho_u$ means the density of
solvent and solute, respectively.
%
Since we have a single solute and many $n >> 1$ solvent atoms,
we can safely set $\rho_u = 0$:
\begin{equation}
  h_{uv} = c_{uv} + \rho_v \, c_{uv}*h_{vv}.
  \label{eq:ozuv}
\end{equation}
Similarly, for the solvent-solvent version
\begin{equation}
  h_{vv} = c_{vv} + \rho_v \, c_{vv}*h_{vv}.
  \label{eq:ozvv}
\end{equation}
From \eqref{eq:ozuv} and \eqref{eq:ozvv}, we know that
  (i) $h_{vv}$ and $c_{vv}$ are independent of the solute,
  and
  (ii) $h_{uv}$ is a linear function with respect to $c_{uv}$.
%
Thus,
\begin{align}
  \int d\vr \, c_{uv}(\vr) \, \plam h_{uv}(\vr)
  &=
  \int c_{uv}(\vr) \, \plam c_{uv}(\vr) d\vr \notag \\
  & \;\;\;\;\;
    +
    \int d\vr \, d\vr'
    \, \rho_v
    \, c_{uv}(\vr)
    \, \plam c_{uv}(\vr') \, h_{vv}(\vr - \vr') \notag \\
  &= \int d\vr \, h_{uv}(\vr) \, \plam c_{uv}(\vr) \notag \\
  &= \int d\vr \plam
    \left[
      \frac{1}{2} h_{uv}(\vr) \,  c_{uv}(\vr)
    \right].
  \label{eq:phig}
\end{align}
%
By using \eqref{eq:phig}, \eqref{eq:hc1} in \eqref{eq:uphig}, we get
%
\begin{align*}
  n \,
  \langle
  \beta \phi(\vr)
  \rangle_\lambda
  &=
  \rho_u \, \int \beta \phi(\vr) \, g_{uv}(\vr) \, d \vr \\
  &=
   \rho_u \, \plam \int \left[
    - c_{uv}(\vr)
    + \frac{1}{2} h_{uv}^2(\vr)
    - \frac{1}{2} h_{uv}(\vr) \, c_{uv}(\vr)
    \right] d \vr.
  \end{align*}
And by \eqref{eq:muti}, we get \eqref{eq:singer}.



\begin{thebibliography}{100}

\bibitem{singer}
  Sherwin J. Singer and David Chandler,
  ``Free energy functions in the extended RISM approximation,''
  Molecular Physics, Vol. 55, No. 3, 621-625,
  1985.

\bibitem{hansen}
  J.-P. Hansen and I.R. McDonald,
  {\it Theory of Simple Liquids}, 3rd ed.,
  Academic Press, London, UK 2006.

\end{thebibliography}

\end{document}


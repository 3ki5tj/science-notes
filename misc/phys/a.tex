\documentclass{article}
\usepackage{amsmath}
\begin{document}

Below are some thoughts about question 3.  That is,

> How to systematically construct a canonical transformation that does not preserve the Hamiltonian, and how should the symplectic condition be relaxed.

Particularly we have the problem in [this post](http://physics.stackexchange.com/questions/89025) in mind.

\section{Normal canonical transformation}


First, we shall define a normal canonical transformation. Given a time-independent Hamiltonian, $H(q, p)$,
there is always a canonical transformation in the most restricted sense (i.e., generated by a generating function, and preserving the symplectic form) that makes all momenta constants of motions.
Let us represent this transformation as
$$
x \equiv (q, p) \rightarrow \xi \equiv (\theta, \phi).
$$
As this transformation can be generated by a generating function,
say $\cal F_2(q, \phi)$,
it preserves the Hamiltonian.
This is because
$
d{\cal F}_2(q, \phi) = p \, dq + \theta \, d\phi
+ [{\cal H}(\theta, \phi) - H(p, q)] \, dt,
$
and since $d\cal F_2(q, \phi)$ does not explicitly depend on time,
${\cal H}(\theta, \phi)-H(p, q)$ must vanish.
Thus, we shall not distinguish ${\cal H}$ and $H$ below.

Since the conjugate coordinates $\theta_i$ do not enter the Hamiltonian, we have
\begin{align}
\dot \phi_i  &= -\frac{ \partial H } { \partial \theta_i } = 0,
\end{align}
which means for all $\phi_i$ are constants as assumed.

We call a transformation that satisfies the above conditions a \emph{normal} transformation,
and the coordinates $\xi = (\theta, \phi)$ normal coordinates,
for a given Hamiltonian $H(q, p)$.


\section{Decomposition of an arbitrary canonical transformation}

Now the point is that any canonical transformation can be decomposed into a product involving normal transformations. Particularly, we shall use

$$
\frac{ \partial X }{ \partial x }
=
\frac{ \partial X }{ \partial \Xi }
\frac{ \partial \Xi }{ \partial \xi }
\frac{ \partial \xi }{ \partial x }.
$$

Here $\Xi$ and $\xi$ are the normal coordinates for $X$ and $x$, respectively.
We shall assume $\frac{ \partial X }{ \partial \Xi }$
and  $\frac{ \partial \xi }{ \partial x }$ are symplectic in the normal sense,
and try to see what we can do with $\frac{ \partial \Xi }{ \partial \xi }$.


\section{Generalized symplectic condition}

Suppose $\xi = (\theta, \phi)$ and $\Xi = (\Theta, \Phi)$
are the coordinates of $H(x)$ and $K(X)$, respectively.
We start from

\begin{align}
\frac{\partial \Theta_i }{\partial \theta_j}
\frac{d \theta_j }{dt}
&=
\frac{d \Theta_i }{dt}.
\end{align}

Expanding the time derivatives yields,
\begin{align}
\sum_{j}
\frac{\partial \Theta_i }{\partial \theta_j}
\frac{\partial H}{\partial \phi_j}
=
\frac{\partial K}{\partial \Phi_i},
\tag{*}
\end{align}

Further, since we are allowed to freely order the constants of motions,
let us set
\begin{align}
\Theta_i &= \theta_i,
\end{align}
for all $i$.  Then
\begin{align}
\frac{\partial H}{\partial \phi_i}
=
\frac{\partial K}{\partial \Phi_i}.
\tag{$*'$}
\end{align}

Let us see how the symplectic matrix is transformed.
\begin{align}
\sum_{j,k}
\frac{\partial \Xi_i }{\partial \xi_j}
J_{jk}
\frac{\partial \Xi_l}{\partial \xi_k}
&=
\sum_{j}
\frac{\partial \Xi_i }{\partial \theta_j}
\frac{\partial \Xi_l}{\partial \phi_j}
-
\frac{\partial \Xi_i }{\partial \phi_j}
\frac{\partial \Xi_l}{\partial \theta_j}
\\
&= J_{il} \frac{\partial \Phi_{[i,l]} }{\partial \phi_{[i,l]} },
\tag{S}
\end{align}
where $[i,l]$ assumes the momentum index of $i$ or $l$.
This means that the result vanishes unless
$i$ and $l$ point to a pair of conjugate variables.
It differs from the usual symplectic condition by a factor
of $\partial \Phi/\partial \phi$.








\section{Arbitrary canonical transformation}


For a general canonical transformation,
we can set up a composition of three canonical transformations.
$$
x \rightarrow \xi \rightarrow \Xi \rightarrow X,
$$
where the transformation $\xi$ to $\Xi$
only needs to satisfy the generalized symplectic condition $(*)$.

Let us see how the symplectic matrix is transformed
\begin{align}
\sum_{l,p}
\frac{\partial X_i }{\partial x_l }
J_{lp}
\frac{\partial X_m }{\partial x_p }
&=
\sum_{k,l,p,q}
\frac{\partial X_i }{\partial \xi_k }
\frac{\partial \xi_k }{\partial x_l }
J_{lp}
\frac{\partial \xi_q }{\partial x_p }
\frac{\partial X_m }{\partial \xi_q }
\\
&=
\sum_{k,q}
\frac{\partial X_i }{\partial \xi_k }
J_{kq}
\frac{\partial X_m }{\partial \xi_q }
\\
&=
\sum_{j,k,q,r}
\frac{\partial X_i }{\partial \Xi_j }
\frac{\partial \Xi_j }{\partial \xi_k }
J_{kq}
\frac{\partial \Xi_r }{\partial \xi_q }
\frac{\partial X_m }{\partial \Xi_r }
\\
&=
\sum_{j,l}
\frac{\partial X_i }{\partial \Xi_j }
J_{jr}
\frac{\partial \Phi_{[j,r]} }{\partial \phi_{[j,r]} }
\frac{\partial X_m }{\partial \Xi_r }
\\
&=
\sum_{j}
\left(
\frac{\partial X_i }{\partial \Theta_j }
\frac{\partial X_m }{\partial \Phi_j }
-
\frac{\partial X_i }{\partial \Phi_j }
\frac{\partial X_m }{\partial \Theta_j }
\right)
\frac{\partial \Phi_j }{\partial \phi_j }
\end{align}
where we have used the normal symplectic condition on the second line for transformation from $x$ to $\xi$,
and $(S)$, on the fourth line.
This is considerably messier than
the strict symplectic condition (the $\Phi_j = \phi_j$ case)
\begin{align}
\sum_{l,p}
\frac{\partial X_i }{\partial x_l}
J_{lp}
\frac{\partial X_m}{\partial x_p}
&=
J_{im}.
\end{align}

\section{The puzzle problem}

Consider the problem in [this post](http://physics.stackexchange.com/questions/89025).
Here $H = p^2/2$ and $K = p^{3/2}/3$,
\begin{align}
Q &= q, \\
P &= \sqrt{p} - \sqrt{q}.
\end{align}

For a single degree of freedom, Eq. $(*')$ can be satisfied by
$\phi_1 = H$,
$\Phi_1 = K$.
As computed by [joshphysis](http://physics.stackexchange.com/q/89038),
$$
\sum_{jl}
\frac{\partial X_i }{\partial x_j }
J_{jl}
\frac{\partial X_m }{\partial x_l }
=
\left( \frac{J_{im}}{2 \sqrt{p}} \right)
=
J_{im}\left( \frac{ dK }{ dH } \right).
$$
is a generalized symplectic matrix,
agreeing with Eq. (S).

It may be interesting to guess how the problem is designed.  In our framework, it is decomposed to three steps.

For the transformation, $(q, p) \rightarrow (\theta, \phi)$,
we have
\begin{align}
\theta &= q/p, \\
\phi &= p^2/2,
\end{align}
which can be generated by ${\cal F}_2 = q \sqrt{2 \phi}$.

For the transformation, $(\theta, \phi)$ to $(\Theta, \Phi)$,
\begin{align}
\Theta &= \theta, \\
\Phi &= (2 \phi)^{3/4}/3.
\end{align}
This step is the strange part,
which cannot be generated by a generating function,
and it does preserve the strict symplectic structure.

For the transformation, $(\Theta, \Phi) \rightarrow (Q, P)$,
we have
\begin{align}
Q &= (3 \Phi)^{2/3} \Theta, \\
P &= (3 \Phi)^{1/3} (1 - \sqrt \Theta),
\end{align}
which can be generated by ${\cal F}_2 = P^3 \Theta/[3 \, (\sqrt\Theta + 1)^2]$.
Or the inverse transformation:
\begin{align}
\Phi &= Q/(P+\sqrt{Q})^2, \\
\Theta &= (P+\sqrt{Q})^3/3,
\end{align}
generated by $F_2^* = (3 \Phi)^{1/3}Q - \frac 2 3 Q^{3/2}$.

\end{document}

\documentclass{article}
\usepackage{amsmath}
\usepackage{bm}
\usepackage[usenames,dvipsnames]{xcolor}
\usepackage{tikz}
\usepackage{hyperref}

\hypersetup{
    colorlinks,
    linkcolor={red!30!black},
    citecolor={blue!50!black},
    urlcolor={blue!80!black}
}


\begin{document}


\title{A primer on the Lie derivative}
\author{ \vspace{-10ex} }
\date{ \vspace{-10ex} }
\maketitle


The purpose of this article is to give a
gentle introduction to the idea of
Lie derivative and operators.
%
Basically, a Lie derivative is an operator,
e.g., a partial derivative $\partial/\partial x$,
or something fancier like $y \, \partial/\partial x$.
%
When this operator is applied to a function,
it generates an another function.
%
We can also build more complex differential operators, like
$\exp(s \, x \partial/\partial x)$.
%
As we shall see,
these operators allows us to
geometrically manipulate the shape
of a function through rotation, twisting, etc.

In application,
differential operators are commonly used classical mechanics
and statistical mechanics.
%
A prominent example is the Liouville operator
which represents the time evolution of
a quantity (an observable or a distribution).

Below, we shall introduce these ideas through
a simple problem of manipulating the shape of a function.



\section{Reshaping a function a through operators}


A function
\begin{equation}
  y = f(x)
  \label{eq:y_fx}
\end{equation}
can be regarded as a curve of $y$ versus $x$.
Let us consider the problem of changing the shape of this curve.


\subsection{Operations along the $y$ axis}

To translate $f(x)$ along the $-y$ axis by $b$,
we only need to add a constant $c$ to it:
\begin{equation}
  f(x) \rightarrow f(x) + c.
  \label{eq:ys_shift}
\end{equation}
So we can write this operation through an operator
\begin{equation}
  T_y^* f \equiv f + c.
  \label{eq:ys_shift2}
\end{equation}

To scale it along the $y$ axis, we only need to multiply it
by a constant $A$, that is
\begin{equation}
  f(x) \rightarrow A \, f(x).
  \label{eq:ys_scale}
\end{equation}
This can be written in terms of an operator
\begin{equation}
  S_y^* f \equiv A f.
  \label{eq:ys_scale2}
\end{equation}


\subsection{Operations along the $x$ axis}

But how about the same operations along the $x$ axis?

If we shift a function $f(x)$ to the \emph{left} by $a$,
the resulting function is $f(x + a)$.
Can we achieve this result by an operator?
The answer is yes!
Consider the Taylor expansion:
$$
f(x + a) = \sum_{n = 0}^\infty \frac{a^n}{n!} \frac{d^n f(x) }{dx^n}
\equiv \exp\left(a \frac{d}{dx} \right) f(x).
$$
The last step is a shortcut of
writing all orders of derivatives,
$$
T_x \equiv
\exp\left(a \frac{d}{dx} \right)
\equiv
\sum_{n = 0}^\infty \frac{a^n}{n!} \frac{d^n }{ dx^n}.
$$
We shall write convenient shortcut
a differential operator, $T_x$.
%
This operator obviously behaves kind of different from $T_y^*$,
so we shall drop the $*$ on the superscript to distinguish
the two.
%
Another point to note is that the $+$ sign in $T_y^*$
means moving the curve along $+y$ axis,
while the $+$ sign in $T_x$ means
moving the curve along $-x$ axis.
%
The point is that
applying $T_x$ to any function $f(x)$, we get $f(x+a)$.

Similarly, if we \emph{compress} the function along the $x$ axis by a factor of $s$,
then the resulting function is given by $f(s \, x)$.
To find the corresponding operator note that
$$
\begin{aligned}
f(s \, x)
&= f( e^{\ln x + \ln s} ) \\
&= \sum_{n = 0}^\infty \frac{(\ln s)^n}{n!} \frac{d^n f(e^{\ln x + \ln s}) }{ d(\ln x)^n } \\
&= \exp\left[ \ln s \frac{d}{d\ln x} f(x) \right]
= s^{x \frac{d}{dx} } f(x).
\end{aligned}
$$
So the scaling operation along the $x$ axis is given by
$$
S_x \equiv s^{x \frac{d}{dx} }.
$$



\subsection{Composition}


We can join two shaping operations by composition.
%Intuitively, this means two successive substitutions.


\subsubsection{Composition of operations along $y$}

For example, if we want to first translate the function by $c$ vertically,
and then stretch it by a factor of $A$ along the $y$ axis.
Then the resulting function is $A \, [f(x) + c]$
and it can be achieved by a composition of the operators
of $S_y^*$ and $T_y^*$:
\begin{equation}
S_y^* T_y^* f(x) = S_y^* [f(x) + c] = A \, [f(x) + c].
\label{eq:SyTy}
\end{equation}
Note that the first operation $T_y^*$ is placed on the right,
so it is closer to the function $f$.


\subsubsection{Composition of operations along $x$}

Now consider the corresponding composite operation along the $x$ axis.
%
If we translate the function to the left by $a$,
and then compress the resulting curve along $x$
by a factor $s$,
the resulting function should be
$$
  f(sx+a).
$$
%
It is naturally to draw analogy from Eq. \eqref{eq:SyTy} that
the above expression can be obtained from $S_x \, T_x \, f(x)$.
Indeed,
$$
S_x T_x f(x) = S_x f(x + a) = f(s x + a).
$$

\subsection{Unification of $x$ and $y$ operations}

The asymmetry between $x$ and $y$ operations
is somewhat frustrating.
%
Besides, $x$ and $y$
operations appear to have different signs:
moving in the $+y$ direction was associated with $+c$,
whereas moving in the $-x$ direction was associated with $+a$.
So is there a way to unify the two?
The answer is again yes.
The reason of asymmetry is that
we write the curve in terms of $y$ as a function of $x$.
%
If we express $x$ and $y$ in equal footing,
then the asymmetry disappears.

Consider the writing the function
$$
F(x, y) = f(x) - y = 0.
$$
This is a two-variable function
in which $x$ and $y$ are symmetric,
although it means pretty much the same thing as
Eq. \eqref{eq:y_fx}.
So we may guess that in this form
the $y$ operations are the same as the $x$ operations.
Let us verify that.

First, after a shift along the $-y$ axis by $a_y$, we expect
the function $F(x, y)$ to become $F(x, y + a_y)$
$$
F(x, y + a_y) = f(x) - (y + a_y) = 0.
$$
This means
$$
  y = f(x) - a_y.
$$
which agrees with Eq. \eqref{eq:ys_shift} with $c = - a_y$
(i.e., moving the curve along $+y$ by $c = -a_y$).
But now this operation can be accomplished with an $x$-type
operator applied to $F(x, y)$,
\begin{equation}
  T_y \equiv
  \exp\left(a_y \frac{ \partial } { \partial y } \right).
  \label{eq:y_shift}
\end{equation}
We can readily verify
$$
\begin{aligned}
T_y F(x, y)
&= \exp\left(a_y \frac{ \partial } { \partial y } \right) F(x, y) \\
&= \sum_{n = 0}^\infty \frac{ a_y^n }{n!}
\frac{ \partial^n } {\partial y^n} F(x, y) \\
&= F(x, y + a_y).
\end{aligned}
$$


Second,
a compression of the curve
along the $y$ axis by a factor of $s_y$,
can be accomplished by
\begin{equation}
  S_y \equiv
  (s_y)^{ y \, \frac{ \partial } { \partial y } }.
  \label{eq:y_scale}
\end{equation}
Then
$$
S_y \, F(x, y)
=
F(x, s_y y)
= f(x) - s_y \, y.
$$
This is equivalent to
$$
y = \frac{1}{s_y} f(x),
$$
which agrees with Eq. \eqref{eq:y_shift} with $A = 1/s_y$.


\subsection{Generalization}

Of course, we can generally develop the corresponding
operator $T_x^*$ and $S_x^*$, by writing $y = f(x)$ inversely as
$$
x = f^{-1}(y).
$$
The $x$-type (non-star) operators, such as $T_x$ and $S_y$, turns
out to be much easier to handle for a multivariate
function $F(x_1, \dots, x_n)$.  Because, we simply cannot invert
this function to
$$
x_n = f_n(x_1, \dots, x_{n-1}).
$$
On the other hand, the operation along the $k$ axis is
simply some function of $x_k$ and $\frac{ \partial }{ \partial x_k}$
$$
O_{x_k} = O\left(x_k,  \frac{ \partial } { \partial x_k }\right).
$$
For example, a shift along the $-x_k$ axis is given by
$$
T_{x_k} = \exp\left( a_k \frac{ \partial } { \partial x_k } \right).
$$


Further, we can mix different components, for example,
consider a rotation of $\theta$ in the $x$-$y$ plane is given by
$$
R = \exp\left[i\theta \left(x \frac{\partial}{\partial y} - y \frac{\partial }{\partial x} \right) \right]
$$
To verify this, let us denote
$$
L = x \frac{\partial}{\partial y} - y \frac{\partial }{\partial x}.
$$
So for an infinitesimal $\Delta \theta$, we have
$$
\begin{aligned}
\exp( \Delta \theta \, L) f(x, y)
&= (1 + \Delta \theta \, L) f(x, y) \\
&= f(x, y) + \Delta \theta \, x \frac{ \partial f } { \partial y }
- \Delta \theta \, y \frac{ \partial f } { \partial x }
\\
&\approx f(x - \Delta \theta \, y, y + \Delta \theta \, x).
\end{aligned}
$$
This means the argument of function $f$ is now
$$
\left(
  \begin{array}{ccc}
    1 & -\Delta \theta \\
    \Delta \theta & 1
  \end{array}
\right)
\left(
  \begin{array}{ccc}
    x \\
    y
  \end{array}
\right)
=
\exp\left[\Delta \theta \left(
  \begin{array}{ccc}
    0 & -1 \\
    1 & 0
  \end{array}
\right)
\right]
\left(
  \begin{array}{ccc}
    x \\
    y
  \end{array}
\right)
$$
So if we apply $\exp(\Delta \theta \, L)$ twice,
we the argument of function would be
$$
\begin{aligned}
&
\exp\left[
  \Delta \theta
\left(
  \begin{array}{ccc}
    0 & -1 \\
    1 & 0
  \end{array}
\right)\right]
\exp\left[
  \Delta \theta
\left(
  \begin{array}{ccc}
    0 & -1 \\
    1 & 0
  \end{array}
\right)\right]
\left(
  \begin{array}{ccc}
    x \\
    y
  \end{array}
\right)
\\
&=
\exp\left[
  2\Delta \theta
\left(
  \begin{array}{ccc}
    0 & -1 \\
    1 & 0
  \end{array}
\right)\right]
\left(
  \begin{array}{ccc}
    x \\
    y
  \end{array}
\right).
\end{aligned}
$$
%
We can show that with a large number of applications
and $\Delta \theta = \theta / n$,
the vector $(x, y)^T$ is right multiplied by the following matrix
$$
\exp\left[
  \theta
\left(
  \begin{array}{ccc}
    0 & -1 \\
    1 & 0
  \end{array}
\right)\right]
=
\left(
  \begin{array}{ccc}
    \cos \theta & -\sin \theta \\
    \sin \theta & \cos \theta
  \end{array}
\right).
$$
To show this we denote
$$
\mathbf J
=
\left(
  \begin{array}{ccc}
    0 & -1 \\
    1 & 0
  \end{array}
\right),
$$
then
$$
\mathbf J^2 = -\mathbf I, \quad
\mathbf J^3 = -\mathbf J, \quad
\mathbf J^4 = \mathbf I,
$$
where $\mathbf I$ is the identity matrix.
So
$$
\exp(\theta \mathbf J)
=
\sum_{n = 0}^\infty \frac{\theta^n}{n!} \mathbf J^n
=
\sum_{n = 0}^\infty \frac{\theta^{2n}}{(2n)!} (-)^n \mathbf I
+
\sum_{n = 0}^\infty \frac{\theta^{2n+1}}{(2n+1)!} (-)^n \mathbf J
=
\cos\theta \mathbf I
+
\sin \theta \mathbf J.
$$
which is the identity to be shown.

Generally, a geometric operator (such as a shift, scaling and rotation)
can be written as am exponential operator:
$$
\mathbf O = \exp(t \mathbf G),
$$
where $\mathbf G$ is called the \emph{generator},
and the parameter $t$ is the magnitude.


% application
% the derivation of the perturbation formula
% the method of characteristics

\end{document}

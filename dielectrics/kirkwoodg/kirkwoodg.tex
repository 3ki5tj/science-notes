\documentclass[11pt]{article}
\usepackage[width=7.0in, height=10.0in]{geometry}
\usepackage{hyperref}
\usepackage{amsmath}
\usepackage{upgreek}
\usepackage{bm}



\newcommand{\vct}[1]{\boldsymbol{\mathbf{#1}}}
\newcommand{\vr}{\vct{r}}
\newcommand{\vrN}{\vr^N}
\newcommand{\vrn}{\vr^n}
\newcommand{\dvr}{\frac{ d \vr  }{(2\pi)^3}}
\newcommand{\vx}{\vct{x}}
\newcommand{\vxN}{\vx^N}
\newcommand{\vxn}{\vx^n}
\newcommand{\dvx}{\frac{ d \vx  }{(2\pi)^3}}
\newcommand{\vk}{\vct{k}}
\newcommand{\dvk}{\frac{ d \vk  }{(2\pi)^3}}
\newcommand{\vy}{\vct{y}}
\newcommand{\vz}{\vct{z}}
\newcommand{\vE}{\vct{E}}
\newcommand{\vD}{\vct{D}}
\newcommand{\vP}{\vct{P}}
\newcommand{\vM}{\vct{M}}
\newcommand{\vdel}{\vct{\updelta}}



\begin{document}


\title{Kirkwood's formula for the dielectric constant}
\author{ \vspace{-10ex} }
\date{ \vspace{-10ex} }
\maketitle


This note tries to explain the Kirkwood's 1939 paper\cite{kirkwood1939a}
on the dielectric constant.
%
We try to use as basic mathematics
(and to avoid the Legendre polynomials when possible).
%
The reader is, however, encourage to review Feynman's Lectures on Physics,
Vol. II Chapters 10 and 11.
%
We will use the SI units throughout.
If the reader who prefers the Gaussian units
should simply replace $\epsilon_0$ by $1/(4 \pi)$.



\section{Warm-up models}


First, we recall that the Maxwell equations governing electrostatics
are linear with respect to the fields,
such as the electric field $\vE$ and the charge density $\rho$.
%
This simple fact is very useful,
for it allows us to build up a complex field
by linearly combining simpler electric fields.

In this section,
we will study a few basic model of polarization.
%
These models will be used as building blocks
to understand the more complex Kirkwood model for
the dielectric constant.



\subsection{Uniformly polarized ball}



The effect of a uniformly polarized ball of radius $R$ is equivalent to
that of a uniformly and positively charged ball
located at $\vct c_+ = (0, 0, \delta/2)$ (with charge density $\rho$)
and that of a uniformly and negatively charged ball
located at $\vct c_- = (0, 0, -\delta/2)$,
with $\delta \rightarrow 0$.


The electric field caused by the positive ball
can be computed using the Gauss theorem.
%
For $r_+ < R$, we have
\[
  E 4 \pi r_+^2
=
  \frac { 4 \pi r_+^3 } { 3 } \frac{ \rho } { \epsilon_0 },
\]
where
\[
  r_\pm = |\vr - \vct c_\pm| = r \mp \frac{ \delta \cos\theta} { 2 },
\]
with $\theta$ being the angle between $\vr$ and $\vz$,
i.e., $\hat\vr \cdot \hat\vz = \cos\theta$.
So
\[
  \vE_+
=
  \frac{ \rho } { 3 \, \epsilon_0 } ( \vr - \vct c_+ ).
\]
By symmetry the field caused by the negatively charged ball
is given by
\[
  \vE_- = -\frac{ \rho } { 3 \, \epsilon_0 } ( \vr - \vct c_- ).
\]
The superposition of the two yields
\begin{align}
  \vE_{\vP} = \vE_+ + \vE_-
= - \frac{ \rho \, \vct\delta } { 3 \epsilon_0 }
= - \frac{ \vP } { 3 \epsilon_0 },
  \label{eq:EP_in}
\end{align}
where $\vdel = \delta \vz$.
%

Outside of the ball, we can also use the Gauss theorem.
\[
  E 4 \pi r_+^2
=
  \frac { 4 \pi R^3 } { 3 } \frac{ \rho } { \epsilon_0 },
\]
and
\[
  \vE_+
=
  \frac{ \rho \, R^3} { 3 \, \epsilon_0 \, r_+^3} ( \vr - \vct c_+ ).
\]
Similarly,
\[
  \vE_-
=
  -\frac{ \rho \, R^3} { 3 \, \epsilon_0 \, r_-^3} ( \vr - \vct c_- ).
\]
The sum of the above two is the electric field outside the ball.
For a small $\delta$,
we get
\begin{align}
  \vE_{\vP}
&=
  -\frac{ \rho \, R^3 \, (\vct c_+ - \vct c_-) } { 3 \, \epsilon_0 \, r^3 }
  +\frac{ \rho \, R^3 \vr } { 3 \, \epsilon_0 }
   \left(
      \frac{ 1 } { r_+^3 }
      -
      \frac{ 1 } { r_-^3 }
   \right)
   \notag \\
&=
  -\frac{ \rho \, R^3 \vdel } { 3 \, \epsilon_0 \, r^3 }
  +\frac{ \rho \, R^3 \vr } { \epsilon_0 }
      \frac{ \delta \cos\theta } { r^4 }
   \notag \\
&=
  -\frac{ 1 } { 4 \pi \epsilon_0 }
  \frac{ \vM } {r^3}
  +
  \frac{ 1 } { 4 \pi \epsilon_0 }
  \frac{ 3 \, (\vM \cdot \hat\vr) \, \hat\vr } { r^3 }
   \notag \\
&=
  \frac{ 1 } { 4 \pi \epsilon_0 }
  \frac{ \vM \cdot (3 \, \hat\vr \, \hat\vr - 1) } { r^3 }
=
  -\nabla \left(
  \frac{ 1 } { 4 \pi \epsilon_0 }
  \frac{ \vM \cdot \vr } {r^3}
  \right).
  \label{eq:EP_out}
\end{align}
where
\[
  \vM =
  \frac{ 4 \pi R^3 \, \rho \, \vdel } { 3 }
  =
  \frac{ 4 \pi R^3 } { 3 } \, \vP.
\]
is the total dipole moment of the ball.


Physically,
%
A uniformly polarized ball produces,
within the ball,
a uniform field
that is $-1/\epsilon_0$ times the polarization vector,
by Eq. \eqref{eq:EP_in};
and
Eq. \eqref{eq:EP_out} says that
the electric field outside the ball
is equivalent to that of a dipole of moment $\vM$.




\subsection{Dielectric ball in uniform field}



This model is used in the paragraph between Eqs. (6) and (7)
on page 913 of Kirkwood's paper.

First, let us consider a solid ball of radius $R$ with
the dielectric constant $\epsilon$.
The ball is placed in vacuum
under a constant electrostatic field
$\vE_0 = E_0 \vz$
along the $z$ axis.
We wish to know the electric field within the ball.

Solution.
Let us assume that the ball is uniformly polarized,
with polarization vector $\vP$.
Then the electric field caused the by the polarization charges
is given by $-\vP/\epsilon_0$,
according to Eq. \eqref{eq:EP_in}.
The total electric field (for $r < R$) is given by
\[
  \vE = \vE_0 - \frac{ \vP } { 3 \epsilon_0 }.
\]
Further, we have
\[
  \vP = (\epsilon - 1) \vE.
\]
Eliminating $\vP$ from the two above equations yields
\[
  \vE = \frac{ 3 } {2 + \epsilon} \vE_0.
\]



\section{Three-tier (Kirkwood's) model}



The Kirkwood formula for the dielectric constant centers around a three-tier model.




\section{Errata}



Errors of the original paper are listed below.

On page 918, the $\epsilon$ on the left-hand side of Eq. (28)
should be dropped:
\begin{align}
  -\vct\upmu^*\cdot
  \nabla
  \left(
    \frac { 1 }
    { |\vr - \vr'| }
  \right)
&=
  \sum_{n = 0}^\infty
  \sum_{m = -n}^{+n}
  \frac { ( n - |m| )! } { ( n + |m| )! }
  \notag
  \\
&\hphantom{=}\times
  \frac { A_{nm} } { r^{n + 1} }
  P_n^m(\cos\vartheta) e^{im \varphi}.
  \tag{28}
\end{align}

On the same page, the term $\epsilon - 1$ in Eq. (33) misses a square sign.
\begin{align}
\overline {\vM}
=
\overline {\vM}(R, r_0)
-
\frac{2}{3\epsilon}
\frac{(\epsilon - 1)^2}{\epsilon + 2}
\vct\upmu^*
+ O\left( \frac{ r_0^3 } { R^3 } \right).
\tag{33}
\end{align}


\bibliography{../../liquid}
\bibliographystyle{alpha}
\end{document}

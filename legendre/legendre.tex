\documentclass[11pt]{article}
\usepackage[width=7.0in, height=10.0in]{geometry}
\usepackage{hyperref}
\usepackage{amsmath}
\usepackage{upgreek}



\newcommand{\vct}[1]{\boldsymbol{\mathbf{#1}}}
\newcommand{\vr}{\vct{r}}
\newcommand{\vrN}{\vr^N}
\newcommand{\vrn}{\vr^n}
\newcommand{\dvr}{\frac{ d \vr  }{(2\pi)^3}}
\newcommand{\vx}{\vct{x}}
\newcommand{\vxN}{\vx^N}
\newcommand{\vxn}{\vx^n}
\newcommand{\dvx}{\frac{ d \vx  }{(2\pi)^3}}
\newcommand{\vk}{\vct{k}}
\newcommand{\dvk}{\frac{ d \vk  }{(2\pi)^3}}



\begin{document}



\title{Notes on the Legendre polynomials}
\author{ \vspace{-10ex} }
\date{ \vspace{-10ex} }
\maketitle



This note concerns some useful formulae of
the (associated) Legendre polynomial
and spherical harmonics.
%
Our focus are
\begin{enumerate}
  \item The addition theorem of the Legendre polynomial and spherical harmonics.
  \item The Rayleigh equation, which relates the Legendre polynomial
        to the spherical Bessel functions.
  %\item Wave equation on a sphere.
\end{enumerate}

Although we try to be as self-contained as possible,
we assumed that the reader
has a basic understanding of the two functions
and the Bessel functions
(Arfken and Weber's Mathematical Methods for Physics\cite{arfken}
is an excellent source).



%\tableofcontents



\section{Review (cheat sheets)}



We shall briefly review some definitions and basic properties of the functions.



\subsection{Legendre polynomials}



The generating function of Legendre polynomials $P_n(x)$ is
%
\begin{equation}
  g(x, t)
\equiv
  (1 - 2 x t + t^2)^{-1/2}
=
  \sum_{n = 0}^\infty P_n(x) \, t^n.
  \label{eq:legendre_gf}
\end{equation}

A more explicit formula, called Rodrigues' formula is
%
\begin{equation}
  P_n(x)
=
  \frac{1}{2^n n!}
  \frac{d^n}{dx^n}
  (x^2 - 1)^n.
  \label{eq:legendre_rodrigues}
\end{equation}
%
Equivalently, we can write it as a contour integral,
called Schl\"{a}fli (Schlaefli) integral, around $t = x$, as
\begin{equation}
  P_n(x)
=
  \frac{1}{2\pi i \, 2^n}
  \oint
  \frac{ (t^2 - 1)^n } { (t - x)^{n+1} } \, dt.
  \label{eq:legendre_schlaefli}
\end{equation}


The Legendre polynomial satisfy the differential equation
\begin{equation}
  \frac{d}{dx}
  \left[
    (1 - x^2) \, \frac{d}{dx} P_n(x)
  \right]
+ n (n + 1) \, P_n(x) = 0.
\label{eq:legendre_diffeq}
\end{equation}

The Legendre polynomials are orthogonal with each other
[which can be shown using Eq. \eqref{eq:legendre_diffeq}],
and the normalization is given by
\begin{equation}
  \int_{-1}^1
  P_n(x) \, P_{n'}(x) \, dx
=
  \frac { 2 } { 2 n + 1} \delta_{n, n'},
  \label{eq:legendre_orthonormal}
\end{equation}
which can shown by squaring
the generating function in Eq. \eqref{eq:legendre_gf},
and integrating from $-1$ to 1.

Below are the first few Legendre polynomials:
%
\begin{align*}
  P_0(x) &= 1, &
  P_1(x) &= x, \\
  P_2(x) &= \frac{1}{2}(3 x^2 - 1), &
  P_3(x) &= \frac{1}{2}(5 x^3 - 3 x),
  \qquad \dots,
\end{align*}
from which we see that
Legendre polynomials $P_n(x)$ share the same parity with $n$.
For more values, one can use the recurrence relation:
\begin{equation}
  (2 n + 1) x P_n(x) = (n + 1) P_{n+1}(x) + n P_{n-1}(x),
  \label{eq:legendre_recur_dt}
\end{equation}
which can be readily shown from the generating function in Eq. \eqref{eq:legendre_gf},
for
$(1 - 2 x t + t^2) \, \partial_t g = (x - t) \, g$.

Finally, some special values are listed below,
which can be shown by, e.g., Eq. \eqref{eq:legendre_schlaefli}:
%
\begin{align}
  P_n(1)    &= 1,
  \label{eq:legendre_1} \\
  P_n(-1)   &= (-1)^n,
  \label{eq:legendre_neg1} \\
  P_{2n}(0) &= \frac{ (-1)^n (2 n)! }{ 2^{2n} n!^2 }
             = (-1)^n \frac{ (2 n - 1)!! } { (2n)!! },
  \qquad
  P_{2n-1}(0) = 0.
  \label{eq:legendre_0}
\end{align}



\subsection{Associated Legendre polynomials}



The Rodrigues' formula of the associated Legendre polynomials is
%
\begin{equation}
  P_n^m(x)
= (1 - x^2)^{m/2}
  \frac{d^m}{dx^m}
  P_n(x)
=
  \frac{(1 - x^2)^{m/2}}{2^n n!}
  \frac{d^{n+m}}{dx^{n+m}}
  (x^2 - 1)^n.
  \label{eq:alegendre_rodrigues}
\end{equation}
%
The equivalent Schl\"{a}fli integral, around $t = x$, is
\begin{equation}
  P_n^m(x)
=
  \frac{ (n + m)! } { n! }
  \frac{ (1 - x^2)^{m/2} }{2\pi i \, 2^n}
  \oint
  \frac{ (t^2 - 1)^n } { (t - x)^{n+m+1} } \, dt.
  \label{eq:alegendre_schlaefli}
\end{equation}

The generating function of the associated Legendre polynomials is
%
\begin{equation}
  g_m(x, t)
\equiv
  \frac{ (2 m - 1)!! \, (1 - x^2)^{m/2} }
  { (1 - 2 x t + t^2)^{m + 1/2} }
=
  \sum_{s = 0}^\infty P_{s+m}^m(x) \, t^s.
  \label{eq:alegendre_gf}
\end{equation}

The associated Legendre polynomial satisfy the differential equation
\begin{equation}
  \frac{d}{dx}
  \left[
    (1 - x^2) \, \frac{d}{dx} P_n^m(x)
  \right]
+
  \left[
    n (n + 1) - \frac{m^2}{1 - x^2}
  \right]
  \, P_n^m(x) = 0.
\label{eq:alegendre_diffeq}
\end{equation}

The associated Legendre polynomials satisfy
\begin{equation}
  \int_{-1}^1
  P_n^m(x) \, P_{n'}^m(x) \, dx
=
  \frac { 2 } { 2 \, n + 1 }
  \frac{ (n + m)! } { (n - m)! }
  \,
  \delta_{n, n'},
  \label{eq:alegendre_orthonormal}
\end{equation}
%
and
%
\begin{equation}
  \int_{-1}^1
  \frac{
    P_n^m(x) \, P_n^{m'}(x)
  } {
    \sqrt{ 1 - x^2 }
  }\, dx
=
  \frac{ (n + m)! } { m \, (n - m)! }
  \,
  \delta_{m, m'}.
  \label{eq:alegendre_orthonormal2}
\end{equation}

$P_n^m(x)$ and $P_n^{-m}(x)$ are linearly related:
\begin{equation}
  \frac{ P_n^m(x) } { (n + m)! }
=
  (-1)^m
  \frac{ P_n^{-m}(x) } { (n - m)! }.
  \label{eq:PnmPnnegm}
\end{equation}

In analogous to \eqref{eq:legendre_recur_dt},
we have
$(1 - 2 x t + t^2) \, \partial_t g_m = (2 m + 1) (x - t) \, g$,
and
\begin{equation}
  (2 n + 1) \, x \, P_n^m(x) = (n - m + 1) \, P_{n+1}^m(x) + (n + m) \, P_{n-1}^m(x).
  \label{eq:alegendre_recur_dt}
\end{equation}

Finally, some special values, derived from, e.g., Eq. \eqref{eq:alegendre_schlaefli}, are listed below:
\begin{align}
  P_m^m(x)    &= (1 - x^2)^{m/2} \, (2 m - 1)!!,
  \label{eq:alegendre_mm} \\
  P_n^m(1)    &= \delta_{m, 0},
  \label{eq:alegendre_1} \\
  P_n^m(-1)   &= (-1)^n \delta_{m, 0},
  \label{eq:alegendre_neg1} \\
  P_{m+2s}^m(0) &= (-)^s \, \frac{ (2m + 2s - 1)!! }{ (2s)!! },
  \qquad
  P_{m+2s-1}^m(0) = 0.
  \label{eq:alegendre_0}
\end{align}



\subsection{Spherical harmonics}



Spherical harmonics are defined as
normalized associated Legendre polynomials:
\begin{equation}
  Y_n^{m}(\theta, \phi)
=
  (-1)^{m}
  \sqrt{
    \frac{ 2 n + 1 } { 4 \pi }
    \frac{ (n - m)! } { (n + m)! }
  }
  \,
  P_{n}^{m}\bigl( \cos\theta \bigr) e^{i m \phi},
  \label{eq:Ynm}
\end{equation}

The normalization condition reads
\begin{equation}
  \int_0^{2\pi} \int_0^\pi
  Y_n^{m}(\theta, \phi) \, Y_{n'}^{m'}(\theta, \phi)
  \, \sin\theta \, d\theta \, d\phi
  = \delta_{n, n'} \, \delta_{m, m'}.
  \label{eq:Ynm_orthonorm}
\end{equation}

Spherical harmonics of positive and negative $m$ are
related by complex conjugation:
\begin{equation}
  Y_n^{-m}(\theta, \phi) = (-)^m Y_n^{m*}(\theta, \phi).
\end{equation}

Spherical harmonics satisfy the differential equation
\begin{equation}
  \frac{1}{\sin\theta}
  \frac{d}{d\theta}
  \left[
    \sin\theta
    \frac{d}{d\theta}
    Y_n^m(\theta, \phi)
  \right]
  +
  \frac{1}{\sin^2\theta}
  \frac{d^2 Y_n^m(\theta, \phi)}{d\phi^2}
  +
  n (n + 1) Y_n^m(\theta, \phi)
  = 0.
  \label{eq:Ynm_diffeq}
\end{equation}
%
Now the first two terms are the angular part of
Laplacian $\nabla^2$ (see Appendix \ref{sec:spcoords_laplacian}).
%
So we can write
\begin{equation}
  \nabla^2 Y_n^m(\theta, \phi)
  =
  -n \,(n + 1) \, Y_n^m(\theta, \phi),
  \qquad \mbox{for $r = 1$}.
  \label{eq:Ynm_diffeq2}
\end{equation}
%
In other words,
the spherical harmonics
are the solutions of the wave equation
in spherical polar coordinates,
just as the exponential functions $\psi(\vr) = \exp(i\vk\cdot\vr)$
are the solutions of the wave equation
in the Cartesian coordinates:
\[
  \nabla^2 \psi(\vr) = -k^2 \psi(\vr).
\]



\subsection{Spherical Bessel functions}



As we shall see in Sec. \ref{sec:rayleigh_proof},
the Legendre polynomials
and spherical Bessel functions
are related by Fourier transform.
%
Here we will briefly review the latter.

The spherical Bessel function is defined as
\begin{align}
j_n(x)
&= \sqrt{ \frac{ n } { 2 x } } J_{n + 1/2}(x) \\
&= 2^n x^n
\sum_{s = 0}^\infty
\frac { (-)^s (s + n)! }
{ s! \, (2 s + 2 n + 1)! }
\,
x^{2s}
\label{eq:besselj}
\end{align}

The first few spherical Bessel functions are
\begin{align*}
  j_0(x) &= \frac { \sin x } { x }, \\
  j_1(x) &= \frac { \sin x - x \, \cos x } { x^2 }, \\
  j_2(x) &= \frac { 3 \, \sin x - 3 \, x \, \cos x - x^2 \, \sin x } { x^3 },
\end{align*}

There are two useful recurrence relations:
\begin{align}
  \frac{d}{dx}
  \left[
    x^{n+1} j_n(x)
  \right]
&=
  x^{n+1} j_{n - 1}(x)
\label{eq:recur_jn1}
  \\
  \frac{d}{dx}
  \left[
    \frac{ j_n(x) }{ x^n }
  \right]
&=
  - \frac{ j_{n+1}(x) }{ x^n }.
\label{eq:recur_jn2}
\end{align}



\section{Addition theorem of the Legendre polynomial}



Consider two arbitrary points 1 and 2 on the unit sphere,
whose positions are described in solid angles
$\omega_1 = (\theta_1, \phi_1)$
and
$\omega_2 = (\theta_2, \phi_2)$, respectively.
%
Further define $\gamma$ as the angle between points 1 and 2.
%
The addition theorem states that
\begin{equation}
  P_n(\cos\gamma)
= \sum_{m = -n}^n \frac{ 4 \pi } { 2 n + 1 }
  Y_n^m(\theta_1, \phi_1) \, Y_n^{m*}(\theta_2, \phi_2),
\label{eq:addition}
\end{equation}
where $Y_n^m$ is the spherical harmonics,
and the ``$*$'' for complex conjugation
can be assigned to either spherical harmonic.
Below is a somewhat detailed proof of this theorem.



\subsection{Expansion in spherical harmonics}



First, we can expand the $P_n(\cos\gamma)$
as a combination of spherical harmonics $Y_{n'}^m(\theta_1, \phi_1)$,
which are orthonormal.
%
One can argue that the relevant $Y_{n'}^m$'s must share the same $n$,
i.e., $n = n'$.\footnote{
  Both sides of \eqref{eq:expansion1}
  satisfy the differential equation of $Y$:
\[
  \frac{ 1 } { \sin \theta }
  \frac{ \partial } { \partial \theta }
  \left(
    \sin \theta
    \,
    \frac{ \partial Y }{ \partial \theta }
  \right)
  +
  \frac { 1 } { \sin^2 \theta }
  \frac { \partial^2 Y } { \partial \phi^2 }
  + n \, (n - 1)  \, Y = 0.
\]
This would not be so,
if the right-hand side contains $n' \ne n$ terms.
}
%
So
\begin{equation}
  P_n(\cos\gamma)
=
  \sum_{m = -n}^n a_m Y_n^m(\theta_1, \phi_1).
  \label{eq:expansion1}
\end{equation}
%
where the coefficient $a_m$ can be computed from the integral
\begin{align}
  a_m
&=
\int d\Omega_1 \, P_n(\cos\gamma) \, Y_n^{m*}(\theta_1, \phi_1),
\label{eq:am1}
\end{align}
where $d\Omega_1$ means
$d(-\cos\theta_1) \, d\phi_1 = \sin\theta_1 d\theta_1 \, d\phi_1$.
%
We want to show that
$a_m = 4 \pi Y_n^m(\theta_2, \phi_2)/ ( 2 n + 1 )$.



\subsection{Spherical harmonics in a new coordination system}



Now let us imagine a new coordination system,
in which the $z$ axis points from the origin to point 2 on unit sphere.
%
In this special coordination system,
we also have a set spherical harmonics $Y_n^m(\gamma, \psi)$,
among these, we have
\[
  P_n(\cos\gamma)
=
\sqrt \frac { 4 \pi } { 2 n + 1 }
  Y_n^0(\gamma, \psi).
\]
Thus, we can rewrite $a_m$ in Eq. \eqref{eq:am1} as
\begin{align}
a_m
&=
\sqrt \frac{4 \pi}{ 2 n + 1 }
\int d\Omega_1 \, Y_n^0(\gamma, \psi) \, Y^*_{n,m}(\theta_1, \phi_1)
=
\sqrt \frac{4 \pi}{ 2 n + 1 } A_m.
\label{eq:am2}
\end{align}
%
where,
\begin{equation}
A_m
\equiv
\int d\Omega_1 \, Y_n^0(\gamma, \psi) \, Y^*_{n,m}(\theta_1, \phi_1).
\label{eq:Am}
\end{equation}



\subsection{The inverse expansion}



In the above expression of $A_m$,
$Y_n^0(\gamma, \psi)$ and $Y_n^{m*}(\theta_1, \phi_1)$
are almost symmetric (except a complex conjugation).
%
Thus $A_m$
can be understood as the $m'=0$th expansion coefficients
of $Y_n^{m*}(\theta_1, \phi_1)$
in terms of $Y_n^{m'*}(\gamma, \psi)$.
%
Consider the inverse expansion
\begin{equation}
  Y_n^{m*}(\theta_1, \phi_1)
=
  \sum_{m' = -n}^n b_{m'} \,Y_n^{m'*}(\gamma, \psi),
  \label{eq:expansion2}
\end{equation}
in which,
the coefficient $b_{m'}$ can be computed as
\begin{equation}
  b_{m'}
=
  \int d \Omega_{\gamma,\psi} \,
  Y_n^{m*}(\theta_1, \phi_1) \,
  Y_n^{m'}(\gamma, \psi),
  \label{eq:bm}
\end{equation}
where, $d\Omega_{\gamma, \psi} \equiv d(-\cos\gamma) d\psi$,
and the integration over $d\Omega_{\gamma, \psi}$
and that over $d\Omega_1$.
Comparing it to \eqref{eq:Am}, we get
\begin{equation}
  b_0 = A_m.
  \label{eq:b0Am}
\end{equation}



\subsection{Determination of $b_0$}


Now our job is to determine $b_0$.
The trick is \emph{not} to do it from Eq. \eqref{eq:bm},
but to deduce it from Eq. \eqref{eq:expansion2}
from a pair of special values of $\theta_1$ and $\phi_1$
that will \emph{single out} the $m' = 0$ term.

Let us set $\theta_1 = \theta_2$ and $\phi_1 = \phi_2$.
Then the two points 1 and 2 coincide,
and $\gamma = 0$ (with $\psi$ being arbitrary).
%
We show below that only the $m' = 0$ term in the sum
of Eq. \eqref{eq:expansion2} survives.
%
This is because
\begin{equation}
  Y_n^{m'}(0, \psi)
=
  (-1)^{m'}
  \sqrt{
    \frac{ 2 n + 1 } { 4 \pi }
    \frac{ (n - m')! } { (n + m')! }
  }
  \,
  P_{n}^{m'}\bigl( \cos(0) = 1 \bigr) e^{i m' \psi},
  \label{eq:Ynm0}
\end{equation}
and the associated Legendre polynomial
\begin{equation}
  P_{n}^{m'}(x)
\equiv
  \frac{ (1 - x^2)^{m'/2} } { 2^{n} n! }
  \frac{ d^{n+m'} } { d x^{n+m'} } (x^2 - 1)^{n}.
  \label{eq:Pnm}
\end{equation}
vanishes at $x = 1$, unless $m' = 0$.
Thus,
\[
Y_n^{m*}(\theta_2, \phi_2)
=
b_0 \, Y_n^{0*}(0, \psi).
\]

The final step is to determine $Y_n^0(0, \psi)$.
From Eq. \eqref{eq:Ynm0}, we get
\[
  Y_n^{0*}(0, \psi)
=
  Y_n^0(0, \psi)
=
  \sqrt{
    \frac { 2 n + 1 } { 4 \pi }
  }
  P_{n}^0(1),
\]
and
from Eq. \eqref{eq:Pnm}, we get
\[
  P_{n}^{0}(1)
=
  \left.
  \frac{ 1 } { 2^{n} n! }
  \frac{ d^{n} } { d x^{n} }
  \Bigl[
    (x - 1)^{n}
    (x + 1)^{n}
  \Bigr]
  \right|_{x = 1}\
= \frac{ 1 } { 2^{n} n! }
  n! \, (1 + 1)^n
= 1.
\]
In the second step,
we have used the fact that
the $n$ differentiations must be
all applied to the factor $(x - 1)^n$
to get a non-vanishing final result at $x = 1$.
%
Thus,
$Y_n^0 = \sqrt{4\pi/(2n+1)}$.
and
\begin{equation}
b_0 = \sqrt{ \frac {4 \pi} { 2 n + 1 } } \, Y_n^{m*}(\theta_2, \phi_2).
\label{eq:b0}
\end{equation}



\subsection{Finishing up}



Using Eq. \eqref{eq:b0} in Eqs. \eqref{eq:b0Am} and \eqref{eq:am2},
we get
\begin{equation}
  a_m = \frac{ 4 \pi } { 2 n + 1 }.
\end{equation}
Using this in Eq. \eqref{eq:expansion1}
proves the addition theorem Eq. \eqref{eq:addition}.




\subsection{An analogy}



The centerpiece of the above proof is the shift of coordination system
in the inverse expansion Eq. \eqref{eq:expansion2}.
This step may appear convoluted.
Below we give a simpler analogy
that can hopefully clear up the logic.

Suppose we wish to prove the trigonometry identity:
\begin{equation}
  \cos(\gamma - \theta)
= \cos\gamma \cos\theta
+ \sin\gamma \sin\theta.
\label{eq:cosab}
\end{equation}
One way to do so is to think of it as a Fourier series
\begin{equation}
  \cos(\gamma - \theta)
=
  a \, \cos\theta
+
  a' \, \sin\theta.
  \label{eq:Fourier1}
\end{equation}
where, the expansion coefficients $a$ and $a'$
are to be determined from
\begin{align}
a &= \frac{1}{\pi} \int_0^{2\pi}
  \cos(\gamma - \theta) \, \cos\theta \, d\theta,
  \label{eq:c1}
\\
a' &= \frac{1}{\pi} \int_0^{2\pi}
  \cos(\gamma - \theta) \, \sin\theta \, d\theta.
  \label{eq:s1}
\end{align}
Eq. \eqref{eq:Fourier1} is analogous to Eq. \eqref{eq:expansion1},
and Eqs. \eqref{eq:c1} and \eqref{eq:s1}
are analogous to Eq. \eqref{eq:am1}.


Now consider a Fourier series based on $\cos(\gamma - \theta)$
and $\sin(\gamma - \theta)$.
Then the inverse expansion of $\cos\theta$
in this new basis set, analogous to Eq. \eqref{eq:expansion2}, is
\begin{equation}
  \cos\theta
  =
  b \cos(\gamma - \theta)
+
  b' \sin(\gamma - \theta).
  \label{eq:Fourier2}
\end{equation}
where,
the new coefficients are computed from,
in analogous to Eq. \eqref{eq:bm},
\begin{align}
b &= \frac{1}{\pi} \int_0^{2\pi}
  \cos\theta \, \cos(\gamma - \theta) \, \cos\theta \, d(\gamma - \theta),
  \label{eq:c2}
\\
b' &= \frac{1}{\pi} \int_0^{2\pi}
  \cos\theta \, \sin(\gamma - \theta) \, d(\gamma - \theta).
  \label{eq:s2}
\end{align}
Clearly, $b = a$ [corresponding to Eq. \eqref{eq:b0Am}].

To complete the analogy,
consider the evaluation of $b = a$
by using special values in
Eq. \eqref{eq:Fourier2}.
%
With $\theta = \gamma$,
we can single out the $b$ term
as $\sin(\gamma - \theta) = 0$
in this case.
%
So
\begin{equation}
  \cos\gamma
=
  b \cos(\gamma - \gamma)
= b.
\end{equation}
So $a = b = \cos\gamma$.
%
Similarly one can show that $a' = \sin\gamma$,
which yields Eq. \eqref{eq:cosab}.



\section{\label{sec:rayleigh}Rayleigh equation}



The Rayleigh equation expands a plane wave $e^{i k x}$into spherical waves $j_n(kr)$
\begin{align}
  e^{i\vk \cdot \vr}
=
  e^{ikr\cos\gamma}
&=
  \sum_{n = 0}^\infty
i^n \, ( 2 n + 1 )
j_n(k r) \, P_n(\cos\gamma)
\label{eq:rayleigh} \\
&=
  \sum_{n = 0}^\infty
i^n \, 4 \, \pi
j_n(k r) \, Y_n^m(\Omega_{\vk}) \, Y_n^{m*}(\Omega_{\vr}),
\label{eq:rayleigh_yy}
\end{align}
where, $\gamma$ is angle between the vector $k$ and $r$,
and we have used the addition theorem Eq. \eqref{eq:addition}
on the second line.



\subsection{\label{sec:rayleigh_proof}Proof}



To show this, we need the following lemma.
%
We wish to evaluate the integral
$\int_{-1}^1 x^m \, P_n(x) \, dx$.
%
For $m = n + 2 s$, we can use
Rodrigues' formula, Eq. \eqref{eq:legendre_rodrigues},
and integrate by parts for $n$ times:
\begin{align*}
\int_{-1}^1 x^{n + 2s} \, P_n(x) \, dx
&=
\frac{ 1 } { 2^n n! }
\int_{-1}^1 x^{n + 2 s} \,
\frac{d^n}{dx^n} (x^2 - 1)^n \, dx
\notag \\
&=
\frac{ (-)^n } { 2^n n! }
\int_{-1}^1 \frac{d^n}{dx^n} x^{n + 2 s} \,
(x^2 - 1)^n \, dx
\notag \\
&=
\frac{ (-)^n (n + 2 s)! } { 2^n n! (2 s)! }
\int_{-1}^1 x^{2 s} \,
(x^2 - 1)^n \, dx
\notag \\
&=
\frac{ (n + 2 s)! } { 2^n n! (2 s)! }
\int_{-1}^1 \cos^{2 s}\theta \,
\sin^{2n + 1}\theta \, d\theta
\notag \\
&=
\frac{ (n + 2 s)! } { 2^n n! (2 s)! }
B\left(s + \frac{1}{2}, n + 1\right)
=
\frac{ 2^{n+1} \, (n + 2 s)! \, (n + s)! } { (2 n + 2 s + 1)! \, s! },
\notag \\
\end{align*}
where,
on the fourth line,
we have changed variable $x = -\cos\theta$;
%
$B(x, y) = \Gamma(x) \, \Gamma(y) / \Gamma(x + y)$
is the beta function
[with $\Gamma(x) = (x-1)!$ being the gamma function].
%
An integral for $x^m$ with $m < n$
or for an $m$ that differs in parity with $n$
yields zero.


Consider the expansion,
\[
  e^{iax} = \sum_{m = 0}^\infty c_m P_m(x).
\]
Multiplying both sides by $P_n(x)$ and integrate from $-1$ to $1$,
then for the left-hand side, we have
\begin{align}
  \int_{-1}^1
  e^{iax} P_n(x) \, dx
&=
  \sum_{s = 0}^\infty
  \frac{ (ia)^{n + 2s} } { (n + 2s)! }
  \int_{-1}^1
  x^{n + 2 s} \, P_n(x) \, dx
  \notag \\
&=
  2^{n+1} \, i^n \,
  \sum_{s = 0}^\infty
  \frac{ a^{n + 2s} \, (n + s)! }
  { (2n + 2s + 1)! \, s! }
= 2 \, i^n j_n(a),
\label{eq:PnFT}
\end{align}
where we have used Eq. \eqref{eq:besselj}.
in the last step.
For the right-hand side, we have
\begin{align*}
  \int_{-1}^1 P_n(x) \, \sum_{m = 0}^\infty c_m \, P_m(x) \, dx
&=
  \frac{ 2 \, c_n } { 2 n + 1}.
\end{align*}
%
Thus
\[
  c_n = i^n \, (2 n + 1) \, j_n(a).
\]
and
\[
  e^{iax} = \sum_{n = 0}^\infty i^n \, (2 n + 1) \, P_n(x).
\]
Now with $a = kr$ and $x = \cos\gamma$, we get Eq. \eqref{eq:rayleigh}.


Note that Eq. \eqref{eq:PnFT} is interesting itself.
It shows that the Fourier transform of the Legendre polynomial
is the spherical Bessel function.



\appendix



\section{\label{sec:spcoords_laplacian}
Laplacian in the spherical polar coordinates}



The formula for the Laplacian $\nabla^2 = \nabla \cdot \nabla$
in the spherical polar coordinates
can be a bit difficult to remember.
%
We start with the identity from partial integration:
\begin{equation}
  \int \nabla \chi \cdot \nabla \psi \, d\vr
=
  -\int \chi \, \nabla^2 \psi \, d\vr.
\label{eq:partint}
\end{equation}
In spherical polar coordinates,
the gradient $\nabla$ is given by
\begin{equation}
  \nabla
=
  \hat \vr \, \frac{\partial} {\partial r}
+
  \hat{ \vct \uptheta } \,
  \frac{1} {r}
  \frac{\partial} {\partial \theta}
+
  \hat{ \vct \upphi } \,
  \frac{1} {r \, \sin \theta}
  \frac{\partial} {\partial \phi}
  \label{eq:spcoords_grad}
\end{equation}
and the volume element is given by
$d\vr = r^2 \sin \theta \, dr \, d\theta \, d\phi$.

Now the left hand side of Eq. \eqref{eq:partint}
is equal to
\begin{align*}
  \int \nabla \chi \cdot \nabla \psi \, d\vr
&=
  \int
  \left(
  \frac{\partial \chi}{\partial r}
  \frac{\partial \psi}{\partial r}
  +
  \frac{1}{r^2}
  \frac{\partial \chi}{\partial \theta}
  \frac{\partial \psi}{\partial \theta}
  +
  \frac{1}{r^2 \, \sin^2 \theta}
  \frac{\partial \chi}{\partial \phi}
  \frac{\partial \psi}{\partial \phi}
  \right)
  \, r^2 \, \sin\theta \, dr \, d\theta \, d\phi
\notag \\
&=
  -\int
  \chi \,
  \left[
    \frac{\partial}{\partial r}
    \left(
      r^2 \, \sin\theta \,
      \frac{\partial \psi}{\partial r}
    \right)
    +
    \frac{\partial}{\partial \theta}
    \left(
      \sin\theta \,
      \frac{\partial \psi}{\partial \theta}
    \right)
    +
    \frac{\partial}{\partial \phi}
    \left(
      \frac{1}{\sin\theta}
      \frac{\partial \psi}{\partial \phi}
    \right)
  \right]
  \, dr \, d\theta \, d\phi
\notag \\
&=
  -\int
  \chi \,
  \left[
    \frac{1}{r^2}
    \frac{\partial}{\partial r}
    \left(
      r^2 \,
      \frac{\partial \psi}{\partial r}
    \right)
    +
    \frac{1}{r^2 \, \sin\theta}
    \frac{\partial}{\partial \theta}
    \left(
      \sin\theta \,
      \frac{\partial \psi}{\partial \theta}
    \right)
    +
    \frac{1}{r^2 \, \sin^2\theta}
    \frac{\partial^2 \psi}{\partial \phi^2}
  \right]
  \, d\vr,
\end{align*}
where we have integrated by parts on the second line.

Compare to the right hand side of Eq. \eqref{eq:partint},
we find that
\begin{equation}
  \nabla^2\psi
=
  \frac{1}{r^2}
  \frac{\partial}{\partial r}
  \left(
    r^2 \,
    \frac{\partial \psi}{\partial r}
  \right)
  +
  \frac{1}{r^2 \, \sin\theta}
  \frac{\partial}{\partial \theta}
  \left(
    \sin\theta \,
    \frac{\partial \psi}{\partial \theta}
  \right)
  +
  \frac{1}{r^2 \, \sin^2\theta}
  \frac{\partial^2 \psi}{\partial \phi^2}.
  \label{eq:spcoords_laplacian}
\end{equation}



\bibliography{../liquid}
\bibliographystyle{alpha}
\end{document}

\documentclass[12pt]{article}
\usepackage{amsmath}
\begin{document}



\newcommand{\vct}[1]{\mathbf{#1}}
\newcommand{\vr}{\vct{r}}
\newcommand{\vrN}{\mathbf{r}^N}
\newcommand{\vrn}{\mathbf{r}^n}
\newcommand{\vk}{\vct{k}}
\newcommand{\dvk}{\frac{ d \vk  }{(2\pi)^3}}
% Fourier transform
\newcommand{\FT}[1]{\tilde{#1}}
\newcommand{\FTc}{\FT{c}}
\newcommand{\FTh}{\FT{h}}

\newcommand{\plam}{\partial_\lambda}
\newcommand{\pbet}{\partial_\beta}



\title{Singer-Chandler formula for the chemical potential}
\date{}
\maketitle



\section{Derivation}

Here we give a detailed derivation of the Singer-Chandler formula
  for the chemical potential \cite{singer}.
%
For simplicity, we assume a simple classical liquid,
  with a single solute molecule, denoted by the subscript $u$,
  and a one-atom solvent denoted by the subscript $v$ below.
%
If we further assume the hypernetted-chain (HNC) approximation,
then, the chemical potential of adding the solute into
  the system is given by
%
\begin{equation}
\beta \mu_u
  =
  \rho_v \int d\vr
  \left[
    - c_{uv}(\vr)
    + \frac{1}{2} h_{uv}^2(\vr)
    - \frac{1}{2} h_{uv}(\vr) \, c_{uv}(\vr)
  \right],
  \label{eq:singer}
\end{equation}
%
where $\rho_v$ is the solvent density,
  $h_{uv}(\vr) = g_{uv}(\vr) - 1$ is the total correlation function
  [with $g_{uv}(\vr)$ being the radial distribution function],
and $c_{uv}(\vr)$ is the direct correlation function.
We refer the reader to Ref. \cite{hansen}
  for the basic concepts.
%



\subsection{Thermodynamic integration}

We start with the thermodynamic integration formula.
If the potential energy
\begin{equation}
  U(\vrN, \lambda) = U_0(\vrN) + \lambda U_1(\vrN)
  \label{eq:potene}
\end{equation}
depends on the parameter $\lambda$,
then the partition function
%
\begin{equation}
  Z(\lambda) = \int d\vrN \exp[-\beta U(\vrN, \lambda)].
  \label{eq:partfunc}
\end{equation}
%
satisfies
\begin{align}
  \frac{ d Z(\lambda) / d\lambda }{Z(\lambda)}
  &=
    -\beta \int d\vrN U_1(\vrN)
    \left\{
      \frac{ \exp
             \left[
             -\beta U(\vrN, \lambda)
             \right]}
      { Z(\lambda) }
    \right\} \notag \\
  &=
    -\beta
    \left\langle
      U_1(\vrN)
    \right\rangle_\lambda.
\end{align}
In the second step, note that the term in the braces
  is the normalized Boltzmann weight,
  and the integral is equal to an ensemble average
  $\langle \dots \rangle_\lambda$.
%
Equivalently, in terms of the free energy $\beta F = -\log Z$,
\begin{equation}
  \beta d F(\lambda) / d\lambda
  = \beta
    \left\langle
      U_1(\vrN)
    \right\rangle_\lambda.
\end{equation}

Let the first atom be the solute $u = 0$,
  the next $n = N - 1$ atoms be the solvent atoms.
Also,
  let $U_0(\vrn)$ be the potential energy of the $n$ solvent atoms
  and $U_1(\vrN)$ be the interaction between the solute atom $u$
  and the solvent atoms $v = 1, 2, \dots, n$.
By changing $\lambda$ from 0 to 1, we gradually turn on
  the solute-solvent interaction.
%
So the chemical potential $\mu_u = dF/dN$ is
\begin{align}
  \mu_u
  =
  F(1) - F(0)
  =
  \int_0^1 \frac{d F(\lambda)}{d \lambda} \, d\lambda
  =
  \int_0^1
    \left\langle
      U_1(\vrN)
    \right\rangle_\lambda d\lambda.
  \label{eq:mu_ti0}
\end{align}




In a simple liquid with pairwise interactions,
  $U_1(\vrN) = \sum_{i = 1}^{n} \phi(\vr_{0i})$,
  where $\phi(\vr_{0i})$ means the pair potential
  between the solute atom $0$ and a solute atom $i$.
So
\begin{equation}
    \left\langle
      U_1(\vrN)
    \right\rangle_\lambda
    =
    n
    \left\langle
      \phi(\vr_{01})
    \right\rangle_\lambda
    =
    \rho_v \int d\vr_{01} \,
      \phi(\vr_{01}) \, g_{uv}(\vr_{01}, \lambda),
  \label{eq:uphig}
\end{equation}
%
where in the second step, we have evaluated the average
pairwise energy in terms of the radial distribution function
$g_{uv}(\vr, \lambda)$.
Using \eqref{eq:uphig} in \eqref{eq:mu_ti0} yields
\begin{equation}
    \mu_u
    =
    \rho_v
    \int_0^1 d\lambda \,
      \int d\vr_{01} \,
        \phi(\vr_{01}) \, g_{uv}(\vr_{01}, \lambda).
  \label{eq:mu_ti}
\end{equation}
Our aim below is to evaluate
$\phi(\vr) \, g(\vr, \lambda)$
by means of the integral equation.
%
Below, we will show that the outer integral of $\lambda$
  can be avoided by a clever algebraic manipulation,
  if we assume the hypernetted-chain approximation.



\subsection{\label{sec:funcs}Integral equation and the HNC closure}

First, we define the total correlation function $h(\vr)$ as
\begin{equation}
  h(\vr) \equiv g(\vr) - 1.
  \label{eq:hr}
\end{equation}
%
%
%
Next, the cavity function $y_{uv}(\vr)$ for a solute-solvent pair
\begin{equation}
  y_{uv}(\vr) \equiv e^{ \beta \lambda \phi(\vr) } g_{uv}(\vr).
  \label{eq:yr}
\end{equation}
%
%
%
The direct correlation function $c(\vr)$
is defined in the Ornstein-Zernike relation:
\begin{equation}
  h(\vr) = c(\vr) + \rho \, [c * h](\vr),
  \label{eq:oz}
\end{equation}
where $[c*h](\vr) \equiv \int c(\vr') \, h(\vr - \vr') \, d\vr'$
  (``$*$'' means a convolution).
We will attach the subscripts ``$u$'' (solute) and ``$v$'' (solvent) later.
%
%
%
%Finally, we define
%\begin{equation}
%  t(\vr) \equiv h(\vr) - c(\vr).
%  \label{eq:tr}
%\end{equation}
%
%
%
Finally, $y(\vr)$ is approximated in the hypernetted-chain closure as
\begin{equation}
  y(\vr) \approx \exp\left[ h(\vr) - c(\vr) \right].
  \label{eq:hnc}
\end{equation}
Since we have four variables
  $g(\vr)$, $h(\vr)$, $c(\vr)$, and $y(\vr)$,
the above four equations permit a solution of $g(\vr)$.
%
%
%
The functions, such as $g(\vr)$, depend on $\lambda$,
although this dependence is not written explicitly for simplicity.



\subsection{Internal energy as a total differential (I)}


We now try to simplify the right-hand side of \eqref{eq:mu_ti}, such that
the inner integral, which represents the internal energy, can be written
as a total differential of $\lambda$.
%
In this way the outer integral can be removed.

Differentiating \eqref{eq:yr}, we have [dropping ``$(\vr)$'' below whenever convenient]
\[
  \beta \phi \, e^{ \beta \lambda \phi } \, g_{uv}
  =
  \plam y_{uv}
    -
    e^{ \beta \lambda \phi } \, \plam g_{uv}.
\]
Then by multiplying $e^{-\beta \lambda \phi}$ to both sides,
and using \eqref{eq:hr}, we get
\begin{equation}
  \beta \phi \, g_{uv}
  =
  e^{ -\beta \lambda \phi } \, \plam y_{uv}
  - \, \plam h_{uv}.
  \label{eq:phiguv}
\end{equation}
%
Next we use the HNC approximation \eqref{eq:hnc}, such that
%
\begin{align*}
  e^{ -\beta \lambda \phi} \plam y_{uv}
  = e^{ -\beta \lambda \phi} y_{uv} \, \plam (h_{uv} - c_{uv})
  = (h_{uv} + 1) \, \plam (h_{uv} - c_{uv})
\end{align*}
%
Using this in \eqref{eq:phiguv} yields
\begin{equation}
  \beta \phi(\vr) \, g_{uv}(\vr)
  =
  - \plam c_{uv}(\vr) + h_{uv}(\vr) \, \plam h_{uv}(\vr) - h_{uv}(\vr) \, \plam c_{uv}(\vr).
  \label{eq:hc1}
\end{equation}



\subsection{Internal energy as a total differential (II)}



We want to write the right-hand side \eqref{eq:hc1} as
a total differential $\plam(\dots)$.
%
Obviously $h_{uv} \plam h_{uv} = \plam ( h_{uv}^2/2 )$.
%
We argue below that $h_{uv} \plam c_{uv} = \plam ( h_{uv} c_{uv} /2) $
in an \emph{average} sense, meaning after the integration $\int d\vr$,
under the condition of \emph{infinite dilution}.
Let us look at \eqref{eq:oz} more carefully.
%
For the solute-solvent pair, we have
\[
  h_{uv} = c_{uv} + \rho_v \, c_{uv}*h_{vv}
  + \rho_u \, c_{uu}*h_{uv},
\]
where $\rho_v$ and $\rho_u$ means the density of
solvent and solute, respectively.
%
Since we have a single solute and many $n \gg 1$ solvent atoms,
we can safely set $\rho_u = 0$:
\begin{equation}
  h_{uv} = c_{uv} + \rho_v \, c_{uv}*h_{vv}.
  \label{eq:ozuv}
\end{equation}
%
Similarly, for the solvent-solvent version
\begin{equation}
  h_{vv} = c_{vv} + \rho_v \, c_{vv}*h_{vv}.
  \label{eq:ozvv}
\end{equation}
From \eqref{eq:ozvv}, we know that
  $h_{vv}$ and $c_{vv}$ are independent of the solute;
and from \eqref{eq:ozuv}, we know that
  $h_{uv}$ is a linear function with respect to $c_{uv}$.
%
Thus,
\begin{align}
  \int d\vr \, h_{uv}(\vr) \, \plam c_{uv}(\vr)
  &=
  \int d\vr \,
  c_{uv}(\vr) \, \plam c_{uv}(\vr) \notag \\
  & \;\;\;\;\;
    +
    \iint d\vr \, d\vr'
    \, \rho_v
    \, c_{uv}(\vr')
    \, h_{vv}(\vr - \vr')
    \, \plam c_{uv}(\vr) \notag \\
  &= \int d\vr \, c_{uv}(\vr) \, \plam h_{uv}(\vr) \notag \\
  &= \int d\vr \plam
    \left[
      \tfrac{1}{2} h_{uv}(\vr) \,  c_{uv}(\vr)
    \right].
  \label{eq:phig}
\end{align}
%
By using \eqref{eq:phig}, \eqref{eq:hc1} in \eqref{eq:uphig}, we get
%
\begin{align}
  - \langle
  \beta U_1(\vr)
  \rangle_\lambda
  &=
  - \rho_v \,
  \int d \vr
  \, \beta \phi(\vr) \, g_{uv}(\vr) \notag \\
  &=
   \rho_v \, \plam \int d \vr \,
   \left[
    c_{uv}(\vr)
    - \tfrac{1}{2} h_{uv}^2(\vr)
    + \tfrac{1}{2} h_{uv}(\vr) \, c_{uv}(\vr)
    \right].
\end{align}
%
%
%
By \eqref{eq:mu_ti}, we get \eqref{eq:singer}, or more explicitly,
\begin{align}
  -\beta \mu_u
  &=
   \rho_v \, \int
    d \vr \,
   \left[
     c_{uv}(\vr)
    - \tfrac{1}{2} h_{uv}^2(\vr)
    \right]
  \notag \\
  & \;\;\;\;\;
    +
    \rho_v \, \iint d\vr \, d\vr'
    \, c_{uv}(\vr)
    \, \chi_{vv}(\vr - \vr')
    \, c_{uv}(\vr'),
 \label{eq:muexpanded}
\end{align}
where
$\chi_{vv}(\vr) = \delta_{vv}(r) + \rho_v \, h_{vv}(\vr)$.
%
This is the equivalent to (2.8) in \cite{singer}
[note,
(2.8) misses the $\omega_{\alpha\gamma}$ (for the intra-solute interaction)
in the last term].




\subsection{Kovalenko-Hirata (KH) closure}

Note that the Singer-Chandler formula is valid
  only under the HNC closure.
Under a general closure (like the Percus-Yevick one),
  we may not have a compact expression for $\mu_u$.
Fortunately, under the Kovalenko-Hirata (KH) closure,
\begin{equation}
  g_{uv}(\vr) =
  \begin{cases}
    1 + d_{uv}(\vr)
    & d_{uv}(\vr) > 0
  \\
    \exp[ d_{uv}(\vr) ]
    & d_{uv}(\vr) \le 0,
  \end{cases}
  \label{eq:kh}
\end{equation}
where $d_{uv}(\vr) = -\beta \lambda \phi(\vr) + h_{uv}(\vr) - c_{uv}(\vr)$,
we do have a counterpart of \eqref{eq:singer}.


First, in \eqref{eq:kh}, the $d_{uv} \le 0$ case is equivalent to the HNC case.
%
If $d_{uv} > 0$, we have $c_{uv}(\vr) = -\beta \lambda \phi(\vr)$, and
\[
  \plam c_{uv}(\vr) = -\beta \phi(\vr).
\]
Then
\[
  -\beta \phi \, g_{uv}
  = \plam c_{uv} \, (1 + h_{uv}),
\]
which differs from the HNC case [cf. \eqref{eq:hc1}]
  only by the $h_{uv}^2/2$ term.
Thus
%
\begin{equation}
-\beta \mu_u
  =
  \rho_v \int d\vr
  \left\{
    c_{uv}(\vr)
    - \frac{1}{2} h_{uv}^2(\vr) \, \Theta[ -d_{uv}(\vr) ]
    + \frac{1}{2} h_{uv}(\vr) \, c_{uv}(\vr)
  \right\},
  \label{eq:singer-kh}
\end{equation}
%
where $\Theta(x)$ is the unit step function,
which is 1.0 if $x \ge 0$ or 0 otherwise.
%
Finally, if we drop the $h_{uv}^2/2$ term entirely,
it becomes the Gaussian density field theory mentioned in \cite{singer}.


%\clearpage

\section{Extension to the Helmholtz free energy}

The same technique can be used to derive the Helmholtz free energy $F$ (or $A$).
%
Below we assume the HNC closure.
%
In this case, we switch from $\lambda$ to $\beta$,
and the partition function
%
\begin{equation}
  Z(\beta) = \int d\vrN \exp[-\beta U(\vrN)].
  \label{eq:partfunc_beta}
\end{equation}
%
satisfies
\begin{align*}
  \frac{ d \left[ \beta F(\beta) \right] }
  { d \beta }
  = -\frac{ d Z(\beta) / d \beta }{Z(\beta)}
  = \left\langle
      U(\vrN)
    \right\rangle_\beta.
\end{align*}
%
So, to obtain $F(\beta)$, we only need to integrate
  the average potential energy
  $\left\langle U(\vrN) \right\rangle_\beta$
  over $\beta$.
%
The rest derivation is very similar to that in Sec. \ref{sec:funcs}.
%
Note that
%
\begin{equation}
  \left\langle
    U(\vrN)
  \right\rangle_\beta
  =
  \frac{\rho^2 V}{2} \int d \vr \, \phi(\vr) \, g(\vr),
\end{equation}
%
where, the factor $1/2$ is due to the double counting of pairs.
%
The counterpart of \eqref{eq:hc1} is
%
\begin{equation}
  \phi(\vr) \, g(\vr)
  =
  - \pbet c(\vr) + h(\vr) \, \pbet h(\vr) - h(\vr) \, \pbet c(\vr).
  \label{eq:hcall}
\end{equation}
%
The key difference here is that we are switching on the interaction
  of \emph{all} pairs of particles, whereas previously, we only switch on
  the interaction with a single solute $u$.
%
We have dropped the subscripts ``$u$'' and ``$v$''
  for there is no difference between the solute and solvents.


Now the last term $h(\vr) \, \pbet c(\vr)$
  is no longer $\pbet \big[ \frac{1}{2} h(\vr) c(\vr) \big]$,
  and it is handled as follows.
%
We will obtain a closed expression under the assumption
  that $c(\vr) = c(-\vr)$.
%
First, from the $\vk$-space version of \eqref{eq:oz}
\begin{align*}
  \FTh(\vk)
  &= \FTc(\vk) + \rho \, \FTc(\vk) \, \FTh(\vk) \\
  &= \FTc(\vk) + \rho \, \FTc^2(\vk)
               + \rho^2 \, \FTc^3(\vk)
               + \dots
\end{align*}
%
where $\FTc(\vk) = \int d\vr \, c(\vr) \, e^{-i \vk\cdot\vr}$.
%
Let's now define a similar function,
%
\begin{align*}
  \FT{H}(\vk)
  &= \FTc(\vk)/2
   + \rho \FTc^2(\vk)/3
   + \rho^2 \FTc^3(\vk)/4
    + \dots \\
  &= -\left\{
        \rho \FTc(\vk)
        + \log[1 - \rho \FTc(\vk)]
    \right\} / [\rho^2 \FTc(\vk)],
\end{align*}
%
and we will show below that
\begin{equation}
  \pbet \int H(\vr) c(\vr) \, d \vr
  = \int h(\vr) \, \pbet c(\vr) \, d \vr.
  \label{eq:bigH}
\end{equation}




Consider the $\rho \, \FTc^2(\vk)$ term,
  which is the Fourier transform of $\rho \, (c*c)(\vr)$.
%
We have
\begin{align*}
&  \int \dvk \,
  \rho \, \FTc^2(\vk) \, \FTc(-\vk) \\
=&
  \int \dvk \,
    \rho \,
    \int c(\vr')  e^{ i \vk\cdot \vr'}   d \vr'
    \int c(\vr'') e^{ i \vk\cdot \vr''}  d \vr''
    \int c(\vr)   e^{-i \vk\cdot \vr}    d \vr \\
=&
    \iiint
    \rho \,
    c(\vr') c(\vr'') c(\vr)
    \, \delta(\vr' + \vr'' - \vr) \,
    d \vr'
    d \vr''
    d \vr \\
=&
    \iint
    \rho \,
    c(\vr')  c(\vr - \vr') c(\vr) \,
    d \vr'
    d \vr \\
=&  \int \rho \, (c*c)(\vr) \, c(\vr) \,
    d \vr,
\end{align*}
where we have used the identity
$\int e^{i \vk\cdot \vr} d \vr = (2\pi)^3 \delta(\vk)$.



Generally, we have
%
\begin{align*}
  & \int \dvk
    \rho^{n-1} \FTc^n(\vk)
     \FTc(-\vk)
  = \int d\vr \,
  \rho^{n-1} c^{*n}(\vr) \, c(\vr),
\end{align*}
where
%
\[
  c^{*n}(\vr) \equiv [\underbrace{c*\dots*c}_n](\vr).
\]
%
If the function $c(\vr)$ is even, then so is $\FTc(\vk)$.
We assume this case below.
Then $\FTc^n(\vk) \, \FTc(-\vk) = \FTc^{n+1}(\vk)$,
  and
\[
  \pbet \big[ \FTc^n(\vk) \, \FTc(-\vk) \big]
  = (n+1) \FTc^n(\vk) \, \pbet \FTc(\vk)
  = (n+1) \FTc^n(\vk) \, \pbet \FTc(-\vk)
\]
So
%
\begin{align*}
  \pbet  \int \dvk
  \frac{\rho^{n-1} \FTc^n(\vk) }{n+1} \FTc(-\vk)
  &= \int \dvk
  \rho^{n-1} \FTc^n(\vk) \, \pbet \FTc(-\vk) \\
  &=  \int d \vr \,
  \rho^{n-1} \, c^{*n}(\vr) \, \pbet c(\vr).
\end{align*}
%
Summing over $n$ yields
%
\begin{align*}
  \pbet  \int \dvk
  \FT{H}(\vk) \, \FTc(-\vk)
  &= \int \sum_{n} \rho^{n-1} \, c^{*n}(\vr) \, \pbet c(\vr) \,
    d \vr \\
  &= \int h(\vr) \, \pbet c(\vr) \,
    d \vr.
\end{align*}
%
%
%
Thus
%
\begin{align}
-\frac{\beta F}{V}
&=
  \frac{\rho^2}{2}
  \int d\vr
  \left[
    c(\vr)
    - \tfrac{1}{2} h^2(\vr)
    + H(\vr) \, c(\vr)
  \right] \notag \\
&=
  \frac{1}{2}
  \int d\vr \rho^2
  \left[
     c(\vr)
    - \tfrac{1}{2} h^2(\vr)
  \right]
  -
  \frac{1}{2}
  \int \dvk
    \Big\{
      \rho \FTc(\vk)
      + \log\big[ 1 - \rho \FTc(\vk)  \big]
    \Big\} \notag \\
&=
  \frac{1}{2}
  \left\{
  \int d\vr \rho^2
  \left[
     c(\vr)
    - \tfrac{1}{2} h^2(\vr)
  \right]
  +
  \int \dvk
    \sum_{n = 2}^{\infty}
      \frac{  \rho^n \FTc^n(\vk) }{n}
  \right\},
\label{eq:freeEnergy}
\end{align}
%
where we have used \eqref{eq:bigH} in the second step.
%
This is a simplified special case of (3.1) in the original paper.
%
In terms of a real space series, we have
%
\begin{align}
-\frac{\beta F}{V}
&=
  \frac{1}{2}
  \int d\vr
  \left\{
    \rho^2
    \left[
      c(\vr)
      - \tfrac{1}{2} h^2(\vr)
    \right]
  +
  \sum_{n = 2}^{\infty}
  \frac{\rho^{n}}{n} \, c^{*(n-1)}(\vr) \, c(\vr)
  \right\}.
\end{align}
%

\emph{Note}, for the KH closure, we simply attach a $\Theta[-d(\vr)]$ before $\tfrac{1}{2}h^2(\vr)$.
%
We may further expand $h(\vr) = c(\vr) + t(\vr)$,
  and rewrite the formula as
%
\begin{align}
-\frac{\beta F}{V}
&=
  \frac{1}{2}
  \left\{
  \int d\vr \rho^2
  \left[
     c(\vr)
    - \tfrac{1}{2} t^2(\vr)
  \right]
  +
  \int \dvk
    \sum_{n = 3}^{\infty}
      \frac{1-n}{n} \rho^n \FTc^n(\vk)
  \right\} \notag \\
&=
  \frac{1}{2}
  \int d\vr
  \left\{
  \rho^2
  \left[
     c(\vr)
    - \tfrac{1}{2} t^2(\vr)
  \right]
  +
  \sum_{n = 3}^{\infty}
  \frac{1-n}{n} \rho^{n} \, c^{*(n-1)}(\vr) \, c(\vr)
  \right\}.
\label{eq:freeEnergy_t2}
\end{align}
%



\section*{Acknowledgment}
It is a pleasure to thank C.-L. Lai for helpful discussions.


\begin{thebibliography}{100}

\bibitem{singer}
  Sherwin J. Singer and David Chandler,
  ``Free energy functions in the extended RISM approximation,''
  Molecular Physics, Vol. 55, No. 3, 621-625,
  1985.

\bibitem{hansen}
  J.-P. Hansen and I.R. McDonald,
  {\it Theory of Simple Liquids}, 3rd ed.,
  Academic Press, London, UK 2006.

\end{thebibliography}

\end{document}


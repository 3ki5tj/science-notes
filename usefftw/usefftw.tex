\documentclass[12pt]{article}

\usepackage{amsmath}
\usepackage{hyperref}
%\usepackage{rotating}
%\usepackage{booktabs}
\newcommand{\vct}[1]{\mathbf{#1}}

\begin{document}

\title{Using FFTW3}
\date{}
\maketitle





\tableofcontents




\section{Getting started}

The FFTW library is a C library that does fast Fourier transforms.
%
This is a brief introduction on how to use it.
%
The package can be downloaded from \url{http://www.fftw.org/download.html}.
%
In Ubuntu, simply type
\begin{verbatim}
  sudo apt-get install fftw3-dev
\end{verbatim}
to install it.
%
By default, FFTW uses double precision;
so, e.g., the declaration
\texttt{fftw\_complex z} below means \texttt{double z[2]}.



We use FFTW in a typical program as
%
\begin{enumerate}
  \item Allocate and initialize the input and output arrays.
  \item Make an FFT plan, which specifies the input and output arrays and their sizes.
  \item Execute the plan.
\end{enumerate}

So the code looks like this
{\footnotesize
\begin{verbatim}
  fftw_complex *in = (fftw_complex *) fftw_malloc(sizeof(*in) * N);
  fftw_complex *out = (fftw_complex *) fftw_malloc(sizeof(*out) * N);
  ...
  fftw_plan p = fftw_plan_dft_1d(N, in, out, FFTW_FORWARD, FFTW_ESTIMATE);
  ...
  fftw_execute(p);
\end{verbatim}
}

If you have written such a program, say \url{1d.c},
then compile it by typing
\begin{verbatim}
  gcc 1d.c -lfftw3 -lm
\end{verbatim}
and run it by typing \texttt{./a.out}.
%




\section{Different types of FFTs}

FFTW allows different types of fast Fourier transforms.
For example, we have 1-, 2-, 3-dimensional transformations;
the input array can be real or complex,
and it may be symmetric, etc.
%
Thus, we need to create different FFT plans for different FFTs.
%
Plans are always executed by
\texttt{fftw\_execute(p)}.

The advantage of the two-step process---creating an FFT plan first
  and then executing it
  (instead of executing FFT directly)---is, I suppose,
  to save time for multiple executions.

The list of plans can be found in:

\url{http://www.fftw.org/fftw3_doc/Basic-Interface.html}.

The mathematical meaning of different plans:

\url{http://www.fftw.org/fftw3_doc/What-FFTW-Really-Computes.html}

The tutorial covers the most common plans with examples:

\url{http://www.fftw.org/fftw3_doc/Tutorial.html}.


Below we try to enumerate different classes of transformations
and give an example code for each case,
where we compare the fast version provided by the library
with the straightforward version that follows directly from the definition.
%
The reader can use the code to
  see precisely what each transform does.
%
The file \url{usefftw.zip} contains all example codes.




\subsection{1D Fourier transform}

The one-dimensional complex Fourier transform computes
%
\begin{equation}
  Y_k = \sum_{j = 0}^{N - 1} X_j \exp(\mp 2 \pi j k i / N),
  \label{eq:fft1d}
\end{equation}
%
%


The plan is created by
\begin{verbatim}
  fftw_plan_dft_1d(N, X, Y, FFTW_FORWARD, FFTW_ESTIMATE);
\end{verbatim}
Here, input \texttt{X} and output \texttt{Y} may be the same variable,
  the flags \texttt{FFTW\_FORWORD} and \texttt{FFTW\_BACKWORD}
  correspond to the minus and plus signs in (\ref{eq:fft1d}).
%
Example code: \url{1d.c}.
%
An even shorter example \url{1dmin.c} is shown below.
Can you guess the output?
{\footnotesize
\begin{verbatim}
#include <fftw3.h>
int main(void) {
  fftw_complex a[5] = {{0,0},{1,0},{2,0},{3,0},{4,0}}; int i;
  fftw_plan p = fftw_plan_dft_1d(5, a, a, FFTW_FORWARD, FFTW_ESTIMATE);
  fftw_execute(p); fftw_execute(p); /* execute p twice */
  for (i = 0; i < 5; i++) printf("%d: %8.3f %8.3f\n", i, a[i][0], a[i][1]);
  fftw_destroy_plan(p); fftw_cleanup(); return 0;
}
\end{verbatim}
}
%



\subsection{1D real-to-complex Fourier transform}

If the input array is real (e.g., \texttt{double X[N];}).
%
The plan is created by
\begin{verbatim}
  fftw_plan_dft_r2c_1d(N, X, Y, FFTW_FORWARD, FFTW_ESTIMATE);
\end{verbatim}
%
The output array needs to hold \texttt{N/2 + 1} complex elements,
for $Y_{n - k} = Y_k^*$.
%
An example can be found in \url{1dr2c.c}.




\subsection{1D cosine and sine transforms}

The situation becomes complicated for cosine and sine transforms of
  a real array,
  and we have many different cases with subtle differences in definitions.
%
Fortunately they all share the same interface:
\begin{verbatim}
  fftw_plan_r2r_1d(N, X, Y, FFTW_REDFT00, FFTW_ESTIMATE);
\end{verbatim}
Here \texttt{N} is always the larger of
  the numbers of elements in \texttt{X} and \texttt{Y},
and
\texttt{FFTW\_REDFT00} can be replaced by one of the following options.
%
The cases listed below differ subtly in details,
  e.g. sometimes it's an $n$ instead of $n-1$,
  and sometimes there is a $1/2$, etc.





\subsubsection{Type-I cosine transform}
The type-I discrete cosine transform (DCT) computes
\begin{equation}
  Y_k = X_0 + (-1)^k X_{n}
  + 2 \sum_{j = 1}^{n - 1} X_j \cos(\pi j k / n)
\end{equation}
for $(X_0, \dots, X_n)$ and $(Y_0, \dots, Y_n)$.
%
The values $Y_{n+1}, \dots, Y_{2n}$ are unnecessary because $Y_{2n - k} = Y_k$.
%
The plan is created by
\begin{verbatim}
  fftw_plan_r2r_1d(n + 1, X, Y, FFTW_REDFT00, FFTW_ESTIMATE);
\end{verbatim}
Example code: \url{1dc1.c}.

This transform is its own inverse.
%
So the same transform on $(Y_0, \dots, Y_n)$
  yields $2n$ times $(X_0, \dots, X_n)$.
%
The output is a single peak if the input is of the form of
  $X_j = \cos(\pi k j/n)$
  with an integer frequency $k$.





\subsubsection{Type-II cosine transform}
The type-II DCT computes
\begin{equation}
  Y_k = 2 \sum_{j = 0}^{n - 1} X_j \cos\big(  \pi (j + 1/2) k / n  \big)
\end{equation}
%
for $(X_0, \dots, X_{n-1})$ and $(Y_0, \dots, Y_{n-1})$.
%
The values $Y_{n}, \dots, Y_{2n}$ are unnecessary because
$Y_{2n - k} = -Y_k$
and
$Y_n = 0$.
%
The plan is created by
\begin{verbatim}
  fftw_plan_r2r_1d(n, X, Y, FFTW_REDFT10, FFTW_ESTIMATE);
\end{verbatim}
Example code: \url{1dc2.c}.

This is the inverse of the type-III DCT.
%
So applying the type-III DCT on $(Y_0, \dots, Y_{n-1})$
  gives $2n$ times $(X_0, \dots, X_{n - 1})$.
%
The output is a single peak if the input is of the form of
  $X_j = \cos(\pi k (j + 1/2)/n)$
  with an integer frequency $k$.





\subsubsection{Type-III cosine transform}
The type-III DCT computes
\begin{equation}
  Y_k = X_0 + 2 \sum_{j = 1}^{n - 1} X_j \cos(\pi j (k + 1/2) / n)
\end{equation}
%
for $(X_0, \dots, X_{n-1})$ and $(Y_0, \dots, Y_{n-1})$.
%
The values $Y_{n}, \dots, Y_{2n - 1}$ are unnecessary because
$Y_{2n - 1 - k} = Y_k$.
%
The plan is created by
\begin{verbatim}
  fftw_plan_r2r_1d(n, X, Y, FFTW_REDFT01, FFTW_ESTIMATE);
\end{verbatim}
Example code: \url{1dc3.c}.

This is the inverse of the type-II DCT.
%
So applying the type-II DCT on $(Y_0, \dots, Y_{n-1})$
  gives $2n$ times $(X_0, \dots, X_{n - 1})$.
%
The output is a single peak if the input is of the form of
  $X_j = \cos(\pi (k + 1/2) j/n)$
  with a half-integer frequency $k + 1/2$.





\subsubsection{Type-IV cosine transform}
The type-IV DCT computes
\begin{equation}
  Y_k = 2 \sum_{j = 0}^{n - 1} X_j \cos\big( \pi (j + 1/2) (k + 1/2) / n \big)
\end{equation}
%
for $(X_0, \dots, X_{n-1})$ and $(Y_0, \dots, Y_{n-1})$.
%
The values $Y_{n}, \dots, Y_{2n - 1}$ are unnecessary because
$Y_{2n - 1 - k} = -Y_k$.
%
The plan is created by
\begin{verbatim}
  fftw_plan_r2r_1d(n, X, Y, FFTW_REDFT11, FFTW_ESTIMATE);
\end{verbatim}
Example code: \url{1dc4.c}.


This transform is its own inverse.
%
So the same transform on $(Y_0, \dots, Y_{n-1})$
  yields $2n$ times $(X_0, \dots, X_{n-1})$.
The output is a single peak if the input is of the form of
  $X_j = \cos(\pi (k + 1/2) (j + 1/2) / n)$
  with a half-integer frequency $k + 1/2$.





\subsubsection{Type-I sine transform}
The type-I DST computes
\begin{equation}
  Y_k = 2 \sum_{j = 1}^{n - 1} X_j \sin\big( \pi j k / n \big)
\end{equation}
%
for $(X_0 = 0, X_1, \dots, X_{n-1})$ and $(Y_0 = 0, Y_1, \dots, Y_{n-1})$.
%
The values $Y_{n}, \dots, Y_{2n}$ are unnecessary because
$Y_{2n - k} = Y_k$ and $Y_n = 0$.
%
The plan is created by
\begin{verbatim}
  fftw_plan_r2r_1d(n - 1, X + 1, Y + 1, FFTW_RODFT00, FFTW_ESTIMATE);
\end{verbatim}
where the offset ``+1'' skips the first zero elements.
Example code: \url{1ds1.c}.


This transform is its own inverse.
%
So the same transform on $(Y_0, \dots, Y_{n-1})$
  yields $2n$ times $(X_0, \dots, X_{n-1})$.
The output is a single peak if the input is of the form of
  $X_j = \sin\big(\pi k j / n \big)$
  with an integer frequency $k$.





\subsubsection{Type-II sine transform}
The type-II DST computes
\begin{equation}
  Y_k = 2 \sum_{j = 0}^{n - 1} X_j \sin\big( \pi (j + 1/2) k / n \big)
\end{equation}
%
for $(X_0, \dots, X_{n-1})$ and $(Y_0 = 0, Y_1, \dots, Y_{n-1})$.
%
The values $Y_{n}, \dots, Y_{2n}$ are unnecessary because
$Y_{2n - k} = -Y_k$ and $Y_n = 0$.
%
The plan is created by
\begin{verbatim}
  fftw_plan_r2r_1d(n, X, Y + 1, FFTW_RODFT10, FFTW_ESTIMATE);
\end{verbatim}
Example code: \url{1ds2.c}.


This transform is the inverse of the type-III DST.
%
So applying the type-III DST on $(Y_1, \dots, Y_{n-1})$
  yields $2n$ times $(X_0, \dots, X_{n-1})$.
The output is a single peak if the input is of the form of
  $X_j = \sin\big(\pi k (j + 1/2) / n \big)$
  with an integer frequency $k$.





\subsubsection{Type-III sine transform}
The type-III DST computes
\begin{equation}
  Y_k = (-1)^k X_n
    + 2 \sum_{j = 1}^{n - 1} X_j \sin\big( \pi j (k + 1/2) / n \big)
\end{equation}
%
for $(X_0 = 0, X_1, \dots, X_{n})$ and $(Y_0, \dots, Y_{n-1})$.
%
The values $Y_{n}, \dots, Y_{2n}$ are unnecessary because
$Y_{2n - 1 - k} = -Y_k$.
%
The plan is created by
\begin{verbatim}
  fftw_plan_r2r_1d(n, X + 1, Y, FFTW_RODFT01, FFTW_ESTIMATE);
\end{verbatim}
Example code: \url{1ds3.c}.


This transform is the inverse of the type-II DST.
%
So applying the type-II DST on $(Y_0, \dots, Y_{n-1})$
  yields $2n$ times $(X_1, \dots, X_{n})$.
The output is a single peak if the input is of the form of
  $X_j = \sin\big(\pi (k + 1/2) j / n \big)$
  with a half-integer frequency $k + 1/2$.





\subsubsection{Type-IV sine transform}
The type-IV DST computes
\begin{equation}
  Y_k = 2 \sum_{j = 0}^{n - 1} X_j \sin\big( \pi (j + 1/2) (k + 1/2) / n \big)
\end{equation}
%
for $(X_0, \dots, X_{n - 1})$ and $(Y_0, \dots, Y_{n-1})$.
%
The values $Y_{n}, \dots, Y_{2n}$ are unnecessary because
$Y_{2n - 1 - k} = Y_k$.
%
The plan is created by
\begin{verbatim}
  fftw_plan_r2r_1d(n, X, Y, FFTW_RODFT11, FFTW_ESTIMATE);
\end{verbatim}
Example code: \url{1ds4.c}.


This transform is its own inverse,
%
and applying the same transform on $(Y_0, \dots, Y_{n-1})$
  yields $2n$ times $(X_0, \dots, X_{n - 1})$.
The output is a single peak if the input is of the form of
  $X_j = \sin\big(\pi (k + 1/2) (j + 1/2) / n \big)$
  with a half-integer frequency $k + 1/2$.





\subsubsection{Hartley transform}
The Hartley computes
\begin{equation}
  Y_k = \sum_{j = 0}^{n - 1} X_j
    \left[
      \sin\big( 2 \pi j k / n \big)
      + \cos\big( 2 \pi j k / n \big)
    \right]
\end{equation}
%
for $(X_0, \dots, X_{n - 1})$ and $(Y_0, \dots, Y_{n-1})$.
%
The plan is created by
\begin{verbatim}
  fftw_plan_r2r_1d(n, X, Y, FFTW_DHT, FFTW_ESTIMATE);
\end{verbatim}
Example code: \url{1dh.c}.


This transform is its own inverse,
%
and applying the same transform on $(Y_0, \dots, Y_{n-1})$,
  yields $n$ times $(X_0, \dots, X_{n-1})$.
The output is a single peak if the input is of the form of
  $X_j = \sin\big(\pi k (j + \phi) / n \big)$
  with an integer frequency $k$
  and \emph{arbitrary} real phase $\phi$.





\subsection{3D Fourier transform}

The three-dimensional plan is created by
\begin{verbatim}
  fftw_plan_dft_3d(N0, N1, N2, &X[0][0][0], &Y[0][0][0],
    FFTW_FORWARD, FFTW_ESTIMATE);
\end{verbatim}
where the two arrays are declared as
\begin{verbatim}
  fftw_complex X[N0][N1][N2], Y[N0][N1][N2];
\end{verbatim}
%
Here, we use
\texttt{\&X[0][0][0]} (the input)
to get the one-dimensional address,
which is equivalent to
\texttt{(fftw\_complex *) X}.
%
Both arrays can be dynamically allocated
to hold $\mathtt{N0} \times \mathtt{N1} \times \mathtt{N2}$
complex elements.
%
See \url{3d.c} for an example.




\section{Thread and OpenMP}

There are two ways to parallelize your program.
%
One is to use threads.
%
Threads share the memory space,
  but hopefully uses different CPU cores.
%
The other is to use MPI (message passing interface),
%
in which each core has its own memory space,
%
with only ``messages'' exchanged among cores.
%
We discuss the former case here.





\subsection{Installation and compilation}

First, you may need to compile and install FFTW
  with \texttt{--enable-threads} and/or \texttt{--enable-openmp}.
If the compilation is successful,
  you should get \texttt{libfftw3\_threads.a}
  and \texttt{libfftw3\_omp.a}.



To compile your Pthread program with FFTW, type
\begin{verbatim}
  gcc a.c -lfftw3_threads -lm -lpthread
\end{verbatim}
%
For OpenMP, change it to
\begin{verbatim}
  gcc -fopenmp a.c -lfftw3_omp -lm
\end{verbatim}
If you use the Intel C compiler,
  substitute ``\texttt{icc -openmp}''
  for ``\texttt{gcc -fopenmp}.''
See also Sec. \ref{sec:install}.




\subsection{Using threads}

To use the thread version of FFTW, do the following three things.
(these are well explained in
\url{http://www.fftw.org/fftw3_doc/Usage-of-Multi_002dthreaded-FFTW.html}.)


First, call
\begin{verbatim}
  fftw_init_threads();
\end{verbatim}
at the beginning of the program, before calling any other FFTW functions.

Second, call
\begin{verbatim}
  fftw_plan_with_nthreads(nthreads);
\end{verbatim}
%
All subsequent plans are made with this many \texttt{nthreads} threads.
Setting \texttt{nthreads = 1} disables threads.
In the OpenMP case, the function \texttt{omp\_get\_max\_threads()}
can be used to obtain the number of threads.

Last, when you are done, call
\begin{verbatim}
  fftw_cleanup_threads();
\end{verbatim}
instead of \texttt{fftw\_cleanup()}.

The example code: \url{2d_t.c}.





\section{Message passing interface (MPI)}

An MPI program differs from a program using threads
in that each CPU core owns a private memory space in the former case.
%
So if you want to Fourier transform an array of $N$ elements
using $K$ processors.
%
Ideally each core should hold only $N/K$ elements.
%
Correspondingly, we have a different set functions,
%
and the reader should follow the documentation:
\url{http://www.fftw.org/fftw3_doc/Distributed_002dmemory-FFTW-with-MPI.html}.
%
Here we only give an example \url{2d_mpi.c} adapted from there.






\subsection{Installation and compilation}

First, you may need to compile and install FFTW
  with \texttt{--enable-mpi}
  and preferably with \texttt{--program-suffix=\_mpi}.
%
If the compilation is successful,
  you should get \texttt{libfftw3\_mpi.a}
  and \texttt{libfftw3.a}.

To compile your MPI program with FFTW, type
\begin{verbatim}
  mpicc 2d_mpi.c -lfftw3_mpi -lfftw3 -lm
\end{verbatim}
%
Note
1) Both ``\texttt{-lfftw3\_mpi}'' and ``\texttt{-lfftw3}''
  and needed, and place them in this order.
%
2) ``\texttt{-lm}'' may not needed if
``\texttt{mpicc}'' is not made of GCC (e.g., Intel C compiler).
See also Sec. \ref{sec:install}.





\section{\label{sec:install}Customized installation}


%\subsection{Building}
If you want a different setting than the default one,
  you need to compile FFTW directly.
Further, if you use FFTW on a supercomputer
  without the superuser privilege,
  you might want to install FFTW on a private directory.
Here is an example.

We want to compile the thread and OpenMP versions,
  which are independent, using GCC (the default compiler),
  and install them under \texttt{\$HOME/app}.


First, download and unzip the package.
\begin{verbatim}
  tar -xvzf fftw-3.3.3.tar.gz
  cd fftw-3.3.3
\end{verbatim}

Second, make a directory \texttt{build\_omp}, and build it there
\begin{verbatim}
  mkdir build_omp && cd build_omp
  ../configure --enable-threads --enable-omp --prefix=$HOME/app
  make install
\end{verbatim}

Now \texttt{ls \$HOME/app/lib} gives six files:
\begin{verbatim}
libfftw3.a  libfftw3_omp.a  libfftw3_threads.a
libfftw3.la libfftw3_omp.la libfftw3_threads.la
\end{verbatim}
We also have a header \texttt{\$HOME/app/include/fftw3.h}.


The MPI case is similar (we will make \texttt{libfftw3\_mpi.a}):
\begin{verbatim}
  mkdir build_mpi && cd build_mpi
  ../configure MPICC=mpicc --enable-mpi --prefix=$HOME/app \
      --program-suffix=_mpi
  make install
\end{verbatim}


%\subsection{Compilation}

Because the libraries are installed in a non-default directory,
add the flag ``\texttt{-I\$HOME/app/include -L\$HOME/app/lib}''
when compiling your program.
\begin{verbatim}
  mpicc your_mpi.c -lfftw3_mpi -lfftw3 -lm \
      -I$HOME/app/include -L$HOME/app/lib
\end{verbatim}
Also add \texttt{\$HOME/app/lib} to the environment variable
  ``\texttt{LD\_LIBRARY\_PATH}'':
\begin{verbatim}
  export LD_LIBRARY_PATH=$LD_LIBRARY_PATH:$HOME/app/lib
\end{verbatim}
so that the resulting program can run smoothly.



\section{Application: integral equation}


As an application, consider the solution of
  the integral equations in the liquid theory
  under the Percus-Yevick closure:
%
%
%
\newcommand{\FC}[1]{\tilde{#1}}
\newcommand{\vk}{\vct{k}}
\newcommand{\vr}{\vct{r}}
%
%
%
\begin{equation}
\begin{split}
  \FC{t}(\vk)
          &= \frac{ \rho \, \FC{c}^2(\vk) }
                  { 1 - \rho \, \FC{c}(\vk) } \\
  c(\vr)  &= f(r) \big[
                    1 + t(\vr)
                  \big],
\end{split}
\label{eq:PY}
\end{equation}
%
%
%
where $r = |\vr|$, $f(r) = e^{-\beta u(r)} - 1$ is a known function,
 and $A(\vk) \equiv \int A(\vr) e^{-i\vk\cdot\vr} d\vr$
 is the Fourier transform of the function $A(\vr)$.

To simply the calculation, we observe that
  the 3D Fourier transform of a spherical function $A(r)$
  can be reduced to a 1D sine transform:
\begin{equation}
  \int A(\vr) e^{-i\vk\cdot\vr} d\vr
  = \frac{4\pi}{k} \int_0^{+\infty} A(r) r \sin(kr) dr.
\end{equation}
%
The following code implements this trick
  and solves \eqref{eq:PY} by FFTW (type-I sine transform).
%
%
%
{\footnotesize
\begin{verbatim}
fftw_plan p = fftw_plan_r2r_1d(n-1, c+1, c+1, FFTW_RODFT00, FFTW_ESTIMATE);
fftw_plan q = fftw_plan_r2r_1d(n-1, t+1, t+1, FFTW_RODFT00, FFTW_ESTIMATE);

for ( iter = 0; iter < 1000; iter++ ) {
  for ( i = 1; i < n; i++ ) c[i] *= i*dr; /* c(r) r */
  fftw_execute(p); /* c(k) = 2 Pi/k Int 2r c(r) sin(kr) dr */
  for ( i = 1; i < n; i++ ) {
    c[i] *= 2*PI*dr/(i*dk); /* c(k) / k */
    t[i] = rho*c[i]*c[i] / (1 - rho*c[i]) * (i*dk); /* t(k) k */
  }
  fftw_execute(q); /* t(r) = 1/(4 Pi^2 r) Int 2k t(k) sin(kr) dk */
  for ( i = 1; i < n; i++ ) {
    t[i] *= dk/(4*PI*PI*i*dr); /* t(r) / r */
    c[i] = f[i] * (1 + t[i]); /* PY closure */
  }
}
\end{verbatim}
}
%
%
%
The full code: \url{inteq.c}.


\end{document}


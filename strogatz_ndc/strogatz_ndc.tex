\documentclass{book}
\usepackage{amsmath}
\begin{document}


\chapter{5. Linear Systems}

\section{5.1 Definition and Examples}

The set of equations we are interested in is
$$
\begin{aligned}
  \mathbf {\dot x}
  &= \mathbf f(\mathbf x) \\
  &= \mathbf A \mathbf x.
\end{aligned}
$$
The $n\times n$ matrix $\mathbf A$ always has $n$ eigenvalues,
but not always have $n$ eigenvectors (see degenerate node below).
But if there are $n$ eigenvectors,
we can diagonalizing the matrix by combinations of $x_i$,
i.e., for each $j = 1, \dots, n$, we have
$$
\dot v_j = \lambda_j v_j,
$$
where $v_j = \sum_{j = 1}^n c_{ji} x_i$.

\begin{itemize}

\item
Fixed point. A fixed point, $\mathbf x^*$, is the solution of
$$
\mathbf f(\mathbf x^*) = 0.
$$
In our case, $\mathbf x^* = \mathbf 0$ is the origin.
\\

\item
Node.
If all eigenvalues share the same sign, then the fixed point is a \emph{node}.
If the sign is positive, it unstable; negative, stable.
\\

\item
Star.
A special node with all eigenvalues being identical is a \emph{star}.

\item
Saddle points.
If the eigenvalues are all real, but some are positive, some are negative, the fixed point is saddle.
The space spanned the set of eigenvectors of the negative/positive eigenvalues are called the \emph{stable/unstable manifold}.

\item
Spiral.
In two dimensions, $n = 2$,
if the eigenvalues are complex numbers, then the fixed point is a spiral.

\item
Center.
A special spiral with purely imaginary eigenvalues is a center. For example,
\begin{align*}
  \dot x &= y \\
  \dot y &= -x,
\end{align*}
has two eigenvalues $\lambda = \pm i$.

\end{itemize}


\subsection{Stability Language}


\begin{itemize}

\item
Attracting.
If any trajectory starting from somewhere near the fixed point, $\mathbf x^*$, ultimately approaches $\mathbf x^*$, then $\mathbf x^*$ is attracting.

\item
Globally attracting.
If any trajectory starting from anywhere ultimately approaches $\mathbf x^*$, then $\mathbf x^*$ is \emph{globally attracting}.

\item
Lyapunov stable.
A fixed point, $\mathbf x^*$, is \emph{Lyapunov stable} if all trajectories starting close to $\mathbf x^*$ remain close to it at all times.

\item
\emph{Neutrally stable} means Lyapunov stable but not attracting.  An example is a center.

\item
\emph{Asymptotically stable} means Lyapunov stable and attracting.

\end{itemize}





\end{document}
